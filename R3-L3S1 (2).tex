% Options for packages loaded elsewhere
\PassOptionsToPackage{unicode}{hyperref}
\PassOptionsToPackage{hyphens}{url}
%
\documentclass[
  11pt,
  french,
]{article}
\usepackage{lmodern}
\usepackage{amssymb,amsmath}
\usepackage{ifxetex,ifluatex}
\ifnum 0\ifxetex 1\fi\ifluatex 1\fi=0 % if pdftex
  \usepackage[T1]{fontenc}
  \usepackage[utf8]{inputenc}
  \usepackage{textcomp} % provide euro and other symbols
\else % if luatex or xetex
  \usepackage{unicode-math}
  \defaultfontfeatures{Scale=MatchLowercase}
  \defaultfontfeatures[\rmfamily]{Ligatures=TeX,Scale=1}
\fi
% Use upquote if available, for straight quotes in verbatim environments
\IfFileExists{upquote.sty}{\usepackage{upquote}}{}
\IfFileExists{microtype.sty}{% use microtype if available
  \usepackage[]{microtype}
  \UseMicrotypeSet[protrusion]{basicmath} % disable protrusion for tt fonts
}{}
\makeatletter
\@ifundefined{KOMAClassName}{% if non-KOMA class
  \IfFileExists{parskip.sty}{%
    \usepackage{parskip}
  }{% else
    \setlength{\parindent}{0pt}
    \setlength{\parskip}{6pt plus 2pt minus 1pt}}
}{% if KOMA class
  \KOMAoptions{parskip=half}}
\makeatother
\usepackage{xcolor}
\IfFileExists{xurl.sty}{\usepackage{xurl}}{} % add URL line breaks if available
\IfFileExists{bookmark.sty}{\usepackage{bookmark}}{\usepackage{hyperref}}
\hypersetup{
  pdftitle={R3-L3S1 Le genre},
  pdfauthor={Lola LUBIN et Alexis VIALATTE},
  pdflang={fr},
  hidelinks,
  pdfcreator={LaTeX via pandoc}}
\urlstyle{same} % disable monospaced font for URLs
\usepackage[a4paper, top = 2cm, bottom = 2cm, left = 1.5cm, right = 1.5cm]{geometry}
\usepackage{color}
\usepackage{fancyvrb}
\newcommand{\VerbBar}{|}
\newcommand{\VERB}{\Verb[commandchars=\\\{\}]}
\DefineVerbatimEnvironment{Highlighting}{Verbatim}{commandchars=\\\{\}}
% Add ',fontsize=\small' for more characters per line
\usepackage{framed}
\definecolor{shadecolor}{RGB}{248,248,248}
\newenvironment{Shaded}{\begin{snugshade}}{\end{snugshade}}
\newcommand{\AlertTok}[1]{\textcolor[rgb]{0.94,0.16,0.16}{#1}}
\newcommand{\AnnotationTok}[1]{\textcolor[rgb]{0.56,0.35,0.01}{\textbf{\textit{#1}}}}
\newcommand{\AttributeTok}[1]{\textcolor[rgb]{0.77,0.63,0.00}{#1}}
\newcommand{\BaseNTok}[1]{\textcolor[rgb]{0.00,0.00,0.81}{#1}}
\newcommand{\BuiltInTok}[1]{#1}
\newcommand{\CharTok}[1]{\textcolor[rgb]{0.31,0.60,0.02}{#1}}
\newcommand{\CommentTok}[1]{\textcolor[rgb]{0.56,0.35,0.01}{\textit{#1}}}
\newcommand{\CommentVarTok}[1]{\textcolor[rgb]{0.56,0.35,0.01}{\textbf{\textit{#1}}}}
\newcommand{\ConstantTok}[1]{\textcolor[rgb]{0.00,0.00,0.00}{#1}}
\newcommand{\ControlFlowTok}[1]{\textcolor[rgb]{0.13,0.29,0.53}{\textbf{#1}}}
\newcommand{\DataTypeTok}[1]{\textcolor[rgb]{0.13,0.29,0.53}{#1}}
\newcommand{\DecValTok}[1]{\textcolor[rgb]{0.00,0.00,0.81}{#1}}
\newcommand{\DocumentationTok}[1]{\textcolor[rgb]{0.56,0.35,0.01}{\textbf{\textit{#1}}}}
\newcommand{\ErrorTok}[1]{\textcolor[rgb]{0.64,0.00,0.00}{\textbf{#1}}}
\newcommand{\ExtensionTok}[1]{#1}
\newcommand{\FloatTok}[1]{\textcolor[rgb]{0.00,0.00,0.81}{#1}}
\newcommand{\FunctionTok}[1]{\textcolor[rgb]{0.00,0.00,0.00}{#1}}
\newcommand{\ImportTok}[1]{#1}
\newcommand{\InformationTok}[1]{\textcolor[rgb]{0.56,0.35,0.01}{\textbf{\textit{#1}}}}
\newcommand{\KeywordTok}[1]{\textcolor[rgb]{0.13,0.29,0.53}{\textbf{#1}}}
\newcommand{\NormalTok}[1]{#1}
\newcommand{\OperatorTok}[1]{\textcolor[rgb]{0.81,0.36,0.00}{\textbf{#1}}}
\newcommand{\OtherTok}[1]{\textcolor[rgb]{0.56,0.35,0.01}{#1}}
\newcommand{\PreprocessorTok}[1]{\textcolor[rgb]{0.56,0.35,0.01}{\textit{#1}}}
\newcommand{\RegionMarkerTok}[1]{#1}
\newcommand{\SpecialCharTok}[1]{\textcolor[rgb]{0.00,0.00,0.00}{#1}}
\newcommand{\SpecialStringTok}[1]{\textcolor[rgb]{0.31,0.60,0.02}{#1}}
\newcommand{\StringTok}[1]{\textcolor[rgb]{0.31,0.60,0.02}{#1}}
\newcommand{\VariableTok}[1]{\textcolor[rgb]{0.00,0.00,0.00}{#1}}
\newcommand{\VerbatimStringTok}[1]{\textcolor[rgb]{0.31,0.60,0.02}{#1}}
\newcommand{\WarningTok}[1]{\textcolor[rgb]{0.56,0.35,0.01}{\textbf{\textit{#1}}}}
\usepackage{graphicx,grffile}
\makeatletter
\def\maxwidth{\ifdim\Gin@nat@width>\linewidth\linewidth\else\Gin@nat@width\fi}
\def\maxheight{\ifdim\Gin@nat@height>\textheight\textheight\else\Gin@nat@height\fi}
\makeatother
% Scale images if necessary, so that they will not overflow the page
% margins by default, and it is still possible to overwrite the defaults
% using explicit options in \includegraphics[width, height, ...]{}
\setkeys{Gin}{width=\maxwidth,height=\maxheight,keepaspectratio}
% Set default figure placement to htbp
\makeatletter
\def\fps@figure{htbp}
\makeatother
\setlength{\emergencystretch}{3em} % prevent overfull lines
\providecommand{\tightlist}{%
  \setlength{\itemsep}{0pt}\setlength{\parskip}{0pt}}
\setcounter{secnumdepth}{5}
\usepackage{tikz}
\usepackage{booktabs}
\usepackage{longtable}
\usepackage{array}
\usepackage{multirow}
\usepackage{wrapfig}
\usepackage{float}
\usepackage{colortbl}
\usepackage{pdflscape}
\usepackage{tabu}
\usepackage{threeparttable}
\usepackage{threeparttablex}
\usepackage[normalem]{ulem}
\usepackage{makecell}
\ifxetex
  % Load polyglossia as late as possible: uses bidi with RTL langages (e.g. Hebrew, Arabic)
  \usepackage{polyglossia}
  \setmainlanguage[]{french}
\else
  \usepackage[shorthands=off,main=french]{babel}
\fi

\title{\texttt{R3-L3S1} Le genre}
\author{Lola LUBIN et Alexis VIALATTE}
\date{MAJ : 16/12/2020}

\begin{document}
\maketitle

{
\setcounter{tocdepth}{2}
\tableofcontents
}
\newpage

\hypertarget{support}{%
\section{Support :}\label{support}}

\hypertarget{la-mise-en-place}{%
\subsection{La mise en place :}\label{la-mise-en-place}}

Nous allons nous servir de ces \texttt{packages} :

\begin{Shaded}
\begin{Highlighting}[]
\KeywordTok{library}\NormalTok{( readr )}\CommentTok{# pour l'import de la base de données,}
\KeywordTok{library}\NormalTok{( DT )}\CommentTok{# pour la base de données,}
\KeywordTok{library}\NormalTok{( knitr )}\CommentTok{# pour les tableaux,}
\KeywordTok{library}\NormalTok{( kableExtra )}\CommentTok{# pour les tableaux,}
\KeywordTok{library}\NormalTok{( gtsummary )}\CommentTok{# pour les tableaux,}
\CommentTok{#library( printr )# pour les tableaux à trois variables,}
\CommentTok{# Problème entre `printr` et `knitr` lors du 'knit' en PDF...}
\KeywordTok{library}\NormalTok{( modelsummary )}\CommentTok{# pour les tableaux automatiques,}
\KeywordTok{library}\NormalTok{( dplyr )}\CommentTok{# pour la syntaxe des tableaux,}
\KeywordTok{library}\NormalTok{( ggplot2 )}\CommentTok{# pour les figures,}
\KeywordTok{library}\NormalTok{( ggridges )}\CommentTok{# pour les figures,}
\KeywordTok{library}\NormalTok{( ggpubr )}\CommentTok{# pour les figures,}

\KeywordTok{library}\NormalTok{( tidyr )}\CommentTok{# pour `drop_na` et `pivot_*` notamment,}
\KeywordTok{library}\NormalTok{( forcats )}\CommentTok{# pour les levels notamment,}
\end{Highlighting}
\end{Shaded}

Nous allons nous servir de ces \texttt{function} :

\begin{enumerate}
\def\labelenumi{\arabic{enumi}.}
\tightlist
\item
  Pour les moyennes des enseignements de Macro, de Micro, de Maths, de
  Stat et d'Anglais.
\end{enumerate}

\begin{Shaded}
\begin{Highlighting}[]
\NormalTok{tool1 <-}\StringTok{ }\ControlFlowTok{function}\NormalTok{( x, y )\{}
\NormalTok{  z <-}\StringTok{ }\NormalTok{( x }\OperatorTok{+}\StringTok{ }\NormalTok{y ) }\OperatorTok{/}\StringTok{ }\DecValTok{2}
\NormalTok{  z <-}\StringTok{ }\KeywordTok{round}\NormalTok{( z,}
              \DecValTok{2}\NormalTok{ )}
  \KeywordTok{return}\NormalTok{( z )}
\NormalTok{\}}\CommentTok{# Moyenne des notes.}
\end{Highlighting}
\end{Shaded}

\begin{enumerate}
\def\labelenumi{\arabic{enumi}.}
\setcounter{enumi}{1}
\tightlist
\item
  Pour l'évolution de la moyenne enntre le semestre 2 et le semestre 1
  (en pourcentage).
\end{enumerate}

\begin{Shaded}
\begin{Highlighting}[]
\NormalTok{tool2 <-}\StringTok{ }\ControlFlowTok{function}\NormalTok{( x, y )\{}
\NormalTok{  z <-}\StringTok{ }\NormalTok{( x }\OperatorTok{-}\StringTok{ }\NormalTok{y ) }\OperatorTok{/}\StringTok{ }\NormalTok{y }\OperatorTok{*}\StringTok{ }\DecValTok{100}
\NormalTok{  z <-}\StringTok{ }\KeywordTok{round}\NormalTok{( z,}
              \DecValTok{2}\NormalTok{ )}
  \KeywordTok{return}\NormalTok{( z )}
\NormalTok{\}}\CommentTok{# L'évolution des moyennes en poucentage,}
\end{Highlighting}
\end{Shaded}

\begin{enumerate}
\def\labelenumi{\arabic{enumi}.}
\setcounter{enumi}{2}
\tightlist
\item
  Pour les tableaux.
\end{enumerate}

\begin{Shaded}
\begin{Highlighting}[]
\NormalTok{tool3 <-}\StringTok{ }\ControlFlowTok{function}\NormalTok{( x, y, a )\{}
\NormalTok{  z <-}\StringTok{ }\NormalTok{x }\OperatorTok
\StringTok{    }\KeywordTok{kable}\NormalTok{( }\DataTypeTok{caption =}\NormalTok{ y,}
           \DataTypeTok{digits =} \DecValTok{2}\NormalTok{,}
           \DataTypeTok{booktabs =}\NormalTok{ T ) }\OperatorTok
\StringTok{    }\KeywordTok{kable_styling}\NormalTok{( }\DataTypeTok{full_width =}\NormalTok{ F,}
                   \DataTypeTok{position =} \StringTok{"center"}\NormalTok{,}
                   \DataTypeTok{latex_options =} \KeywordTok{c}\NormalTok{( }\StringTok{"striped"}\NormalTok{,}
                                      \StringTok{"condensed"}\NormalTok{,}
                                      \StringTok{"hold_position"}\NormalTok{,}
\NormalTok{                                      a ) )}
  \KeywordTok{return}\NormalTok{( z )}
\NormalTok{\}}\CommentTok{# Tableaux,}
\end{Highlighting}
\end{Shaded}

\begin{enumerate}
\def\labelenumi{\arabic{enumi}.}
\setcounter{enumi}{3}
\tightlist
\item
  Pour les résumés.
\end{enumerate}

\begin{Shaded}
\begin{Highlighting}[]
\NormalTok{tool4 <-}\StringTok{ }\ControlFlowTok{function}\NormalTok{( x )\{}
\NormalTok{  z <-}\StringTok{ }\KeywordTok{c}\NormalTok{( }\KeywordTok{mean}\NormalTok{( x,}
                \DataTypeTok{na.rm =}\NormalTok{ T ),}
          \KeywordTok{sd}\NormalTok{( x,}
              \DataTypeTok{na.rm =}\NormalTok{ T ),}
          \KeywordTok{quantile}\NormalTok{( x,}
                    \DataTypeTok{na.rm =}\NormalTok{ T ) )}
\NormalTok{  z <-}\StringTok{ }\KeywordTok{round}\NormalTok{( z,}
              \DecValTok{2}\NormalTok{ )}
  \KeywordTok{names}\NormalTok{( z ) <-}\StringTok{ }\KeywordTok{c}\NormalTok{( }\StringTok{"Moyenne"}\NormalTok{,}
                   \StringTok{"Ecart-type"}\NormalTok{,}
                   \StringTok{"Minimum"}\NormalTok{,}
                   \StringTok{"Q1"}\NormalTok{,}
                   \StringTok{"Q2"}\NormalTok{,}
                   \StringTok{"Q3"}\NormalTok{,}
                   \StringTok{"Maximum"}\NormalTok{ )}
  \KeywordTok{return}\NormalTok{( z )}
\NormalTok{\}}\CommentTok{# Résumés,}
\end{Highlighting}
\end{Shaded}

\begin{enumerate}
\def\labelenumi{\arabic{enumi}.}
\setcounter{enumi}{4}
\tightlist
\item
  Pour les tests d'appartenances :
  \(X\rightsquigarrow\mathcal{N}(\mu,\sigma)\).
\end{enumerate}

\begin{Shaded}
\begin{Highlighting}[]
\NormalTok{tool5 <-}\StringTok{ }\ControlFlowTok{function}\NormalTok{( x )\{}
\NormalTok{  test1 <-}\StringTok{ }\KeywordTok{shapiro.test}\NormalTok{( x )}
\NormalTok{  z <-}\StringTok{ }\NormalTok{test1}\OperatorTok{$}\NormalTok{p.value}
\NormalTok{  z <-}\StringTok{ }\KeywordTok{round}\NormalTok{( z,}
              \DecValTok{2}\NormalTok{ )}
  \KeywordTok{names}\NormalTok{( z ) <-}\StringTok{ "p.value"}
  \KeywordTok{return}\NormalTok{( z )}
\NormalTok{\}}\CommentTok{# Shapiro,}
\end{Highlighting}
\end{Shaded}

\begin{enumerate}
\def\labelenumi{\arabic{enumi}.}
\setcounter{enumi}{5}
\tightlist
\item
  Pour les tests de Student.
\end{enumerate}

\begin{Shaded}
\begin{Highlighting}[]
\NormalTok{tool6 <-}\StringTok{ }\ControlFlowTok{function}\NormalTok{( x, y )\{}
\NormalTok{ test2 <-}\StringTok{ }\KeywordTok{var.test}\NormalTok{( x}\OperatorTok{~}\NormalTok{y )}
\NormalTok{ test3 <-}\StringTok{ }\KeywordTok{t.test}\NormalTok{( x}\OperatorTok{~}\NormalTok{y,}
                  \DataTypeTok{equal =}\NormalTok{ test2}\OperatorTok{$}\NormalTok{p.value }\OperatorTok{>}\StringTok{ }\FloatTok{0.05}\NormalTok{ )}
\NormalTok{ z <-}\StringTok{ }\KeywordTok{c}\NormalTok{( }\KeywordTok{table}\NormalTok{( y[ }\OperatorTok{!}\KeywordTok{is.na}\NormalTok{( x ) ] ),}
         \KeywordTok{ifelse}\NormalTok{( test3}\OperatorTok{$}\NormalTok{p.value }\OperatorTok{>}\StringTok{ }\FloatTok{0.05}\NormalTok{,}
                 \StringTok{"Oui"}\NormalTok{,}
                 \StringTok{"Non"}\NormalTok{ ),}
         \KeywordTok{round}\NormalTok{( }\KeywordTok{c}\NormalTok{( test3}\OperatorTok{$}\NormalTok{estimate,}
\NormalTok{                   test3}\OperatorTok{$}\NormalTok{p.value ),}
                \DecValTok{2}\NormalTok{ ) )}
 \KeywordTok{names}\NormalTok{( z ) <-}\StringTok{ }\KeywordTok{c}\NormalTok{( }\KeywordTok{names}\NormalTok{( }\KeywordTok{table}\NormalTok{( y[ }\OperatorTok{!}\KeywordTok{is.na}\NormalTok{(x) ] ) ),}
                  \StringTok{"var.equal"}\NormalTok{,}
                  \KeywordTok{names}\NormalTok{( test3}\OperatorTok{$}\NormalTok{estimate ),}
                  \StringTok{"p.value"}\NormalTok{ )}
 \KeywordTok{return}\NormalTok{( z )}
\NormalTok{\}}\CommentTok{# Student}
\end{Highlighting}
\end{Shaded}

\begin{enumerate}
\def\labelenumi{\arabic{enumi}.}
\setcounter{enumi}{6}
\tightlist
\item
  Pour les tests d'indépendances.
\end{enumerate}

\begin{Shaded}
\begin{Highlighting}[]
\NormalTok{tool7 <-}\StringTok{ }\ControlFlowTok{function}\NormalTok{( x, y )\{}
\NormalTok{  test4 <-}\StringTok{ }\KeywordTok{chisq.test}\NormalTok{( x,}
\NormalTok{                       y )}
\NormalTok{  z <-}\StringTok{ }\KeywordTok{c}\NormalTok{( }\KeywordTok{min}\NormalTok{( test4}\OperatorTok{$}\NormalTok{expected ),}
\NormalTok{          test4}\OperatorTok{$}\NormalTok{p.value )}
\NormalTok{  z <-}\StringTok{ }\KeywordTok{round}\NormalTok{( z,}
              \DecValTok{2}\NormalTok{ )}
  \KeywordTok{names}\NormalTok{( z ) <-}\StringTok{ }\KeywordTok{c}\NormalTok{( }\StringTok{"Eff_théorique_min"}\NormalTok{,}
                   \StringTok{"p-value"}\NormalTok{ )}
  \KeywordTok{return}\NormalTok{( z )}
\NormalTok{\}}\CommentTok{# Chi 2,}
\end{Highlighting}
\end{Shaded}

\begin{enumerate}
\def\labelenumi{\arabic{enumi}.}
\setcounter{enumi}{7}
\tightlist
\item
  Autres.
\end{enumerate}

\begin{Shaded}
\begin{Highlighting}[]
\NormalTok{tool01 <-}\StringTok{ }\ControlFlowTok{function}\NormalTok{( x, y )\{}
\NormalTok{  z <-}\StringTok{ }\NormalTok{x[ x }\OperatorTok{==}\StringTok{ }\NormalTok{y ]}
  \KeywordTok{return}\NormalTok{( z )}
\NormalTok{\}}\CommentTok{# Aucun,}

\NormalTok{tool02 <-}\StringTok{ }\ControlFlowTok{function}\NormalTok{( x, y )\{}
\NormalTok{  z <-}\StringTok{ }\KeywordTok{round}\NormalTok{( }\KeywordTok{mean}\NormalTok{ ( x[ x }\OperatorTok{<}\StringTok{ }\NormalTok{y ] ),}
              \DecValTok{2}\NormalTok{ )}
  \KeywordTok{return}\NormalTok{( z )}
\NormalTok{\}}\CommentTok{# Regroupement des moyennes,}

\NormalTok{tool03 <-}\StringTok{ }\ControlFlowTok{function}\NormalTok{( x, y, a )\{}
\NormalTok{  z <-}\StringTok{ }\KeywordTok{round}\NormalTok{( }\KeywordTok{mean}\NormalTok{ ( x[ x }\OperatorTok{<}\StringTok{ }\NormalTok{y }\OperatorTok{&}\StringTok{ }\NormalTok{x }\OperatorTok{>}\StringTok{ }\NormalTok{a ] ),}
              \DecValTok{2}\NormalTok{ )}
  \KeywordTok{return}\NormalTok{( z )}
\NormalTok{\}}\CommentTok{# Regroupement des moyennes,}
\end{Highlighting}
\end{Shaded}

\newpage

\begin{figure}
\centering
\includegraphics[width=0.5\textwidth,height=\textheight]{lecot (2).png}
\caption{Le genre dans la licence d'Économie à Tours.}
\end{figure}

\hypertarget{r3-enquete-l3.csv}{%
\section{\texorpdfstring{\texttt{R3-enquete-L3.csv}
:}{R3-enquete-L3.csv :}}\label{r3-enquete-l3.csv}}

\begin{quote}
\emph{Nous avons entendu de nombreuses fois que le genre, par des
méchanismes qui échappent à notre champ d'étude au sens strict,
``\textbf{influence}'' nos \textbf{choix} et nos
\textbf{résultats}.\textsuperscript{1}}
\end{quote}

Nous allons analyser les \textbf{choix} et les \textbf{résulats} des
étudiants de la licence d'Économie à Tours grâce à la base de donné
contenu dans \texttt{R3-enquete-L3.csv}.

Le but de cette analyse est d'essayer de répondre à ces trois questions
:

\begin{enumerate}
\def\labelenumi{\arabic{enumi}.}
\tightlist
\item
  La motivation diffère-t-elle entre les femmes et les hommes ?
\item
  Les moyennes dépendent-elles du genre ?
\item
  Le niveau d'éducation visé diffère-t-il entre les femmes et les hommes
  ?
\end{enumerate}

Nous allons formuler trois hypothèses :

\begin{enumerate}
\def\labelenumi{\arabic{enumi}.}
\tightlist
\item
  Les femmes sont plus motivées que les hommes.
\item
  Les femmes ont de meilleurs résultats que les hommes.
\item
  Les femmes veulent faire des études plus longues que les hommes.
\end{enumerate}

Les variables que nous avons choisi sont :

\begin{itemize}
\tightlist
\item
  pour les \textbf{résultats} :

  \begin{itemize}
  \tightlist
  \item
    la motivation,
  \item
    l'évolution de la moyenne entre le semestre 1 et le semestre 2 (en
    pourcentage),
  \item
    la moyenne générale,
  \item
    les moyennes des enseignements de Macro, de Micro, de Maths, de Stat
    et d'Anglais,
  \end{itemize}
\item
  pour les \textbf{choix} :

  \begin{itemize}
  \tightlist
  \item
    le niveau d'éducation visé.
  \end{itemize}
\end{itemize}

\begin{center}\rule{0.5\linewidth}{0.5pt}\end{center}

\textsuperscript{1} Revues de sociologie et de psychologie, manuels
d'économie et de sociologie du lycée.

\hypertarget{avant-de-commencer}{%
\subsection{Avant de commencer :}\label{avant-de-commencer}}

\hypertarget{les-individus-et-les-variables-uxe0-analyser}{%
\subsubsection{Les individus et les variables à analyser
:}\label{les-individus-et-les-variables-uxe0-analyser}}

\begin{enumerate}
\def\labelenumi{\arabic{enumi}.}
\tightlist
\item
  Pour l'import de \texttt{R3-enquete-L3.csv} et de
  \texttt{licence-2019\_2020.csv}.
\end{enumerate}

\begin{Shaded}
\begin{Highlighting}[]
\NormalTok{data1 <-}\StringTok{ }\KeywordTok{read_delim}\NormalTok{( }\StringTok{"R3-enquete-L3.csv"}\NormalTok{,}
                     \DataTypeTok{delim =} \StringTok{";"}\NormalTok{ )}\CommentTok{# Première base de données,}
\NormalTok{data2 <-}\StringTok{ }\KeywordTok{read_delim}\NormalTok{( }\StringTok{"licence-2019_2020.csv"}\NormalTok{,}
                     \DataTypeTok{delim =} \StringTok{";"}\NormalTok{ )}\CommentTok{# Deuxième base de données,}
\end{Highlighting}
\end{Shaded}

\begin{enumerate}
\def\labelenumi{\arabic{enumi}.}
\setcounter{enumi}{1}
\tightlist
\item
  Pour l'usage de \texttt{data}.
\end{enumerate}

\begin{Shaded}
\begin{Highlighting}[]
\NormalTok{data <-}\StringTok{ }\NormalTok{data1[ , }\OperatorTok{-}\KeywordTok{c}\NormalTok{( }\DecValTok{1}\NormalTok{,}
                     \DecValTok{3}\OperatorTok{:}\DecValTok{17}\NormalTok{,}
                     \DecValTok{19}\NormalTok{,}
                     \DecValTok{21}\OperatorTok{:}\DecValTok{29}\NormalTok{,}
                     \DecValTok{30}\OperatorTok{:}\DecValTok{38}\NormalTok{,}
                     \DecValTok{40}\NormalTok{ ) ]}\CommentTok{# Base de données "finale",}
\KeywordTok{colnames}\NormalTok{( data ) <-}\StringTok{ }\KeywordTok{c}\NormalTok{( }\StringTok{"Genre"}\NormalTok{,}
                       \StringTok{"Motivation"}\NormalTok{,}
                       \StringTok{"Niveau_edu_vise"}\NormalTok{,}
                       \StringTok{"Note"}\NormalTok{ )}\CommentTok{# Noms de variables "finaux",}
\end{Highlighting}
\end{Shaded}

\begin{table}[!h]

\caption{\label{tab:table1}Base de données sans les nouvelles variables :}
\centering
\begin{tabular}[t]{lrlr}
\toprule
Genre & Motivation & Niveau\_edu\_vise & Note\\
\midrule
\cellcolor{gray!6}{Homme} & \cellcolor{gray!6}{8} & \cellcolor{gray!6}{BAC+5 ET +} & \cellcolor{gray!6}{15.53}\\
\bottomrule
\end{tabular}
\end{table}

\begin{table}[!h]

\caption{\label{tab:table2}Base de données avec les nouvelles variables :}
\centering
\resizebox{\linewidth}{!}{
\begin{tabular}[t]{lrlrrrrrrrl}
\toprule
Genre & Motivation & Niveau\_edu\_vise & Note & Micro & Macro & Stat & Maths & Anglais & Evo\_note & Genre\_motivation\\
\midrule
\cellcolor{gray!6}{Homme} & \cellcolor{gray!6}{8} & \cellcolor{gray!6}{BAC+5 ET +} & \cellcolor{gray!6}{15.53} & \cellcolor{gray!6}{16.15} & \cellcolor{gray!6}{18.6} & \cellcolor{gray!6}{17.5} & \cellcolor{gray!6}{9.85} & \cellcolor{gray!6}{14.5} & \cellcolor{gray!6}{4.61} & \cellcolor{gray!6}{Homme\_8}\\
\bottomrule
\end{tabular}}
\end{table}

Nous avons créé les variables des moyennes des enseignements de Macro,
de Micro, de Maths, de Stat et d'Anglais, de l'évolution entre la
moyenne du semestre 1 et la moyenne du semestre 2 (en pourcentage) et du
genre selon la motivation. Il y a 146 individus et 11 variables dans
\texttt{data}. Il y a 0 donnée manquante concernant le genre, la
motivation et le niveau d'éducation visé. Il y a 665 données manquantes
concernant 103 individus et 7 variables.

\hypertarget{lanalyse-des-ruxe9pondants}{%
\subsubsection{L'analyse des répondants
:}\label{lanalyse-des-ruxe9pondants}}

\begin{table}[!h]

\caption{\label{tab:table3}Les répondants et le taux de répondants par année :}
\centering
\begin{tabular}[t]{lrrr}
\toprule
  & Étudiants & Répondants & Taux de répondants (en pourcentage)\\
\midrule
\cellcolor{gray!6}{L1} & \cellcolor{gray!6}{264} & \cellcolor{gray!6}{70} & \cellcolor{gray!6}{26.52}\\
L2 & 95 & 36 & 37.89\\
\cellcolor{gray!6}{L3} & \cellcolor{gray!6}{65} & \cellcolor{gray!6}{40} & \cellcolor{gray!6}{61.54}\\
\bottomrule
\end{tabular}
\end{table}

\begin{table}[!h]

\caption{\label{tab:table4}Le genre des répondants et du taux de répondants par année  :}
\centering
\resizebox{\linewidth}{!}{
\begin{tabular}[t]{lrrrrrr}
\toprule
  & Femme & Homme & Répondants femme & Répondants homme & Taux de répondants femme (en pourcentage) & Taux de répondants homme (en pourcentage)\\
\midrule
\cellcolor{gray!6}{L1} & \cellcolor{gray!6}{75} & \cellcolor{gray!6}{189} & \cellcolor{gray!6}{33} & \cellcolor{gray!6}{37} & \cellcolor{gray!6}{44.00} & \cellcolor{gray!6}{19.58}\\
L2 & 40 & 55 & 17 & 19 & 42.50 & 34.55\\
\cellcolor{gray!6}{L3} & \cellcolor{gray!6}{26} & \cellcolor{gray!6}{39} & \cellcolor{gray!6}{15} & \cellcolor{gray!6}{25} & \cellcolor{gray!6}{57.69} & \cellcolor{gray!6}{64.10}\\
\bottomrule
\end{tabular}}
\end{table}

\begin{center}\includegraphics{R3-L3S1--2-_files/figure-latex/figure1, figure2, figure3 and figure4-1} \end{center}

Nous pouvons constater que le taux de répondants des femmes est
largement supérieur à celui des hommes en L1 et en L2 mais pas en L3.
Nous devons prendre en compte que les hommes sont plus nombreux que les
femmes en L1, en L2 et en L3 ce qui peut biaiser notre analyse.
Néanmoins, nous pouvons conclure que les femmes sont plus investies que
les hommes et que les dernières années sont plus investies que les
premières années. Nous pensons intéressant de regarder l'investissement
dans l'association A€T pour voir si les femmes sont plus investies que
les hommes dans cette dernière mais nous n'avons pas choisi cet axe
d'analyse.

\hypertarget{les-ruxe9sultats}{%
\section{Les résultats :}\label{les-ruxe9sultats}}

Nous allons analyser dans un premier temps la motivation selon le genre
et dans un second temps les moyennes selon le genre des étudiants de la
licence d'Économie à Tours. Les hypothèses que nous avons choisi sont :

\begin{itemize}
\tightlist
\item
  Les femmes sont plus motivées que les hommes,
\item
  Les femmes sont plus motivées que les hommes à augmenter leur moyenne
  entre le semestre 1 et le semestre 2,
\item
  Les femmes ont une meilleure moyenne que les hommes,
\item
  Les femmes ont de meilleures moyennes que les hommes dans les
  enseignements ``littéraires'' et à contrario les hommes ont de
  meilleurs moyennes que les femmes dans les enseignements
  ``scientifiques''.
\end{itemize}

Pour confirmer ou non nos hypothèses, nous allons analyser les variables
\texttt{Genre}, \texttt{Motivation}, \texttt{Evo\_note}, \texttt{Note},
\texttt{Macro}, \texttt{Micro}, \texttt{Maths}, \texttt{Stat} et
\texttt{Anglais} de la base de données finale \texttt{data}. Nous
interrogerons et testerons nos hypothèses en visualisant ces données
dans des tableaux, des graphiques et en réalisant des tests.

\hypertarget{la-motivation}{%
\subsection{\texorpdfstring{La \emph{motivation}
:}{La motivation :}}\label{la-motivation}}

\hypertarget{graphiques-et-tableaux}{%
\subsubsection{Graphiques et tableaux :}\label{graphiques-et-tableaux}}

\includegraphics{R3-L3S1--2-_files/figure-latex/figure5 and figure6-1.pdf}

\begin{table}[!h]

\caption{\label{tab:table5}La motivation selon le genre :}
\centering
\begin{tabular}[t]{rrr}
\toprule
Motivation & Femme & Homme\\
\midrule
\cellcolor{gray!6}{1} & \cellcolor{gray!6}{1} & \cellcolor{gray!6}{2}\\
2 & 1 & 2\\
\cellcolor{gray!6}{3} & \cellcolor{gray!6}{3} & \cellcolor{gray!6}{1}\\
4 & 0 & 3\\
\cellcolor{gray!6}{5} & \cellcolor{gray!6}{1} & \cellcolor{gray!6}{5}\\
\addlinespace
6 & 9 & 10\\
\cellcolor{gray!6}{7} & \cellcolor{gray!6}{12} & \cellcolor{gray!6}{22}\\
8 & 21 & 21\\
\cellcolor{gray!6}{9} & \cellcolor{gray!6}{9} & \cellcolor{gray!6}{6}\\
10 & 8 & 9\\
\bottomrule
\end{tabular}
\end{table}

La motivation moyenne chez les femmes est de 7.45 tandis que chez les
hommes elle est de 7.06. La motivation médiane chez les femmes est de 8
tandis que chez les hommes elle est de 7. Les femmes sont moins
nombreuses que les hommes à avoir une motivation en dessous de
\(\frac{5}{10}\). Nous en supposons donc que les femmes sont plus
motivées que les hommes.

\hypertarget{tests}{%
\subsubsection{Tests :}\label{tests}}

Nous allons tester si les femmes sont plus motivées que les hommes.

\begin{table}[!h]

\caption{\label{tab:tests1_non}Le test d'appartenance :}
\centering
\begin{tabular}[t]{lr}
\toprule
  & x\\
\midrule
\cellcolor{gray!6}{Motivation.p.value} & \cellcolor{gray!6}{0}\\
\bottomrule
\end{tabular}
\end{table}

\begin{table}[!h]

\caption{\label{tab:tests1_non}Le test de moyenne de la motivation selon le genre}
\centering
\begin{tabular}[t]{lllllll}
\toprule
  & Femme & Homme & var.equal & mean in group Femme & mean in group Homme & p.value\\
\midrule
\cellcolor{gray!6}{Motivation} & \cellcolor{gray!6}{65} & \cellcolor{gray!6}{81} & \cellcolor{gray!6}{Oui} & \cellcolor{gray!6}{7.45} & \cellcolor{gray!6}{7.06} & \cellcolor{gray!6}{0.24}\\
\bottomrule
\end{tabular}
\end{table}

\begin{table}[!h]

\caption{\label{tab:tests1_non}La p.value et l'effectif théorique mninimum du test d'indépendance :}
\centering
\begin{tabular}[t]{lr}
\toprule
  & Motivation\\
\midrule
\cellcolor{gray!6}{Eff\_théorique\_min} & \cellcolor{gray!6}{1.34}\\
p-value & 0.41\\
\bottomrule
\end{tabular}
\end{table}

Nous constatons grâce au test d'appartenance que la \texttt{p.value} est
inférieure au risque de premièe espèce : \(\alpha = 0.05\). Avec un
risque de 5 \%, la motivation ne suit pas une loi
normale.\textsuperscript{1}

Les conclusions suivantes sont à relativiser.\textsuperscript{1}
\textsuperscript{2} Nous pouvons constater grâce au test des variances
que ces dernières sont égales et nous constatons grâce au test bilatéral
de Student que la \texttt{p.value} est supérieure au risque de première
espèce : \(\alpha = 0.05\). Nous en arrivons à la conclusion, avec un
risque de 5\%, que la motivation entre les femmes et les hommes n'est
pas significativement différente.

\begin{center}\rule{0.5\linewidth}{0.5pt}\end{center}

\textsuperscript{1} Nous avons supposé que la motivation suit une loi
normale du fait que les effectifs soient petits.

\textsuperscript{2} Les effectifs théoriques minimums sont inférieurs à
5 donc les conditions de validité ne sont pas remplies.

\hypertarget{luxe9volution-de-la-moyenne-entre-le-semestre-1-et-le-semestre-2-en-pourcentage}{%
\subsection{\texorpdfstring{L'évolution de la moyenne entre le
\emph{semestre 1} et le \emph{semestre 2} (en pourcentage)
:}{L'évolution de la moyenne entre le semestre 1 et le semestre 2 (en pourcentage) :}}\label{luxe9volution-de-la-moyenne-entre-le-semestre-1-et-le-semestre-2-en-pourcentage}}

\hypertarget{graphiques-et-tableaux-1}{%
\subsubsection{Graphiques et tableaux
:}\label{graphiques-et-tableaux-1}}

\includegraphics{R3-L3S1--2-_files/figure-latex/figure7-1.pdf}

\begin{table}[!h]

\caption{\label{tab:table6}Le résumé de l'évolution de la moyenne entre le semestre 1 et le semestre 2 selon le genre (en pourcentage) :}
\centering
\begin{tabular}[t]{lrrrrrrr}
\toprule
  & Moyenne & Ecart-type & Minimum & Q1 & Q2 & Q3 & Maximum\\
\midrule
\cellcolor{gray!6}{Evo\_note\_Femme} & \cellcolor{gray!6}{29.88} & \cellcolor{gray!6}{22.34} & \cellcolor{gray!6}{-25.31} & \cellcolor{gray!6}{20.59} & \cellcolor{gray!6}{26.54} & \cellcolor{gray!6}{38.13} & \cellcolor{gray!6}{77.87}\\
Evo\_note\_Homme & 9.83 & 29.51 & -50.84 & -6.93 & 11.31 & 24.91 & 61.29\\
\bottomrule
\end{tabular}
\end{table}

Les femmes ont augmenté leur moyenne entre le semestre 1 et le semestre
2 de 29,88 \% en moyenne tandis que les hommes ont augmenté leur moyenne
entre le semestre 1 et le semestre 2 de 9,83 \% en moyenne. Nous pouvons
aussi constater une différence significative entre l'évolution minimum
qui est de 25,31 \% chez les femmes contre 50,84 \% chez les hommes.
Cela se voit également sur la boîte à moustahe, où l'on constate un
écart majeur entre les femmes et les hommes.Nous en supposons donc que
les femmes ``rebondissent'' ou tout du moins ``s'adaptent'' mieux que
les hommes en première année de la licence d'Économie à Tours.

\hypertarget{tests-1}{%
\subsubsection{Tests :}\label{tests-1}}

Nous allons tester si les femmes ``rebondissent'' mieux que les hommes
en première année de la licence d'Économie à Tours.

\begin{table}[!h]

\caption{\label{tab:tests2_non}La p.value du test d'appartenance :}
\centering
\begin{tabular}[t]{lr}
\toprule
  & x\\
\midrule
\cellcolor{gray!6}{Evo\_note.p.value} & \cellcolor{gray!6}{0.39}\\
\bottomrule
\end{tabular}
\end{table}

\begin{table}[!h]

\caption{\label{tab:tests2_non}Le test de moyenne de l'évolution de la moyenne entre le semestre 1 et le semestre 2 selon le genre}
\centering
\begin{tabular}[t]{lllllll}
\toprule
  & Femme & Homme & var.equal & mean in group Femme & mean in group Homme & p.value\\
\midrule
\cellcolor{gray!6}{Evo\_note} & \cellcolor{gray!6}{26} & \cellcolor{gray!6}{22} & \cellcolor{gray!6}{Non} & \cellcolor{gray!6}{29.88} & \cellcolor{gray!6}{9.83} & \cellcolor{gray!6}{0.01}\\
\bottomrule
\end{tabular}
\end{table}

\begin{table}[!h]

\caption{\label{tab:tests2_non}La p.value et l'effectif théorique mninimum du test d'indépendance :}
\centering
\begin{tabular}[t]{lr}
\toprule
  & Evo\_note\\
\midrule
\cellcolor{gray!6}{Eff\_théorique\_min} & \cellcolor{gray!6}{0.46}\\
p-value & 0.43\\
\bottomrule
\end{tabular}
\end{table}

Nous constatons grâce au test d'appartenance que la \texttt{p.value} est
supérieure au risque de première espèce : \(\alpha = 0.05\). Avec un
risque de 5 \%, l'évolution de la moyenne entre le semestre 1 et 2 suit
une loi normale.

Les conclusions suivantes sont à relativiser.\textsuperscript{1} Nous
pouvons constater grâce au test des variances que ces dernières ne sont
pas égales et nous constatons grâce au test bilatéral de Student que la
\texttt{p.value} est inférieure au risque de première espèce :
\(\alpha = 0.05\). Nous en arrivons à la conclusion, avec un risque de
5\%,que l'évolution de la moyenne entre le semestre 1 et 2 est
significativement différente entre les femmes et les hommes.

\begin{center}\rule{0.5\linewidth}{0.5pt}\end{center}

\textsuperscript{1} Les effectifs théoriques minimums sont inférieurs à
5 donc les conditions de validité ne sont pas remplies.

\textbf{Nous allons regrouper des données :}

Nous transformons les femmes dont l'évolution de la moyenne entre le
semestre 1 et le semestre 2 a diminué et nous leur donnons la valeur de
-25.31 (moyenne) ainsi que les femmes dont l'évolution de la moyenne
entre le semestre 1 et le semestre 2 a augmenté et nous leur donnons la
valeur de 32.09 (moyenne). Nous appliquons les mêmes transformations aux
hommes.

\begin{table}[!h]

\caption{\label{tab:tests2_oui1}La p.value et l'effectif théorique mninimum du test d'indépendance :}
\centering
\begin{tabular}[t]{lr}
\toprule
  & Evo\_note\\
\midrule
\cellcolor{gray!6}{Eff\_théorique\_min} & \cellcolor{gray!6}{0.46}\\
p-value & 0.00\\
\bottomrule
\end{tabular}
\end{table}

Les effectifs théoriques minimums sont inférieurs à 5 donc les
conditions de validité ne sont toujours pas remplies.

Nous retransformons les femmes et nous leur donnons la valeur de 29.88
(moyenne). Nous appliquons la même transformation aux hommes.

\begin{table}[!h]

\caption{\label{tab:tests2_oui2}La p.value et l'effectif théorique mninimum du test d'indépendance :}
\centering
\begin{tabular}[t]{lr}
\toprule
  & Evo\_note\\
\midrule
\cellcolor{gray!6}{Eff\_théorique\_min} & \cellcolor{gray!6}{28.94}\\
p-value & 0.00\\
\bottomrule
\end{tabular}
\end{table}

Les effectifs théoriques minimums sont supérieurs à 5 donc les
conditions de validité sont remplies. Nous constatons grâce au test
d'indépendance que la \texttt{p.value} est inférieure au risque de
première espèce : \(\alpha = 0.05\). Avec un risque de 5 \%, l'évolution
de la moyenne entre le semestre 1 et 2 est dépendante du genre.

En conclusion, nous pouvons affirmer que l'évolution de la moyenne entre
le semestre 1 et le semestre 2 d'un étudiant de la licence d'Économie à
Tours dépend de son genre. Les femmes augmentent plus leur moyenne entre
le semestre 1 et le semestre 2 que les hommes. Il peut être intéressant
de voir si les femmes ont une moyenne moins élevée que les hommes au
semestre 1 et plus élevée que les hommes au semestre 2 mais nous n'avons
pas choisi cet axe d'analyse.

\hypertarget{la-moyenne-guxe9nuxe9rale}{%
\subsection{\texorpdfstring{La \emph{moyenne générale}
:}{La moyenne générale :}}\label{la-moyenne-guxe9nuxe9rale}}

\hypertarget{graphiques-et-tableau}{%
\subsubsection{Graphiques et tableau :}\label{graphiques-et-tableau}}

\includegraphics{R3-L3S1--2-_files/figure-latex/figure8 and figure9-1.pdf}

\includegraphics{R3-L3S1--2-_files/figure-latex/figure10-1.pdf}

\begin{table}[!h]

\caption{\label{tab:table7}La moyenne générale selon le genre :}
\centering
\begin{tabular}[t]{lrrrr}
\toprule
  & 7.78 & 10.87 & 13.04 & 15.34\\
\midrule
\cellcolor{gray!6}{Femme} & \cellcolor{gray!6}{10} & \cellcolor{gray!6}{9} & \cellcolor{gray!6}{6} & \cellcolor{gray!6}{2}\\
Homme & 6 & 13 & 4 & 2\\
\bottomrule
\end{tabular}
\end{table}

Nous pouvons constater une différence non-significative de la moyenne
entre les femmes et les hommes. En effet, nous observons très légèrement
que les femmes ont une moyenne générale moins élevée que les hommes.
Nous pouvons supposer que les femmes réussissent moins que les hommes.

\hypertarget{tests-2}{%
\subsubsection{Tests :}\label{tests-2}}

Nous allons tester si les femmes réussissent moins que les hommes.

\begin{table}[!h]

\caption{\label{tab:tests3_non}La p.value du test d'appartenance :}
\centering
\begin{tabular}[t]{lr}
\toprule
  & x\\
\midrule
\cellcolor{gray!6}{Note.p.value} & \cellcolor{gray!6}{0.55}\\
\bottomrule
\end{tabular}
\end{table}

\begin{table}[!h]

\caption{\label{tab:tests3_non}Le test de moyenne de la moyenne générale selon le genre}
\centering
\begin{tabular}[t]{lllllll}
\toprule
  & Femme & Homme & var.equal & mean in group Femme & mean in group Homme & p.value\\
\midrule
\cellcolor{gray!6}{Note} & \cellcolor{gray!6}{27} & \cellcolor{gray!6}{25} & \cellcolor{gray!6}{Oui} & \cellcolor{gray!6}{10.62} & \cellcolor{gray!6}{10.74} & \cellcolor{gray!6}{0.87}\\
\bottomrule
\end{tabular}
\end{table}

\begin{table}[!h]

\caption{\label{tab:tests3_non}La p.value et l'effectif théorique mninimum du test d'indépendance :}
\centering
\begin{tabular}[t]{lr}
\toprule
  & Note\\
\midrule
\cellcolor{gray!6}{Eff\_théorique\_min} & \cellcolor{gray!6}{0.48}\\
p-value & 0.43\\
\bottomrule
\end{tabular}
\end{table}

Nous constatons grâce au test d'appartenance que la \texttt{p.value} est
supérieure au risque de première espèce : \(\alpha = 0.05\). Avec un
risque de 5 \%, la moyenne générale suit une loi normale.

Les conclusions suivantes sont à relativiser.\textsuperscript{1} Nous
pouvons constater grâce au test des variances que ces dernières sont
égales et nous constatons grâce au test bilatéral de Student que la
\texttt{p.value} est supérieure au risque de première espèce :
\(\alpha = 0.05\). Nous en arrivons à la conclusion, avec un risque de 5
\%, que la moyenne générale entre les femmes et les hommes n'est pas pas
significativement différente.

\begin{center}\rule{0.5\linewidth}{0.5pt}\end{center}

\textsuperscript{1} Les effectifs théoriques minimums sont inférieurs à
5 donc les conditions de validité ne sont pas remplies.

\hypertarget{les-moyennes-des-enseignements-de-macro-de-micro-de-maths-de-stat-et-danglais}{%
\subsection{\texorpdfstring{Les \emph{moyennes} des enseignements de
\emph{Macro}, de \emph{Micro}, de \emph{Maths}, de \emph{Stat} et
d'\emph{Anglais}
:}{Les moyennes des enseignements de Macro, de Micro, de Maths, de Stat et d'Anglais :}}\label{les-moyennes-des-enseignements-de-macro-de-micro-de-maths-de-stat-et-danglais}}

\hypertarget{graphiques-et-tableaux-2}{%
\subsubsection{Graphiques et tableaux
:}\label{graphiques-et-tableaux-2}}

\includegraphics{R3-L3S1--2-_files/figure-latex/figure11 to figure14-1.pdf}

\includegraphics{R3-L3S1--2-_files/figure-latex/figure15 to figure18-1.pdf}

\includegraphics{R3-L3S1--2-_files/figure-latex/figure19 and figure20-1.pdf}

\begin{table}[!h]

\caption{\label{tab:table8 and table9}Le résumé des moyennes des enseignements pour les femmes :}
\centering
\begin{tabular}[t]{lrrrrrrr}
\toprule
  & Moyenne & Ecart-type & Minimum & Q1 & Q2 & Q3 & Maximum\\
\midrule
\cellcolor{gray!6}{Macro} & \cellcolor{gray!6}{11.57} & \cellcolor{gray!6}{3.38} & \cellcolor{gray!6}{4.25} & \cellcolor{gray!6}{9.75} & \cellcolor{gray!6}{12.10} & \cellcolor{gray!6}{13.57} & \cellcolor{gray!6}{17.13}\\
Micro & 10.68 & 2.88 & 4.25 & 9.05 & 11.05 & 12.25 & 15.70\\
\cellcolor{gray!6}{Maths} & \cellcolor{gray!6}{5.93} & \cellcolor{gray!6}{4.28} & \cellcolor{gray!6}{0.00} & \cellcolor{gray!6}{2.15} & \cellcolor{gray!6}{5.40} & \cellcolor{gray!6}{8.88} & \cellcolor{gray!6}{14.40}\\
Stat & 10.34 & 4.48 & 2.44 & 7.38 & 10.45 & 14.32 & 17.75\\
\cellcolor{gray!6}{Anglais} & \cellcolor{gray!6}{10.70} & \cellcolor{gray!6}{2.41} & \cellcolor{gray!6}{5.88} & \cellcolor{gray!6}{9.25} & \cellcolor{gray!6}{10.25} & \cellcolor{gray!6}{12.75} & \cellcolor{gray!6}{15.25}\\
\bottomrule
\end{tabular}
\end{table}

\begin{table}[!h]

\caption{\label{tab:table8 and table9}Le résumé des moyennes des enseignements pour les hommes :}
\centering
\begin{tabular}[t]{lrrrrrrr}
\toprule
  & Moyenne & Ecart-type & Minimum & Q1 & Q2 & Q3 & Maximum\\
\midrule
\cellcolor{gray!6}{Macro} & \cellcolor{gray!6}{12.05} & \cellcolor{gray!6}{3.24} & \cellcolor{gray!6}{5.80} & \cellcolor{gray!6}{9.84} & \cellcolor{gray!6}{12.09} & \cellcolor{gray!6}{14.68} & \cellcolor{gray!6}{18.60}\\
Micro & 11.56 & 3.05 & 3.75 & 9.51 & 11.82 & 13.57 & 16.15\\
\cellcolor{gray!6}{Maths} & \cellcolor{gray!6}{6.40} & \cellcolor{gray!6}{4.20} & \cellcolor{gray!6}{0.00} & \cellcolor{gray!6}{3.30} & \cellcolor{gray!6}{5.66} & \cellcolor{gray!6}{9.59} & \cellcolor{gray!6}{17.40}\\
Stat & 11.19 & 4.20 & 3.40 & 8.24 & 11.77 & 14.39 & 17.50\\
\cellcolor{gray!6}{Anglais} & \cellcolor{gray!6}{10.26} & \cellcolor{gray!6}{2.97} & \cellcolor{gray!6}{3.38} & \cellcolor{gray!6}{9.25} & \cellcolor{gray!6}{10.43} & \cellcolor{gray!6}{12.00} & \cellcolor{gray!6}{15.95}\\
\bottomrule
\end{tabular}
\end{table}

Nous pouvons constater qu'il n'y a pas de différence significative des
moyennes entre les femmes et les hommes. En revanche, en Micro et en
Stats, nous pouvons constater une différence non-significative entre les
femmes et les hommes. La moyenne de Micro pour les femmes est de 10.68
tandis que pour les hommes elle est de 11.56, de même pour les Stats,
les femmes ont une moyenne de 10.34 tandis que les hommes ont une
moyenne de 11.19. La moyenne d'Anglais pour les femmes est de 10.7
(données eronnées enlevées) tandis que pour les hommes elle est de
10.26. Nous pouvons également confirmer cela par la densité. En effet,
en Micro Et stat la densité se situe plus sur la droite pour les hommes
que pour les femmes. Nous pouvons supposer que les femmes réussissent
légèrement moins que les hommes dans les enseignements scientifiques
mais légèrement mieux dans les enseignements littéraires.

\hypertarget{tests-3}{%
\subsubsection{Tests :}\label{tests-3}}

Nous allons tester si femmes réussissent légèrement moins que les hommes
dans les enseignements scientifiques et légèrement mieux dans les
enseignements littéraires.

\begin{table}[!h]

\caption{\label{tab:tests4_non}Le test d'appartenance :}
\centering
\begin{tabular}[t]{lr}
\toprule
  & x\\
\midrule
\cellcolor{gray!6}{Macro.p.value} & \cellcolor{gray!6}{0.74}\\
Micro.p.value & 0.37\\
\cellcolor{gray!6}{Maths.p.value} & \cellcolor{gray!6}{0.02}\\
Stat.p.value & 0.08\\
\cellcolor{gray!6}{Anglais.p.value} & \cellcolor{gray!6}{0.00}\\
\bottomrule
\end{tabular}
\end{table}

\begin{table}[!h]

\caption{\label{tab:tests4_non}Le test de moyenne des moyennes des enseignements selon le genre}
\centering
\begin{tabular}[t]{lllllll}
\toprule
  & Femme & Homme & var.equal & mean in group Femme & mean in group Homme & p.value\\
\midrule
\cellcolor{gray!6}{Macro} & \cellcolor{gray!6}{26} & \cellcolor{gray!6}{21} & \cellcolor{gray!6}{Oui} & \cellcolor{gray!6}{11.57} & \cellcolor{gray!6}{12.05} & \cellcolor{gray!6}{0.63}\\
Micro & 25 & 22 & Oui & 10.68 & 11.56 & 0.32\\
\cellcolor{gray!6}{Maths} & \cellcolor{gray!6}{29} & \cellcolor{gray!6}{26} & \cellcolor{gray!6}{Oui} & \cellcolor{gray!6}{5.93} & \cellcolor{gray!6}{6.4} & \cellcolor{gray!6}{0.68}\\
Stat & 27 & 23 & Oui & 10.34 & 11.19 & 0.49\\
\cellcolor{gray!6}{Anglais} & \cellcolor{gray!6}{29} & \cellcolor{gray!6}{29} & \cellcolor{gray!6}{Oui} & \cellcolor{gray!6}{11.8} & \cellcolor{gray!6}{10.26} & \cellcolor{gray!6}{0.25}\\
\bottomrule
\end{tabular}
\end{table}

\begin{table}[!h]

\caption{\label{tab:tests4_non}La p.value et l'effectif théorique mninimum du test d'indépendance :}
\centering
\begin{tabular}[t]{lrrrrr}
\toprule
  & Macro & Micro & Maths & Stat & Anglais\\
\midrule
\cellcolor{gray!6}{Eff\_théorique\_min} & \cellcolor{gray!6}{0.45} & \cellcolor{gray!6}{0.47} & \cellcolor{gray!6}{0.47} & \cellcolor{gray!6}{0.46} & \cellcolor{gray!6}{0.50}\\
p-value & 0.39 & 0.24 & 0.56 & 0.43 & 0.42\\
\bottomrule
\end{tabular}
\end{table}

Nous constatons grâce au test d'appartenance que les \texttt{p.values}
sont supérieures pour la Macro, la Micro et les Stats et sont
inférieures pour les Maths et l'Anglais au risque de première espèce :
\(\alpha = 0.05\). Avec un risque de 5 \%, les moyennes de Macro, de
Micro et de Stat suivent des lois normales et celles de Maths et
d'Anglais ne suivent pas des lois normales.\textsuperscript{1}

Les conclusions suivants sont à relativiser.\textsuperscript{1}
\textsuperscript{2} Nous constatons grâce au test des variances que ces
dernières sont égales et nous constatons grâce au test bilatéral de
Student que les \texttt{p.values} sont supérieures au risque de première
espèce : \(\alpha = 0.05\). Nous concluons, avec un risque de 5 \%, que
les moyennes de Macro, de Micro, de Maths, de Stat et d'Anglais entre
les femmes et les hommes ne sont pas sigificativement différentes.

\begin{center}\rule{0.5\linewidth}{0.5pt}\end{center}

\textsuperscript{1} Nous avons supposé que les moyennes de Maths et
d'Anglais suivents des lois normalesdu faitque les effectifs soient
petits.

\textsuperscript{2} Les effectifs théoriques minimums sont inférieurs à
5 donc les conditions de validité ne sont pas remplies.

\hypertarget{les-choix}{%
\section{Les choix :}\label{les-choix}}

Nous allons analyser le choix du niveau d'éducation visé selon le genre
pour les étudiants de la licence d'Économie à Tours. Les hypothèses que
nous avons choisi sont :

\begin{itemize}
\tightlist
\item
  Les femmes veulent faire des études plus longues que les hommes.
\item
  Les femmes et les hommes motivés veulent faire des études plus
  longues.
\end{itemize}

Pour confirmer ou non nos hypothèse, nous allons analyser les variables
\texttt{Genre}, \texttt{Motivation} et \texttt{Niveau\_edu\_vise} de la
base de données finale \texttt{data}. Nous interrogerons nos hypothèses
en visualisant ces données dans des tableaux et des graphiques.

\hypertarget{le-niveau-duxe9ducation-visuxe9}{%
\subsection{\texorpdfstring{\emph{Le niveau d'éducation visé}
:}{Le niveau d'éducation visé :}}\label{le-niveau-duxe9ducation-visuxe9}}

\hypertarget{graphique-et-tableau}{%
\subsubsection{Graphique et tableau :}\label{graphique-et-tableau}}

\begin{quote}
\textbf{Cette analyse est descriptive.}
\end{quote}

\includegraphics{R3-L3S1--2-_files/figure-latex/figure21-1.pdf}

\begin{table}[!h]

\caption{\label{tab:table10}Le niveau d'éducation visé selon le genre :}
\centering
\begin{tabular}[t]{lrr}
\toprule
  & Femme & Homme\\
\midrule
\cellcolor{gray!6}{BAC+2} & \cellcolor{gray!6}{1} & \cellcolor{gray!6}{1}\\
BAC+3 & 7 & 8\\
\cellcolor{gray!6}{BAC+4} & \cellcolor{gray!6}{0} & \cellcolor{gray!6}{1}\\
BAC+5 & 47 & 52\\
\cellcolor{gray!6}{BAC+5 ET +} & \cellcolor{gray!6}{10} & \cellcolor{gray!6}{19}\\
\bottomrule
\end{tabular}
\end{table}

Nous pouvons constater une différence non-significative du niveau
d'éducation visé entre les femmes et les hommes et une différence
significative du niveau d'éducation visé selon la motivation. En effet,
nous observons très légèrement que les femmes veulent faire des études
moins longues que les hommes. En revanche, nous observons que plus les
femmes et les hommes sont motivés et plus le niveau d'éducation visé est
élevé.

En conclusion, nous pouvons supposer que le niveau d'éducation visé d'un
étudiant de la licence d'Économie à Tours dépend d'avantage de sa
motivation que de son genre.

\hypertarget{conclusion}{%
\section{Conclusion :}\label{conclusion}}

Pour conclure, nous devons tout de même nuancer nos résultats puisque
nous avons supposé que certaines variables suivaient des lois normales à
contrario des résultats des tests d'appartenances et certains tests ne
respectaient pas les conditions de validité.

Nous pouvons rejetter l'hypothèse que le genre va influencer les
résultats. En effet, nous constatons que les femmes ont des résultats
presque similaires aux hommes. En revanche, l'hypothèse qui est basée
sur l'évolution entre la moyenne du semestre 1 et du semestre 2, est
conservée. Les femmes vont mieux réussir leur semestre 2 par rapport au
semestre 1 que les hommes.

Enfin, notre analyse descriptive sur le niveau d'éducation visé, nous
``indique'', en l'abscence de tests, que le genre ne va pas avoir
d'influence sur ce dernier mais qu'ici, cela dépend d'avantage de la
motivation des étudiants.

\end{document}
