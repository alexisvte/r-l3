% Options for packages loaded elsewhere
\PassOptionsToPackage{unicode}{hyperref}
\PassOptionsToPackage{hyphens}{url}
%
\documentclass[
  11pt,
  french,
]{article}
\usepackage{lmodern}
\usepackage{amssymb,amsmath}
\usepackage{ifxetex,ifluatex}
\ifnum 0\ifxetex 1\fi\ifluatex 1\fi=0 % if pdftex
  \usepackage[T1]{fontenc}
  \usepackage[utf8]{inputenc}
  \usepackage{textcomp} % provide euro and other symbols
\else % if luatex or xetex
  \usepackage{unicode-math}
  \defaultfontfeatures{Scale=MatchLowercase}
  \defaultfontfeatures[\rmfamily]{Ligatures=TeX,Scale=1}
\fi
% Use upquote if available, for straight quotes in verbatim environments
\IfFileExists{upquote.sty}{\usepackage{upquote}}{}
\IfFileExists{microtype.sty}{% use microtype if available
  \usepackage[]{microtype}
  \UseMicrotypeSet[protrusion]{basicmath} % disable protrusion for tt fonts
}{}
\makeatletter
\@ifundefined{KOMAClassName}{% if non-KOMA class
  \IfFileExists{parskip.sty}{%
    \usepackage{parskip}
  }{% else
    \setlength{\parindent}{0pt}
    \setlength{\parskip}{6pt plus 2pt minus 1pt}}
}{% if KOMA class
  \KOMAoptions{parskip=half}}
\makeatother
\usepackage{xcolor}
\IfFileExists{xurl.sty}{\usepackage{xurl}}{} % add URL line breaks if available
\IfFileExists{bookmark.sty}{\usepackage{bookmark}}{\usepackage{hyperref}}
\hypersetup{
  pdftitle={E2-L3S2 Le prix des logements dans l'État de Washington aux États-Unis},
  pdfauthor={Lola LUBIN et Alexis VIALATTE},
  pdflang={fr},
  hidelinks,
  pdfcreator={LaTeX via pandoc}}
\urlstyle{same} % disable monospaced font for URLs
\usepackage[a4paper, top = 2cm, bottom = 2cm, left = 1.5cm, right = 1.5cm]{geometry}
\usepackage{color}
\usepackage{fancyvrb}
\newcommand{\VerbBar}{|}
\newcommand{\VERB}{\Verb[commandchars=\\\{\}]}
\DefineVerbatimEnvironment{Highlighting}{Verbatim}{commandchars=\\\{\}}
% Add ',fontsize=\small' for more characters per line
\usepackage{framed}
\definecolor{shadecolor}{RGB}{248,248,248}
\newenvironment{Shaded}{\begin{snugshade}}{\end{snugshade}}
\newcommand{\AlertTok}[1]{\textcolor[rgb]{0.94,0.16,0.16}{#1}}
\newcommand{\AnnotationTok}[1]{\textcolor[rgb]{0.56,0.35,0.01}{\textbf{\textit{#1}}}}
\newcommand{\AttributeTok}[1]{\textcolor[rgb]{0.77,0.63,0.00}{#1}}
\newcommand{\BaseNTok}[1]{\textcolor[rgb]{0.00,0.00,0.81}{#1}}
\newcommand{\BuiltInTok}[1]{#1}
\newcommand{\CharTok}[1]{\textcolor[rgb]{0.31,0.60,0.02}{#1}}
\newcommand{\CommentTok}[1]{\textcolor[rgb]{0.56,0.35,0.01}{\textit{#1}}}
\newcommand{\CommentVarTok}[1]{\textcolor[rgb]{0.56,0.35,0.01}{\textbf{\textit{#1}}}}
\newcommand{\ConstantTok}[1]{\textcolor[rgb]{0.00,0.00,0.00}{#1}}
\newcommand{\ControlFlowTok}[1]{\textcolor[rgb]{0.13,0.29,0.53}{\textbf{#1}}}
\newcommand{\DataTypeTok}[1]{\textcolor[rgb]{0.13,0.29,0.53}{#1}}
\newcommand{\DecValTok}[1]{\textcolor[rgb]{0.00,0.00,0.81}{#1}}
\newcommand{\DocumentationTok}[1]{\textcolor[rgb]{0.56,0.35,0.01}{\textbf{\textit{#1}}}}
\newcommand{\ErrorTok}[1]{\textcolor[rgb]{0.64,0.00,0.00}{\textbf{#1}}}
\newcommand{\ExtensionTok}[1]{#1}
\newcommand{\FloatTok}[1]{\textcolor[rgb]{0.00,0.00,0.81}{#1}}
\newcommand{\FunctionTok}[1]{\textcolor[rgb]{0.00,0.00,0.00}{#1}}
\newcommand{\ImportTok}[1]{#1}
\newcommand{\InformationTok}[1]{\textcolor[rgb]{0.56,0.35,0.01}{\textbf{\textit{#1}}}}
\newcommand{\KeywordTok}[1]{\textcolor[rgb]{0.13,0.29,0.53}{\textbf{#1}}}
\newcommand{\NormalTok}[1]{#1}
\newcommand{\OperatorTok}[1]{\textcolor[rgb]{0.81,0.36,0.00}{\textbf{#1}}}
\newcommand{\OtherTok}[1]{\textcolor[rgb]{0.56,0.35,0.01}{#1}}
\newcommand{\PreprocessorTok}[1]{\textcolor[rgb]{0.56,0.35,0.01}{\textit{#1}}}
\newcommand{\RegionMarkerTok}[1]{#1}
\newcommand{\SpecialCharTok}[1]{\textcolor[rgb]{0.00,0.00,0.00}{#1}}
\newcommand{\SpecialStringTok}[1]{\textcolor[rgb]{0.31,0.60,0.02}{#1}}
\newcommand{\StringTok}[1]{\textcolor[rgb]{0.31,0.60,0.02}{#1}}
\newcommand{\VariableTok}[1]{\textcolor[rgb]{0.00,0.00,0.00}{#1}}
\newcommand{\VerbatimStringTok}[1]{\textcolor[rgb]{0.31,0.60,0.02}{#1}}
\newcommand{\WarningTok}[1]{\textcolor[rgb]{0.56,0.35,0.01}{\textbf{\textit{#1}}}}
\usepackage{graphicx,grffile}
\makeatletter
\def\maxwidth{\ifdim\Gin@nat@width>\linewidth\linewidth\else\Gin@nat@width\fi}
\def\maxheight{\ifdim\Gin@nat@height>\textheight\textheight\else\Gin@nat@height\fi}
\makeatother
% Scale images if necessary, so that they will not overflow the page
% margins by default, and it is still possible to overwrite the defaults
% using explicit options in \includegraphics[width, height, ...]{}
\setkeys{Gin}{width=\maxwidth,height=\maxheight,keepaspectratio}
% Set default figure placement to htbp
\makeatletter
\def\fps@figure{htbp}
\makeatother
\setlength{\emergencystretch}{3em} % prevent overfull lines
\providecommand{\tightlist}{%
  \setlength{\itemsep}{0pt}\setlength{\parskip}{0pt}}
\setcounter{secnumdepth}{5}
\usepackage{tikz}
\usepackage{dcolumn}
\usepackage{booktabs}
\usepackage{longtable}
\usepackage{array}
\usepackage{multirow}
\usepackage{wrapfig}
\usepackage{float}
\usepackage{colortbl}
\usepackage{pdflscape}
\usepackage{tabu}
\usepackage{threeparttable}
\usepackage{threeparttablex}
\usepackage[normalem]{ulem}
\usepackage{makecell}
\ifxetex
  % Load polyglossia as late as possible: uses bidi with RTL langages (e.g. Hebrew, Arabic)
  \usepackage{polyglossia}
  \setmainlanguage[]{french}
\else
  \usepackage[shorthands=off,main=french]{babel}
\fi

\title{\texttt{E2-L3S2} Le prix des logements dans l'État de Washington aux
États-Unis}
\author{Lola LUBIN et Alexis VIALATTE}
\date{2021}

\begin{document}
\maketitle

{
\setcounter{tocdepth}{3}
\tableofcontents
}
\newpage

\hypertarget{formalisation}{%
\section{Formalisation}\label{formalisation}}

\begin{center}\rule{0.5\linewidth}{0.5pt}\end{center}

Nous posons \(x_{i,k}\) le \(i\text{-ième}\) logement et la
\(k\text{-ième}\) variable avec \(i \in [1,\ldots,I]\) et
\(k \in [1,\ldots,K]\) ainsi que \(\mathcal{M}_{j}\) le
\(j\text{-ième}\) modèle avec \(j \in [1,\ldots,J]\) et
\(\mathcal{A}_{n}\) la \(n\text{-ième}\) analyse avec
\(n \in [1,\ldots,N]\). Nous aurons \(\forall j \in [1,\ldots,J]\) et
\(\forall n \in [1,\ldots,N]\) :
\[\mathcal{A}_{n}=\{\mathcal{M}_{1},\ldots,\mathcal{M}_{J}\}\]

Nous poserons la variable expliquée \(y_{i}=x_{i,\tilde{k}}\) où
\(\tilde{k} \in [1,\ldots,K]\), les variables explicatives \(x_{i,k}\)
où \(k \in [1,\ldots,K]\ \setminus\ \tilde{k}\) et la variable
résiduelle \(z_{i}\). Nous aurons : \begin{equation}
  \mathcal{M}_{j}:y_{i}=\alpha+\underset{k=1}{\overset{K}{\sum}}(\beta_{k}x_{i,k})+z_{i}
\end{equation}

Nous minimiserons la somme des résidus au carré. Nous aurons :
\begin{equation}
  \mathcal{M}_{j}:\text{min}\Bigg{(}\underset{i=1}{\overset{I}{\sum}}z_{i}^{2}\Bigg{)}=\text{min}\Bigg\{\underset{i=1}{\overset{I}{\sum}}\Big{[}y_{i}-\alpha-\underset{k=1}{\overset{K}{\sum}}(\beta_{k}x_{i,k})\Big{]}^{2}\Bigg\}
\end{equation}

Nous chercherons à estimer \(\hat{\alpha}\), la constante du
\(j\text{-ième}\) modèle et \(\hat{\beta_{k}}\), le coefficient de la
\(k\text{-ième}\) variable du \(j\text{-ième}\) modèle.

Nous poserons \(\phi\) le risque de première espèce, \(t\) la
statistique du test de Students, \(F\) la statistique du test de Fisher,
\(DW\) la statistique du test de Durbin-Watson, \(W\) la statistique du
test de Shapiro-Wilk, \(BP\) la statistique du test de Breusch-Pagan et
\(HMC\) la statistique du test d'Harrison-McCabe. Nous aurons :

\[\begin{array}{c}
\mathcal{H}_{0}:\alpha=0 \iff p\text{-}value \geqslant \phi\\
\mathcal{H}_{1}:\alpha\not=0 \iff p\text{-}value < \phi
\end{array}\]

\[\begin{array}{c}
\mathcal{H}_{0}:\beta_{k}=0 \iff p\text{-}value \geqslant \phi\\
\mathcal{H}_{1}:\beta_{k}\not=0 \iff p\text{-}value < \phi
\end{array}\]

\[\begin{array}{c}
\mathcal{H}_{0}:\text{absence de significativité} \iff p\text{-}value \geqslant \phi\\
\mathcal{H}_{1}:\text{significativité} \iff p\text{-}value < \phi
\end{array}\]

\[\begin{array}{c}
\mathcal{H}_{0}:\text{absence d'autocorrélation} \iff p\text{-}value \geqslant \phi\\
\mathcal{H}_{1}:\text{autocorrélation} \iff p\text{-}value < \phi
\end{array}\]

\[\begin{array}{c}
\mathcal{H}_{0}:\text{distribution normale} \iff p\text{-}value \geqslant \phi\\
\mathcal{H}_{1}:\text{absence de distribution normale} \iff p\text{-}value < \phi
\end{array}\]

\[\begin{array}{c}
\mathcal{H}_{0}:\text{homoscédasticité} \iff p\text{-}value \geqslant \phi\\
\mathcal{H}_{1}:\text{hétéroscédasticité} \iff p\text{-}value < \phi
\end{array}\]

\begin{center}\rule{0.5\linewidth}{0.5pt}\end{center}

\newpage

\hypertarget{la-base-de-donnuxe9es-bdd_data.csv}{%
\section{\texorpdfstring{La base de données
\texttt{BDD\_data.csv}}{La base de données BDD\_data.csv}}\label{la-base-de-donnuxe9es-bdd_data.csv}}

\[\forall n \in [1,\ldots,N], \forall j \in [1,\ldots,J] \iff \tilde{k}=3\]

Nous allons commencer par nous poser une question : quelles sont les
variables qui expliquent le prix des logements dans l'État de Washington
aux États-Unis ?

\begin{table}[!htbp] \centering 
  \caption{Résumé de la variable price :} 
  \label{} 
\begin{tabular}{@{\extracolsep{5pt}} ccccccc} 
\\[-1.8ex]\hline 
\hline \\[-1.8ex] 
Moyenne & Ecart-type & Minimum & Q1 & Q2 & Q3 & Maximum \\ 
\hline \\[-1.8ex] 
$540,088.100$ & $367,127.200$ & $75,000$ & $321,950$ & $450,000$ & $645,000$ & $7,700,000$ \\ 
\hline \\[-1.8ex] 
\end{tabular} 
\end{table}

Pour répondre à cette question, nous disposons d'une base de données
\texttt{BDD\_data.csv} contenant 21613 logements et 21 variables. Il y a
0 donnée manquante concernant 0 logement. Les variables sont :

\begin{enumerate}
\def\labelenumi{\arabic{enumi}.}
\tightlist
\item
  id,
\item
  date,
\item
  price,
\item
  bedrooms,
\item
  bathrooms,
\item
  sqft\_living,
\item
  sqft\_lot,
\item
  floors,
\item
  waterfront,
\item
  view,
\item
  condition,
\item
  grade,
\item
  sqft\_above,
\item
  sqft\_basement,
\item
  yr\_built,
\item
  yr\_renovated,
\item
  zipcode,
\item
  lat,
\item
  long,
\item
  sqft\_living15,
\item
  sqft\_lot15.
\end{enumerate}

Pour chaque logement, la variable \texttt{ìd} correspond à son index, la
variable \texttt{date} correspond à son année de vente, la variable
\texttt{price} correspond à son prix de vente, la variable
\texttt{bedrooms} correspond à son nombre de chambres, la variable
\texttt{bathrooms} correspond à son nombre de salles de bain, la
variable \texttt{sqft\_living} correspond à sa surface intérieure en
pieds carrés, la variable \texttt{sqft\_lot} correspond à sa surface
totale en pieds carrés, la variable \texttt{floors} correspond à son
nombre d'étages, la variable \texttt{waterfront} répond à s'il a une vue
sur la mer ou non, la variable \texttt{view} correspond à son point de
vue, la variable \texttt{condition} correspond à son état, la variable
\texttt{grade} correspond à son \textit{design}, la variable
\texttt{sqft\_above} correspond à sa surface intérieure en pieds carrés
moins son sous-sol, la variable \texttt{sqft\_basement} correspond à sa
surface intérieure en pieds carrés dans son sous-sol, la variable
\texttt{yr\_built} correspond à son année de construction, la variable
\texttt{yr\_renovated} correspond à son année de rénovation s'il a été
rénové, la variable \texttt{zipcode} correspond à son code postal, les
variables \texttt{lat} et \texttt{long} correspondent à ses coordonnées
géographiques et les variables \texttt{sqft\_living15} et
\texttt{sqft\_lot15} correspondent à sa surface intérieure et à sa
surface totale en 2015.

\newpage

\hypertarget{les-manipulations-de-la-base-de-donnuxe9es}{%
\subsection{Les manipulations de la base de
données}\label{les-manipulations-de-la-base-de-donnuxe9es}}

Nous avons discuté de la base de données et nous ne pensons pas
conserver toutes les variables principalement pour des raisons
techniques. Nous allons regarder chaque variable de la base de données
et nous allons choisir de la conserver, de la transformer ou de la
supprimer.

\hypertarget{les-variables-id-et-date}{%
\subsubsection{\texorpdfstring{Les variables \texttt{id} et
\texttt{date}}{Les variables id et date}}\label{les-variables-id-et-date}}

\begin{Shaded}
\begin{Highlighting}[]
\NormalTok{data}\OperatorTok{$}\NormalTok{date <-}\StringTok{ }\KeywordTok{as.character}\NormalTok{( data}\OperatorTok{$}\NormalTok{date )}

\NormalTok{data}\OperatorTok{$}\NormalTok{date <-}\StringTok{ }\KeywordTok{str_sub}\NormalTok{( data}\OperatorTok{$}\NormalTok{date,}
                      \DecValTok{1}\NormalTok{,}
                      \DecValTok{4}\NormalTok{ )}

\NormalTok{data}\OperatorTok{$}\NormalTok{date <-}\StringTok{ }\KeywordTok{as.numeric}\NormalTok{( data}\OperatorTok{$}\NormalTok{date )}
\end{Highlighting}
\end{Shaded}

La variable \texttt{id} n'est pas intéressante à analyser. Nous nous
sommes demandés si, pour chaque logement, la date de la vente avait une
influence sur le prix de ces derniers. \textit{A priori}, nous supposons
que non. Nous ne nous servirons donc pas des variables \texttt{id} et
\texttt{date} car le graphique ci-dessous ne va pas à l'encontre de
notre avis.

\begin{center}\includegraphics{E2-L3S2--1-_files/figure-latex/data.figure1-1} \end{center}

\newpage

\hypertarget{les-variables-sqft_living-et-sqft_lot}{%
\subsubsection{\texorpdfstring{Les variables \texttt{sqft\_living} et
\texttt{sqft\_lot}}{Les variables sqft\_living et sqft\_lot}}\label{les-variables-sqft_living-et-sqft_lot}}

\begin{table}[!h]

\caption{\label{tab:unnamed-chunk-2}Résumé des variables price, sqft-living et sqft-garden :}
\centering
\begin{tabular}[t]{lrrr}
\toprule
  & price & sqft\_living & sqft\_lot\\
\midrule
\cellcolor{gray!6}{Moyenne} & \cellcolor{gray!6}{540088.1} & \cellcolor{gray!6}{2079.900} & \cellcolor{gray!6}{15106.97}\\
Ecart-type & 367127.2 & 918.441 & 41420.51\\
\cellcolor{gray!6}{Minimum} & \cellcolor{gray!6}{75000.0} & \cellcolor{gray!6}{290.000} & \cellcolor{gray!6}{520.00}\\
Q1 & 321950.0 & 1427.000 & 5040.00\\
\cellcolor{gray!6}{Q2} & \cellcolor{gray!6}{450000.0} & \cellcolor{gray!6}{1910.000} & \cellcolor{gray!6}{7618.00}\\
\addlinespace
Q3 & 645000.0 & 2550.000 & 10688.00\\
\cellcolor{gray!6}{Maximum} & \cellcolor{gray!6}{7700000.0} & \cellcolor{gray!6}{13540.000} & \cellcolor{gray!6}{1651359.00}\\
\bottomrule
\end{tabular}
\end{table}

Nous nous sommes demandés si, pour chaque logement, la surface
intérieure en pieds carrés avait une influence plus ou moins importante
que la surface extérieure en pieds carrés sur le prix de ces derniers.
\textit{A priori}, nous supposons que oui car un pied carré intérieur
est plus important qu'un pied carré extérieur. Nous nous servirons donc
des variables \texttt{sqft\_living} et \texttt{sqft\_lot} car les
graphiques ci-dessous ne vont pas à l'encontre de notre avis.

\begin{center}\includegraphics{E2-L3S2--1-_files/figure-latex/data.figure2-1} \end{center}

\newpage

\hypertarget{les-variables-bedrooms-bathrooms-et-floors}{%
\subsubsection{\texorpdfstring{Les variables \texttt{bedrooms},
\texttt{bathrooms} et
\texttt{floors}}{Les variables bedrooms, bathrooms et floors}}\label{les-variables-bedrooms-bathrooms-et-floors}}

\begin{table}[!h]

\caption{\label{tab:unnamed-chunk-3}Résumé des variables bedrooms, bathrooms et floors :}
\centering
\begin{tabular}[t]{lrrr}
\toprule
  & bedrooms & bathrooms & floors\\
\midrule
\cellcolor{gray!6}{Moyenne} & \cellcolor{gray!6}{3.371} & \cellcolor{gray!6}{2.115} & \cellcolor{gray!6}{1.494}\\
Ecart-type & 0.930 & 0.770 & 0.540\\
\cellcolor{gray!6}{Minimum} & \cellcolor{gray!6}{0.000} & \cellcolor{gray!6}{0.000} & \cellcolor{gray!6}{1.000}\\
Q1 & 3.000 & 1.750 & 1.000\\
\cellcolor{gray!6}{Q2} & \cellcolor{gray!6}{3.000} & \cellcolor{gray!6}{2.250} & \cellcolor{gray!6}{1.500}\\
\addlinespace
Q3 & 4.000 & 2.500 & 2.000\\
\cellcolor{gray!6}{Maximum} & \cellcolor{gray!6}{33.000} & \cellcolor{gray!6}{8.000} & \cellcolor{gray!6}{3.500}\\
\bottomrule
\end{tabular}
\end{table}

Nous nous sommes demandés si, pour chaque logement, le nombre de
chambres, le nombre de salles de bain et le nombre d'étages avaient une
influence sur le prix de ces derniers. \textit{A priori}, nous supposons
que oui car plus le nombre de chambres, le nombre de salles de bain et
le nombre d'étages sont élevés et plus la surface intérieure en pieds
carrés est élévée. Nous nous servirons donc des variables
\texttt{bedrooms}, \texttt{bathrooms} et \texttt{floors} car les
graphiques ci-dessous ne vont pas à l'encontre de notre avis.

\begin{center}\includegraphics{E2-L3S2--1-_files/figure-latex/data.figure3-1} \end{center}

\newpage

\hypertarget{les-variables-sqft_above-et-sqft_basement}{%
\subsubsection{\texorpdfstring{Les variables \texttt{sqft\_above} et
\texttt{sqft\_basement}}{Les variables sqft\_above et sqft\_basement}}\label{les-variables-sqft_above-et-sqft_basement}}

Les variables \texttt{sqft\_built}, \texttt{sqft\_renovated},
\texttt{sqft\_living15} et \texttt{sqft\_lot15} ne sont pas
intéressantes à analyser car elles sont équivalentes des variables
\texttt{yr\_built} et \texttt{yr\_renovated}.

\begin{table}[!h]

\caption{\label{tab:unnamed-chunk-4}Résumé des variables sqft-above et sqft-basement :}
\centering
\begin{tabular}[t]{lrr}
\toprule
  & sqft\_above & sqft\_basement\\
\midrule
\cellcolor{gray!6}{Moyenne} & \cellcolor{gray!6}{1788.391} & \cellcolor{gray!6}{291.509}\\
Ecart-type & 828.091 & 442.575\\
\cellcolor{gray!6}{Minimum} & \cellcolor{gray!6}{290.000} & \cellcolor{gray!6}{0.000}\\
Q1 & 1190.000 & 0.000\\
\cellcolor{gray!6}{Q2} & \cellcolor{gray!6}{1560.000} & \cellcolor{gray!6}{0.000}\\
\addlinespace
Q3 & 2210.000 & 560.000\\
\cellcolor{gray!6}{Maximum} & \cellcolor{gray!6}{9410.000} & \cellcolor{gray!6}{4820.000}\\
\bottomrule
\end{tabular}
\end{table}

Nous nous sommes demandés si, pour chaque logement, la surface
intérieure en pieds carrés moins son sous-sol avait une influence plus
ou moins importante que la surface intérieure en pieds carrés dans son
sous-sol sur le prix de ces derniers. \textit{A priori}, nous supposons
que oui car un pied carré intérieur est plus important qu'un pied carré
intérieur dans son sous-sol. Les graphiques ci-dessous ne vont pas à
l'encontre de notre avis. Néanmoins, nous ne nous servirons pas des
variables \texttt{sqft\_above} et \texttt{sqft\_basement} car elles
interagissent avec la variable \texttt{sqft\_living}.

\begin{center}\includegraphics{E2-L3S2--1-_files/figure-latex/data.figure4-1} \end{center}

\newpage

\hypertarget{les-variables-yr_built-et-yr_renovated}{%
\subsubsection{\texorpdfstring{Les variables \texttt{yr\_built} et
\texttt{yr\_renovated}}{Les variables yr\_built et yr\_renovated}}\label{les-variables-yr_built-et-yr_renovated}}

Les variables \texttt{yr\_built} et \texttt{yr\_renovated} ne sont pas
intéressantes à analyser. Nous allons créer les variables
\texttt{life\_built} et \texttt{life\_renovated} :

\[\forall i \in [1,\ldots,I] \iff x_{i,23}=x_{i,2}-x_{i,15}\]
\[\forall i \in [1,\ldots,I] \iff x_{i,24}=x_{i,2}-x_{i,16}\]

\begin{Shaded}
\begin{Highlighting}[]
\NormalTok{data}\OperatorTok{$}\NormalTok{life_built <-}\StringTok{ }\NormalTok{data}\OperatorTok{$}\NormalTok{date }\OperatorTok{-}\StringTok{ }\NormalTok{data}\OperatorTok{$}\NormalTok{yr_built[ data}\OperatorTok{$}\NormalTok{yr_built }\OperatorTok{!=}\StringTok{ }\DecValTok{0}\NormalTok{ ]}

\NormalTok{data}\OperatorTok{$}\NormalTok{life_renovated <-}\StringTok{ }\NormalTok{data}\OperatorTok{$}\NormalTok{date }\OperatorTok{-}\StringTok{ }\NormalTok{data}\OperatorTok{$}\NormalTok{yr_renovated[ data}\OperatorTok{$}\NormalTok{yr_renovated }\OperatorTok{!=}\StringTok{ }\DecValTok{0}\NormalTok{ ]}
\end{Highlighting}
\end{Shaded}

\begin{table}[!h]

\caption{\label{tab:unnamed-chunk-6}Résumé des variables life-built et life-renovated :}
\centering
\begin{tabular}[t]{lrr}
\toprule
  & life\_built & life\_renovated\\
\midrule
\cellcolor{gray!6}{Moyenne} & \cellcolor{gray!6}{43.318} & \cellcolor{gray!6}{18.492}\\
Ecart-type & 29.375 & 15.511\\
\cellcolor{gray!6}{Minimum} & \cellcolor{gray!6}{-1.000} & \cellcolor{gray!6}{-1.000}\\
Q1 & 18.000 & 7.000\\
\cellcolor{gray!6}{Q2} & \cellcolor{gray!6}{40.000} & \cellcolor{gray!6}{15.000}\\
\addlinespace
Q3 & 63.000 & 27.000\\
\cellcolor{gray!6}{Maximum} & \cellcolor{gray!6}{115.000} & \cellcolor{gray!6}{81.000}\\
\bottomrule
\end{tabular}
\end{table}

Nous nous sommes demandés si, pour chaque logement, l'âge de la
construction et l'âge de la rénovation avaient une influence sur le prix
de ces derniers. \textit{A priori}, nous supposons que oui car plus le
logement est récent et plus le prix de ce dernier est élevé. Nous nous
servirons des variables \texttt{life\_built} et \texttt{life\_renovated}
même si les graphiques ci-dessous vont à l'encontre de notre avis.

\begin{center}\includegraphics{E2-L3S2--1-_files/figure-latex/data.figure5-1} \end{center}

\newpage

\hypertarget{les-variables-waterfront-view-condition-et-grade}{%
\subsubsection{\texorpdfstring{Les variables \texttt{waterfront},
\texttt{view}, \texttt{condition} et
\texttt{grade}}{Les variables waterfront, view, condition et grade}}\label{les-variables-waterfront-view-condition-et-grade}}

Les variables \texttt{waterfront}, \texttt{view}, \texttt{condition} et
\texttt{grade} sont des variables catégorielles :

\begin{Shaded}
\begin{Highlighting}[]
\NormalTok{data}\OperatorTok{$}\NormalTok{waterfront <-}\StringTok{ }\KeywordTok{as.factor}\NormalTok{( data}\OperatorTok{$}\NormalTok{waterfront )}

\NormalTok{data}\OperatorTok{$}\NormalTok{view <-}\StringTok{ }\KeywordTok{as.factor}\NormalTok{( data}\OperatorTok{$}\NormalTok{view )}

\NormalTok{data}\OperatorTok{$}\NormalTok{condition <-}\StringTok{ }\KeywordTok{as.factor}\NormalTok{( data}\OperatorTok{$}\NormalTok{condition )}

\NormalTok{data}\OperatorTok{$}\NormalTok{grade <-}\StringTok{ }\KeywordTok{as.factor}\NormalTok{( data}\OperatorTok{$}\NormalTok{grade )}
\end{Highlighting}
\end{Shaded}

\begin{table}[!h]

\caption{\label{tab:unnamed-chunk-8}Résumé des variables waterfront, view, condition et grade :}
\centering
\begin{tabular}[t]{lrrrr}
\toprule
  & waterfront & view & condition & grade\\
\midrule
\cellcolor{gray!6}{Moyenne} & \cellcolor{gray!6}{1.008} & \cellcolor{gray!6}{1.234} & \cellcolor{gray!6}{3.409} & \cellcolor{gray!6}{6.657}\\
Ecart-type & 0.087 & 0.766 & 0.651 & 1.175\\
\cellcolor{gray!6}{Minimum} & \cellcolor{gray!6}{1.000} & \cellcolor{gray!6}{1.000} & \cellcolor{gray!6}{1.000} & \cellcolor{gray!6}{1.000}\\
Q1 & 1.000 & 1.000 & 3.000 & 6.000\\
\cellcolor{gray!6}{Q2} & \cellcolor{gray!6}{1.000} & \cellcolor{gray!6}{1.000} & \cellcolor{gray!6}{3.000} & \cellcolor{gray!6}{6.000}\\
\addlinespace
Q3 & 1.000 & 1.000 & 4.000 & 7.000\\
\cellcolor{gray!6}{Maximum} & \cellcolor{gray!6}{2.000} & \cellcolor{gray!6}{5.000} & \cellcolor{gray!6}{5.000} & \cellcolor{gray!6}{12.000}\\
\bottomrule
\end{tabular}
\end{table}

Nous nous sommes demandés si, pour chaque logement, la vue sur la mer ou
non, le point de vue, l'état et le \textit{design} avaient une influence
sur le prix de ces derniers. \textit{A priori}, nous supposons que oui
car plus son point de vue est beau, plus l'état est bon, plus son
\textit{design} est classe et plus le prix de ce dernier est élevé. Nous
nous servirons des variables \texttt{waterfront}, \texttt{view},
\texttt{condition} et \texttt{grade} même si les graphiques ci-dessous
vont à l'encontre de notre avis.

\begin{center}\includegraphics{E2-L3S2--1-_files/figure-latex/data.figure6-1} \end{center}

\newpage

\hypertarget{les-variables-zipcode-lat-et-long}{%
\subsubsection{\texorpdfstring{Les variables \texttt{zipcode},
\texttt{lat} et
\texttt{long}}{Les variables zipcode, lat et long}}\label{les-variables-zipcode-lat-et-long}}

Les variables \texttt{zipcode}, \texttt{lat} et \texttt{long} sont
compliquées à analyser. Nous avons la variable \texttt{zipcode} qui est
comprise entre 98001 et 98199 et les variables \texttt{lat} et
\texttt{long} qui sont comprises, respectivement, entre 47.16 et 47.78
et -122.52 et -121.32. La base de données \texttt{BDD\_data.csv}
correspond grossièrement à la ville de Seattle dans l'État de Washington
aux États-Unis.

\begin{figure}
\centering
\includegraphics[width=0.75\textwidth,height=\textheight]{image (2).png}
\caption{Les États-Unis}
\end{figure}

\begin{figure}
\centering
\includegraphics[width=0.75\textwidth,height=\textheight]{image (3).png}
\caption{Zoom sur l'État de Washington}
\end{figure}

\newpage

Est-ce qu'un logement à l'ouest coûte moins cher qu'un logement à l'est
ou inversement ? Est-ce qu'un logement au nord coûte moins cher qu'un
logement au sud ou inversement ? Nous n'avons aucun \textit{a priori}.

\begin{figure}
\centering
\includegraphics[width=0.875\textwidth,height=\textheight]{image (4).png}
\caption{Zoom sur le nuage de points}
\end{figure}

D'après la carte ci dessus, nous supposons que plus un logement est au
centre de Seattle et plus le prix de ce dernier est élevé. Nous allons
zoomer sur le nuage de points le plus clair.

\begin{figure}
\centering
\includegraphics[width=0.75\textwidth,height=\textheight]{image (5).png}
\caption{Zoom sur le centre de Seattle}
\end{figure}

Nous ne nous servirons pas des variables \texttt{zipcode}, \texttt{lat}
et \texttt{long} car elles ne sont pas intéressantes à régresser
linéairement.

\newpage

\hypertarget{la-nouvelle-base-de-donnuxe9es-bdd_data.csv}{%
\section{\texorpdfstring{La nouvelle base de données
\texttt{BDD\_data.csv}}{La nouvelle base de données BDD\_data.csv}}\label{la-nouvelle-base-de-donnuxe9es-bdd_data.csv}}

\begin{table}[!h]

\caption{\label{tab:unnamed-chunk-9}Aperçu de la base de la nouvelle base de données :}
\centering
\resizebox{\linewidth}{!}{
\begin{tabular}[t]{lrrrrrrllllrr}
\toprule
  & price & bedrooms & bathrooms & sqft\_living & sqft\_lot & floors & waterfront & view & condition & grade & life\_built & life\_renovated\\
\midrule
\cellcolor{gray!6}{10609} & \cellcolor{gray!6}{367000} & \cellcolor{gray!6}{3} & \cellcolor{gray!6}{1.75} & \cellcolor{gray!6}{2000} & \cellcolor{gray!6}{12669} & \cellcolor{gray!6}{1} & \cellcolor{gray!6}{1} & \cellcolor{gray!6}{4} & \cellcolor{gray!6}{4} & \cellcolor{gray!6}{6} & \cellcolor{gray!6}{49} & \cellcolor{gray!6}{14}\\
1204 & 510000 & 3 & 1.75 & 1490 & 3800 & 1 & 1 & 1 & 3 & 5 & 102 & 5\\
\cellcolor{gray!6}{6262} & \cellcolor{gray!6}{848000} & \cellcolor{gray!6}{5} & \cellcolor{gray!6}{1.75} & \cellcolor{gray!6}{2290} & \cellcolor{gray!6}{4320} & \cellcolor{gray!6}{2} & \cellcolor{gray!6}{1} & \cellcolor{gray!6}{1} & \cellcolor{gray!6}{3} & \cellcolor{gray!6}{6} & \cellcolor{gray!6}{87} & \cellcolor{gray!6}{2}\\
9621 & 473000 & 3 & 1.00 & 1280 & 10000 & 1 & 1 & 1 & 4 & 6 & 60 & 57\\
\cellcolor{gray!6}{17361} & \cellcolor{gray!6}{840000} & \cellcolor{gray!6}{4} & \cellcolor{gray!6}{1.75} & \cellcolor{gray!6}{2330} & \cellcolor{gray!6}{4000} & \cellcolor{gray!6}{2} & \cellcolor{gray!6}{1} & \cellcolor{gray!6}{1} & \cellcolor{gray!6}{5} & \cellcolor{gray!6}{7} & \cellcolor{gray!6}{90} & \cellcolor{gray!6}{10}\\
\addlinespace
1400 & 408000 & 2 & 2.00 & 1200 & 3900 & 1 & 1 & 1 & 3 & 7 & 9 & 17\\
\cellcolor{gray!6}{5724} & \cellcolor{gray!6}{96500} & \cellcolor{gray!6}{3} & \cellcolor{gray!6}{1.00} & \cellcolor{gray!6}{840} & \cellcolor{gray!6}{12091} & \cellcolor{gray!6}{1} & \cellcolor{gray!6}{1} & \cellcolor{gray!6}{1} & \cellcolor{gray!6}{3} & \cellcolor{gray!6}{5} & \cellcolor{gray!6}{55} & \cellcolor{gray!6}{0}\\
18774 & 322500 & 3 & 2.00 & 1350 & 14200 & 1 & 1 & 1 & 3 & 6 & 25 & 50\\
\cellcolor{gray!6}{3965} & \cellcolor{gray!6}{364950} & \cellcolor{gray!6}{4} & \cellcolor{gray!6}{2.50} & \cellcolor{gray!6}{2310} & \cellcolor{gray!6}{8030} & \cellcolor{gray!6}{2} & \cellcolor{gray!6}{1} & \cellcolor{gray!6}{1} & \cellcolor{gray!6}{3} & \cellcolor{gray!6}{6} & \cellcolor{gray!6}{36} & \cellcolor{gray!6}{34}\\
9097 & 384000 & 6 & 3.00 & 2320 & 4502 & 1 & 1 & 1 & 4 & 6 & 27 & 19\\
\bottomrule
\end{tabular}}
\end{table}

\begin{table}[!h]

\caption{\label{tab:unnamed-chunk-10}Coefficients de corrélation linéaire de Pearson de la nouvelle base de données :}
\centering
\resizebox{\linewidth}{!}{
\begin{tabular}[t]{lrrrrrrrr}
\toprule
  & price & bedrooms & bathrooms & sqft\_living & sqft\_lot & floors & life\_built & life\_renovated\\
\midrule
\cellcolor{gray!6}{price} & \cellcolor{gray!6}{1.000} & \cellcolor{gray!6}{0.308} & \cellcolor{gray!6}{0.525} & \cellcolor{gray!6}{0.702} & \cellcolor{gray!6}{0.090} & \cellcolor{gray!6}{0.257} & \cellcolor{gray!6}{-0.054} & \cellcolor{gray!6}{-0.003}\\
bedrooms & 0.308 & 1.000 & 0.516 & 0.577 & 0.032 & 0.175 & -0.154 & -0.001\\
\cellcolor{gray!6}{bathrooms} & \cellcolor{gray!6}{0.525} & \cellcolor{gray!6}{0.516} & \cellcolor{gray!6}{1.000} & \cellcolor{gray!6}{0.755} & \cellcolor{gray!6}{0.088} & \cellcolor{gray!6}{0.501} & \cellcolor{gray!6}{-0.506} & \cellcolor{gray!6}{-0.003}\\
sqft\_living & 0.702 & 0.577 & 0.755 & 1.000 & 0.173 & 0.354 & -0.318 & -0.002\\
\cellcolor{gray!6}{sqft\_lot} & \cellcolor{gray!6}{0.090} & \cellcolor{gray!6}{0.032} & \cellcolor{gray!6}{0.088} & \cellcolor{gray!6}{0.173} & \cellcolor{gray!6}{1.000} & \cellcolor{gray!6}{-0.005} & \cellcolor{gray!6}{-0.053} & \cellcolor{gray!6}{0.008}\\
\addlinespace
floors & 0.257 & 0.175 & 0.501 & 0.354 & -0.005 & 1.000 & -0.490 & -0.009\\
\cellcolor{gray!6}{life\_built} & \cellcolor{gray!6}{-0.054} & \cellcolor{gray!6}{-0.154} & \cellcolor{gray!6}{-0.506} & \cellcolor{gray!6}{-0.318} & \cellcolor{gray!6}{-0.053} & \cellcolor{gray!6}{-0.490} & \cellcolor{gray!6}{1.000} & \cellcolor{gray!6}{-0.003}\\
life\_renovated & -0.003 & -0.001 & -0.003 & -0.002 & 0.008 & -0.009 & -0.003 & 1.000\\
\bottomrule
\end{tabular}}
\end{table}

D'après la table 8, nous supposons que les coefficients adossés aux
variables \texttt{bedrooms}, \texttt{bathrooms}, \texttt{sqft\_living},
\texttt{sqft\_lot} et \texttt{floors} sont strictement supérieurs à 0 et
que les coefficients adossés aux variables \texttt{life\_built} et
\texttt{life\_renovated} sont strictement inférieurs à 0.

\newpage

\hypertarget{mathcala_1-lanalyse-du-prix-dun-logement-en-fonction-de-sa-superficie-intuxe9rieure-en-pieds-carruxe9s-et-de-sa-superficie-extuxe9rieure-en-pieds-carruxe9s}{%
\section{\texorpdfstring{\(\mathcal{A}_{1}\) : l'analyse du prix d'un
logement en fonction de sa superficie intérieure en pieds carrés et de
sa superficie extérieure en pieds
carrés}{\textbackslash mathcal\{A\}\_\{1\} : l'analyse du prix d'un logement en fonction de sa superficie intérieure en pieds carrés et de sa superficie extérieure en pieds carrés}}\label{mathcala_1-lanalyse-du-prix-dun-logement-en-fonction-de-sa-superficie-intuxe9rieure-en-pieds-carruxe9s-et-de-sa-superficie-extuxe9rieure-en-pieds-carruxe9s}}

\[\mathcal{A}_{1}=\{\mathcal{M}_{1},\mathcal{M}_{3},\mathcal{M}_{4},\mathcal{M}_{4},\mathcal{M}_{5},\mathcal{M}_{6}\}\]

\begin{table}[!htbp] \centering 
  \caption{Régressions linéaires de l'analyse 1 :} 
  \label{} 
\small 
\begin{tabular}{@{\extracolsep{1pt}}lccc} 
\\[-1.8ex]\hline 
\hline \\[-1.8ex] 
 & \multicolumn{3}{c}{\textit{Dependent variable:}} \\ 
\cline{2-4} 
\\[-1.8ex] & \multicolumn{3}{c}{price} \\ 
\\[-1.8ex] & (1) & (2) & (3)\\ 
\hline \\[-1.8ex] 
 sqft\_living & 282.879$^{***}$ & 280.624$^{***}$ &  \\ 
  & (1.964) & (1.936) &  \\ 
  sqft\_lot & $-$0.289$^{***}$ &  & 0.795$^{***}$ \\ 
  & (0.044) &  & (0.060) \\ 
  Constant & $-$43,900.230$^{***}$ & $-$43,580.740$^{***}$ & 528,082.600$^{***}$ \\ 
  & (4,398.565) & (4,402.690) & (2,647.505) \\ 
 \hline \\[-1.8ex] 
Observations & 21,613 & 21,613 & 21,613 \\ 
R$^{2}$ & 0.494 & 0.493 & 0.008 \\ 
Adjusted R$^{2}$ & 0.494 & 0.493 & 0.008 \\ 
Residual Std. Error & 261,192.300 (df = 21610) & 261,452.900 (df = 21611) & 365,657.000 (df = 21611) \\ 
F Statistic & 10,543.990$^{***}$ (df = 2; 21610) & 21,001.910$^{***}$ (df = 1; 21611) & 175.140$^{***}$ (df = 1; 21611) \\ 
\hline 
\hline \\[-1.8ex] 
\textit{Note:}  & \multicolumn{3}{r}{$^{*}$p$<$0.1; $^{**}$p$<$0.05; $^{***}$p$<$0.01} \\ 
\end{tabular} 
\end{table}

\begin{table}[!htbp] \centering 
  \caption{Régressions linéaires de l'analyse 1 :} 
  \label{} 
\small 
\begin{tabular}{@{\extracolsep{1pt}}lccc} 
\\[-1.8ex]\hline 
\hline \\[-1.8ex] 
 & \multicolumn{3}{c}{\textit{Dependent variable:}} \\ 
\cline{2-4} 
\\[-1.8ex] & \multicolumn{3}{c}{log(price)} \\ 
\\[-1.8ex] & (1) & (2) & (3)\\ 
\hline \\[-1.8ex] 
 log(sqft\_living) & 0.873$^{***}$ & 0.837$^{***}$ &  \\ 
  & (0.007) & (0.006) &  \\ 
  log(sqft\_lot) & $-$0.053$^{***}$ &  & 0.080$^{***}$ \\ 
  & (0.003) &  & (0.004) \\ 
  Constant & 6.930$^{***}$ & 6.730$^{***}$ & 12.325$^{***}$ \\ 
  & (0.048) & (0.047) & (0.036) \\ 
 \hline \\[-1.8ex] 
Observations & 21,613 & 21,613 & 21,613 \\ 
R$^{2}$ & 0.463 & 0.456 & 0.019 \\ 
Adjusted R$^{2}$ & 0.463 & 0.455 & 0.019 \\ 
Residual Std. Error & 0.386 (df = 21610) & 0.389 (df = 21611) & 0.522 (df = 21611) \\ 
F Statistic & 9,310.386$^{***}$ (df = 2; 21610) & 18,079.140$^{***}$ (df = 1; 21611) & 417.860$^{***}$ (df = 1; 21611) \\ 
\hline 
\hline \\[-1.8ex] 
\textit{Note:}  & \multicolumn{3}{r}{$^{*}$p$<$0.1; $^{**}$p$<$0.05; $^{***}$p$<$0.01} \\ 
\end{tabular} 
\end{table}

\newpage

\hypertarget{analyse-des-ruxe9sultats}{%
\subsection{Analyse des résultats}\label{analyse-des-ruxe9sultats}}

\begin{table}[!h]

\caption{\label{tab:unnamed-chunk-12}Aperçu de la base de données de l'analyse 1 :}
\centering
\begin{tabular}[t]{lrrr}
\toprule
  & price & sqft\_living & sqft\_lot\\
\midrule
\cellcolor{gray!6}{4789} & \cellcolor{gray!6}{716100} & \cellcolor{gray!6}{1640} & \cellcolor{gray!6}{4000}\\
13910 & 695000 & 2650 & 9990\\
\cellcolor{gray!6}{8533} & \cellcolor{gray!6}{350000} & \cellcolor{gray!6}{900} & \cellcolor{gray!6}{6380}\\
11667 & 650000 & 1910 & 16532\\
\cellcolor{gray!6}{12078} & \cellcolor{gray!6}{285000} & \cellcolor{gray!6}{1930} & \cellcolor{gray!6}{7200}\\
\addlinespace
18904 & 435000 & 670 & 1800\\
\cellcolor{gray!6}{16764} & \cellcolor{gray!6}{829000} & \cellcolor{gray!6}{2690} & \cellcolor{gray!6}{10443}\\
12123 & 251000 & 840 & 4495\\
\cellcolor{gray!6}{19044} & \cellcolor{gray!6}{872000} & \cellcolor{gray!6}{2860} & \cellcolor{gray!6}{40284}\\
8676 & 242000 & 2340 & 7494\\
\bottomrule
\end{tabular}
\end{table}

\begin{table}[!h]

\caption{\label{tab:unnamed-chunk-13}Comparaison des performances des régressions linéaires de l'analyse 1}
\centering
\resizebox{\linewidth}{!}{
\begin{tabular}[t]{llrrrrrrr}
\toprule
Name & Model & AIC & BIC & R2 & R2\_adjusted & RMSE & Sigma & Performance\_Score\\
\midrule
\cellcolor{gray!6}{model4} & \cellcolor{gray!6}{lm} & \cellcolor{gray!6}{20195.57} & \cellcolor{gray!6}{20227.50} & \cellcolor{gray!6}{0.463} & \cellcolor{gray!6}{0.463} & \cellcolor{gray!6}{0.386} & \cellcolor{gray!6}{0.386} & \cellcolor{gray!6}{0.979}\\
model5 & lm & 20486.98 & 20510.93 & 0.456 & 0.455 & 0.389 & 0.389 & 0.974\\
\cellcolor{gray!6}{model6} & \cellcolor{gray!6}{lm} & \cellcolor{gray!6}{33211.62} & \cellcolor{gray!6}{33235.57} & \cellcolor{gray!6}{0.019} & \cellcolor{gray!6}{0.019} & \cellcolor{gray!6}{0.522} & \cellcolor{gray!6}{0.522} & \cellcolor{gray!6}{0.667}\\
model1 & lm & 600498.46 & 600530.39 & 0.494 & 0.494 & 261174.188 & 261192.316 & 0.437\\
\cellcolor{gray!6}{model2} & \cellcolor{gray!6}{lm} & \cellcolor{gray!6}{600540.57} & \cellcolor{gray!6}{600564.51} & \cellcolor{gray!6}{0.493} & \cellcolor{gray!6}{0.493} & \cellcolor{gray!6}{261440.790} & \cellcolor{gray!6}{261452.888} & \cellcolor{gray!6}{0.436}\\
\addlinespace
model3 & lm & 615040.37 & 615064.31 & 0.008 & 0.008 & 365640.079 & 365656.998 & 0.000\\
\bottomrule
\end{tabular}}
\end{table}

Nous allons analyser le modèle 5.

\[\mathcal{M}_{5}:\log(y_{i})=\alpha+\beta_{6}\log(x_{i,6})+z_{i} \iff \mathcal{M}_{5}:\log(y_{i})=6,730+0,837\log(x_{i,6})+z_{i}\]

Pour \(\alpha\), nous avons moins de 0,1\% de chance de se tromper en
rejetant l'hypothèse nulle. Par conséquent, nous rejetons l'hypothèse
nulle et nous en déduisons que la constante du modèle 5 est différente
de 0 au seuil de 0,1\%.

Pour \(\beta_{6}\), nous avons moins de 0,1\% de chance de se tromper en
rejetant l'hypothèse nulle. Par conséquent, nous rejetons l'hypothèse
nulle et nous en déduisons que le coefficient du modèle 5 de la variable
explicative \texttt{sqft\_living} est différent de 0 au seuil de 0,1\%.

Nous avons \(R^{2}=0,456\) dans le modèle 5 donc 45,6\% de la variation
du prix des logements est expliquée par la surface intérieure de ces
derniers.

Pour \(F\), nous avons moins de 0,1\% de chance de se tromper en
rejetant l'hypothèse d'absence globale de significativité. Par
conséquent, nous rejetons l'hypothèse d'absence globale de
significativité et nous en déduisons que le modèle 1 est globalement
avéré au seuil de 0,1\%.

\newpage

\hypertarget{analyse-des-ruxe9sidus}{%
\subsection{Analyse des résidus}\label{analyse-des-ruxe9sidus}}

\begin{center}\includegraphics{E2-L3S2--1-_files/figure-latex/data.figure7-1} \end{center}

\begin{figure}
\centering
\includegraphics[width=1\textwidth,height=\textheight]{plot (3).png}
\caption{Graphiques de l'analyse des résidus}
\end{figure}

\newpage

\begin{table}[!h]

\caption{\label{tab:unnamed-chunk-14}Tests de l'analyse des résidus :}
\centering
\begin{tabular}[t]{lrrr}
\toprule
  & Harrison\_McCabe & Breusch\_Pagan & Durbin\_Watson\\
\midrule
\cellcolor{gray!6}{p-value} & \cellcolor{gray!6}{0.971} & \cellcolor{gray!6}{0.000} & \cellcolor{gray!6}{0.055}\\
statistic & 0.509 & 27.552 & 1.978\\
\bottomrule
\end{tabular}
\end{table}

Pour \(DW\) nous avons 5,5\% de chance de se tomper en rejetant
l'hypothèse d'absence d'autocorrélation. Par conséquent, nous conservons
l'hypothèse d'absence d'autocorrélation et nous en déduisons que les
résidus ne sont pas autocorrélés au seuil de 0,1\%.

Nous observons que les résidus sont globalement répartis le long de la
droite. Nous en déduisons que les résidus suivent une loi normale.

Pour \(BP\), nous avons moins de 0,1\% de chance de se tomper en
rejetant l'hypothèse d'homoscédasticité. Par conséquent, nous rejetons
l'hypothèse d'homoscédasticité et nous en déduisons que les résidus sont
hétéroscédastiques au seuil de 0,1\%.

Nous observons l'existence de résidus qui ne sont globalement pas
linéaires et qui dépassent la distance de Cook. Nous en déduisons que le
logement 15871 et dans une moindre mesure les logements 9255, 8338, 7253
et 6403 ont un résidu aberrant.

\begin{Shaded}
\begin{Highlighting}[]
\NormalTok{data.analyse1 <-}\StringTok{ }\NormalTok{data[ }\OperatorTok{-}\StringTok{ }\KeywordTok{c}\NormalTok{( }\DecValTok{15871}\NormalTok{,}
                            \DecValTok{9255}\NormalTok{,}
                            \DecValTok{8338}\NormalTok{,}
                            \DecValTok{7253}\NormalTok{,}
                            \DecValTok{6403}\NormalTok{ ), ]}
\end{Highlighting}
\end{Shaded}

Les hypothèses nécessaires des résultats et des résidus sont avérés au
seuil de 0,1\%. Notre constante, notre coefficient, notre
significativité, notre linéarité des résidus, notre absence
d'autocorrélation des résidus et notre appartenance à une loi normale
des résidus sont avérés au seuil de 0,1\%. L'hypothèse
d'hétéroscédasticité ne pose pas de problème car le prix des logements
et la variance du prix des logements ne sont pas homogènes.

En conclusion, de manière biaisé par la distance de Cook, nous avons
0,1\% de chance de se tromper en affirmant que plus la surface
intérieure en pieds carrés est élevée et plus le prix des logements est
élevé selon la relation \(\mathcal{M}_{5}\).

\newpage

\hypertarget{analyse-du-moduxe8le-5-moins-les-ruxe9sidus-aberrants-pour-la-distance-de-cook}{%
\subsection{Analyse du modèle 5 moins les résidus aberrants pour la
distance de
Cook}\label{analyse-du-moduxe8le-5-moins-les-ruxe9sidus-aberrants-pour-la-distance-de-cook}}

\begin{table}[!htbp] \centering 
  \caption{Régression linéaire du modèle 5 moins les résidus aberrants :} 
  \label{} 
\small 
\begin{tabular}{@{\extracolsep{1pt}}lc} 
\\[-1.8ex]\hline 
\hline \\[-1.8ex] 
 & \multicolumn{1}{c}{\textit{Dependent variable:}} \\ 
\cline{2-2} 
\\[-1.8ex] & log(price) \\ 
\hline \\[-1.8ex] 
 log(sqft\_living) & 0.835$^{***}$ \\ 
  & (0.006) \\ 
  Constant & 6.740$^{***}$ \\ 
  & (0.047) \\ 
 \hline \\[-1.8ex] 
Observations & 21,608 \\ 
R$^{2}$ & 0.455 \\ 
Adjusted R$^{2}$ & 0.455 \\ 
Residual Std. Error & 0.388 (df = 21606) \\ 
F Statistic & 18,022.850$^{***}$ (df = 1; 21606) \\ 
\hline 
\hline \\[-1.8ex] 
\textit{Note:}  & \multicolumn{1}{r}{$^{*}$p$<$0.1; $^{**}$p$<$0.05; $^{***}$p$<$0.01} \\ 
\end{tabular} 
\end{table}

\begin{table}[!h]

\caption{\label{tab:unnamed-chunk-17}Performances de la régression linéaire du modèle 5 moins les résidus aberrants :}
\centering
\resizebox{\linewidth}{!}{
\begin{tabular}[t]{llrrrrrrr}
\toprule
Name & Model & AIC & BIC & R2 & R2\_adjusted & RMSE & Sigma & Performance\_Score\\
\midrule
\cellcolor{gray!6}{model5} & \cellcolor{gray!6}{lm} & \cellcolor{gray!6}{20444.54} & \cellcolor{gray!6}{20468.48} & \cellcolor{gray!6}{0.455} & \cellcolor{gray!6}{0.455} & \cellcolor{gray!6}{0.388} & \cellcolor{gray!6}{0.388} & \cellcolor{gray!6}{NaN}\\
\bottomrule
\end{tabular}}
\end{table}

\hypertarget{analyse-des-ruxe9sultats-1}{%
\subsubsection{Analyse des résultats}\label{analyse-des-ruxe9sultats-1}}

\[\mathcal{M}_{5}:\log(y_{i})=\alpha+\beta_{6}\log(x_{i,6})+z_{i} \iff \mathcal{M}_{5}:\log(y_{i})=6,74+0,835\log(x_{i,6})+z_{i}\]

Pour \(\alpha\), nous avons moins de 0,1\% de chance de se tromper en
rejetant l'hypothèse nulle. Par conséquent, nous rejetons l'hypothèse
nulle et nous en déduisons que la constante du modèle 5 est différente
de 0 au seuil de 0,1\%.

Pour \(\beta_{6}\), nous avons moins de 0,1\% de chance de se tromper en
rejetant l'hypothèse nulle. Par conséquent, nous rejetons l'hypothèse
nulle et nous en déduisons que le coefficient du modèle 5 de la variable
explicative \texttt{sqft\_living} est différent de 0 au seuil de 0,1\%.

Nous avons \(R^{2}=0,455\) dans le modèle 5 donc 45,5\% de la variation
du prix des logements est expliquée par la surface intérieure de ces
derniers.

Pour \(F\), nous avons moins de 0,1\% de chance de se tromper en
rejetant l'hypothèse d'absence globale de significativité. Par
conséquent, nous rejetons l'hypothèse d'absence globale de
significativité et nous en déduisons que le modèle 1 est globalement
avéré au seuil de 0,1\%.

\newpage

\hypertarget{analyse-des-ruxe9sidus-1}{%
\subsubsection{Analyse des résidus}\label{analyse-des-ruxe9sidus-1}}

\begin{center}\includegraphics{E2-L3S2--1-_files/figure-latex/data.figure12-1} \end{center}

\begin{figure}
\centering
\includegraphics[width=1\textwidth,height=\textheight]{plot (10).png}
\caption{Graphiques de l'analyse des résidus}
\end{figure}

\newpage

\begin{table}[!h]

\caption{\label{tab:unnamed-chunk-18}Tests de l'analyse des résidus :}
\centering
\begin{tabular}[t]{lrrr}
\toprule
  & Harrison\_McCabe & Breusch\_Pagan & Durbin\_Watson\\
\midrule
\cellcolor{gray!6}{p-value} & \cellcolor{gray!6}{0.966} & \cellcolor{gray!6}{0.000} & \cellcolor{gray!6}{0.053}\\
statistic & 0.508 & 23.913 & 1.978\\
\bottomrule
\end{tabular}
\end{table}

Pour \(DW\) nous avons 5,3\% de chance de se tomper en rejetant
l'hypothèse d'absence d'autocorrélation. Par conséquent, nous conservons
l'hypothèse d'absence d'autocorrélation et nous en déduisons que les
résidus ne sont pas autocorrélés au seuil de 0,1\%.

Nous observons que les résidus sont globalement répartis le long de la
droite. Nous en déduisons que les résidus suivent une loi normale.

Pour \(BP\), nous avons moins de 0,1\% de chance de se tomper en
rejetant l'hypothèse d'homoscédasticité. Par conséquent, nous rejetons
l'hypothèse d'homoscédasticité et nous en déduisons que les résidus sont
hétéroscédastiques au seuil de 0,1\%.

Nous observons l'existence de résidus qui ne sont globalement pas
linéaires et qui tendent vers la distance de Cook. Nous en déduisons
que, dans une moindre mesure, les logements 10447, 4412, 3915, 1449 et
1316 ont un résidu aberrant.

Nous avons 0,1\% de chance de se tromper en affirmant que plus la
surface intérieure en pieds carrés est élevée et plus le prix des
logements est élevé selon la relation \(\mathcal{M}_{5}\).

\begin{center}\includegraphics{E2-L3S2--1-_files/figure-latex/data.figure13-1} \end{center}

\newpage

\hypertarget{mathcala_1h-lanalyse-de-lhuxe9tuxe9roscuxe9dasticituxe9}{%
\subsection{\texorpdfstring{\(\mathcal{A}_{1}^{H}\) : l'analyse de
l'hétéroscédasticité}{\textbackslash mathcal\{A\}\_\{1\}\^{}\{H\} : l'analyse de l'hétéroscédasticité}}\label{mathcala_1h-lanalyse-de-lhuxe9tuxe9roscuxe9dasticituxe9}}

Nous allons essayer de conserver l'hypothèse d'homoscédasticité des
résidus en segmentant le prix des logements. Nous allons créer de
nouvelles bases de données en fonction du minimum, du maximum et de la
médiane :

\begin{Shaded}
\begin{Highlighting}[]
\NormalTok{data.analyse1.min_q1 <-}\StringTok{ }\NormalTok{data.analyse1[ data.analyse1}\OperatorTok{$}\NormalTok{price }\OperatorTok{>=}\StringTok{ }\DecValTok{75000} \OperatorTok{&}
\StringTok{                                         }\NormalTok{data.analyse1}\OperatorTok{$}\NormalTok{price }\OperatorTok{<=}\StringTok{ }\DecValTok{321950}\NormalTok{, ]}\CommentTok{# modèle 1,}

\NormalTok{data.analyse1.min_q2 <-}\StringTok{ }\NormalTok{data.analyse1[ data.analyse1}\OperatorTok{$}\NormalTok{price }\OperatorTok{>=}\StringTok{ }\DecValTok{75000} \OperatorTok{&}
\StringTok{                                         }\NormalTok{data.analyse1}\OperatorTok{$}\NormalTok{price }\OperatorTok{<=}\StringTok{ }\DecValTok{450000}\NormalTok{, ]}\CommentTok{# modèle 2,}

\NormalTok{data.analyse1.min_q3 <-}\StringTok{ }\NormalTok{data.analyse1[ data.analyse1}\OperatorTok{$}\NormalTok{price }\OperatorTok{>=}\StringTok{ }\DecValTok{75000} \OperatorTok{&}
\StringTok{                                         }\NormalTok{data.analyse1}\OperatorTok{$}\NormalTok{price }\OperatorTok{<=}\StringTok{ }\DecValTok{645000}\NormalTok{, ]}\CommentTok{# modèle 3,}

\NormalTok{data.analyse1.q1_q2 <-}\StringTok{ }\NormalTok{data.analyse1[ data.analyse1}\OperatorTok{$}\NormalTok{price }\OperatorTok{>=}\StringTok{ }\DecValTok{321950} \OperatorTok{&}
\StringTok{                                        }\NormalTok{data.analyse1}\OperatorTok{$}\NormalTok{price }\OperatorTok{<=}\StringTok{ }\DecValTok{450000}\NormalTok{, ]}\CommentTok{# modèle 4,}

\NormalTok{data.analyse1.q2_q3 <-}\StringTok{ }\NormalTok{data.analyse1[ data.analyse1}\OperatorTok{$}\NormalTok{price }\OperatorTok{>=}\StringTok{ }\DecValTok{450000} \OperatorTok{&}
\StringTok{                                        }\NormalTok{data.analyse1}\OperatorTok{$}\NormalTok{price }\OperatorTok{<=}\StringTok{ }\DecValTok{645000}\NormalTok{, ]}\CommentTok{# modèle 5,}

\NormalTok{data.analyse1.q1_q3 <-}\StringTok{ }\NormalTok{data.analyse1[ data.analyse1}\OperatorTok{$}\NormalTok{price }\OperatorTok{>=}\StringTok{ }\DecValTok{321950} \OperatorTok{&}
\StringTok{                                        }\NormalTok{data.analyse1}\OperatorTok{$}\NormalTok{price }\OperatorTok{<=}\StringTok{ }\DecValTok{645000}\NormalTok{, ]}\CommentTok{# modèle 6,}
\end{Highlighting}
\end{Shaded}

Nous pensons que les logements les plus chers ont une influence sur
l'homogénéité du prix de ces derniers. En effet, le prix d'un logement
varie grandement lorsque ce dernier est très élevé. Quelle est la
différence entre un logement coûtant 1 000 000 dollars et un logement
coûtant 1 500 000 dollars ? D'après les interviews \textit{Asking Price}
de la chaîne YouTube \textbf{Architectural Digest}, nous nous trompons
communément sur le prix d'un logement en fonction de sa surface
intérieure, de sa surface extérieure, etc. En effet, nous omettons le
\textit{standing}. Par conséquent, nous allons segmenter la base de
données \texttt{BDD\_data.csv} et comparer les résultats et les résidus
des nouvelles bases de données.

\newpage

\[\mathcal{A}_{1}^{H}=\{\mathcal{M}_{1}^{H},\mathcal{M}_{2}^{H},\mathcal{M}_{3}^{H},\mathcal{M}_{4}^{H},\mathcal{M}_{5}^{H},\mathcal{M}_{6}^{H}\}\]

\begin{table}[!htbp] \centering 
  \caption{Régressions linéaires supplémentaires de l'analyse 1 :} 
  \label{} 
\small 
\begin{tabular}{@{\extracolsep{1pt}}lccc} 
\\[-1.8ex]\hline 
\hline \\[-1.8ex] 
 & \multicolumn{3}{c}{\textit{Dependent variable:}} \\ 
\cline{2-4} 
\\[-1.8ex] & \multicolumn{3}{c}{log(price)} \\ 
\\[-1.8ex] & (1) & (2) & (3)\\ 
\hline \\[-1.8ex] 
 log(sqft\_living) & 0.276$^{***}$ & 0.301$^{***}$ & 0.447$^{***}$ \\ 
  & (0.008) & (0.007) & (0.007) \\ 
  Constant & 10.416$^{***}$ & 10.427$^{***}$ & 9.508$^{***}$ \\ 
  & (0.059) & (0.052) & (0.050) \\ 
 \hline \\[-1.8ex] 
Observations & 5,404 & 10,864 & 16,239 \\ 
R$^{2}$ & 0.174 & 0.144 & 0.212 \\ 
Adjusted R$^{2}$ & 0.174 & 0.144 & 0.212 \\ 
Residual Std. Error & 0.200 (df = 5402) & 0.259 (df = 10862) & 0.313 (df = 16237) \\ 
F Statistic & 1,140.195$^{***}$ (df = 1; 5402) & 1,828.210$^{***}$ (df = 1; 10862) & 4,376.828$^{***}$ (df = 1; 16237) \\ 
\hline 
\hline \\[-1.8ex] 
\textit{Note:}  & \multicolumn{3}{r}{$^{*}$p$<$0.1; $^{**}$p$<$0.05; $^{***}$p$<$0.01} \\ 
\end{tabular} 
\end{table}

\begin{table}[!htbp] \centering 
  \caption{Régressions linéaires supplémentaires de l'analyse 1 :} 
  \label{} 
\small 
\begin{tabular}{@{\extracolsep{1pt}}lccc} 
\\[-1.8ex]\hline 
\hline \\[-1.8ex] 
 & \multicolumn{3}{c}{\textit{Dependent variable:}} \\ 
\cline{2-4} 
\\[-1.8ex] & \multicolumn{3}{c}{log(price)} \\ 
\\[-1.8ex] & (1) & (2) & (3)\\ 
\hline \\[-1.8ex] 
 log(sqft\_living) & 0.019$^{***}$ & 0.067$^{***}$ & 0.153$^{***}$ \\ 
  & (0.004) & (0.004) & (0.005) \\ 
  Constant & 12.716$^{***}$ & 12.679$^{***}$ & 11.873$^{***}$ \\ 
  & (0.029) & (0.032) & (0.039) \\ 
 \hline \\[-1.8ex] 
Observations & 5,461 & 5,547 & 10,836 \\ 
R$^{2}$ & 0.004 & 0.044 & 0.073 \\ 
Adjusted R$^{2}$ & 0.004 & 0.044 & 0.073 \\ 
Residual Std. Error & 0.100 (df = 5459) & 0.101 (df = 5545) & 0.188 (df = 10834) \\ 
F Statistic & 24.103$^{***}$ (df = 1; 5459) & 254.578$^{***}$ (df = 1; 5545) & 852.732$^{***}$ (df = 1; 10834) \\ 
\hline 
\hline \\[-1.8ex] 
\textit{Note:}  & \multicolumn{3}{r}{$^{*}$p$<$0.1; $^{**}$p$<$0.05; $^{***}$p$<$0.01} \\ 
\end{tabular} 
\end{table}

\begin{table}[!h]

\caption{\label{tab:unnamed-chunk-21}Comparaison des performances des régressions linéaires supplémentaires de l'analyse 1}
\centering
\resizebox{\linewidth}{!}{
\begin{tabular}[t]{llrrrrrrr}
\toprule
Name & Model & AIC & BIC & R2 & R2\_adjusted & RMSE & Sigma & Performance\_Score\\
\midrule
\cellcolor{gray!6}{model5} & \cellcolor{gray!6}{lm} & \cellcolor{gray!6}{-9655.489} & \cellcolor{gray!6}{-9635.626} & \cellcolor{gray!6}{0.044} & \cellcolor{gray!6}{0.044} & \cellcolor{gray!6}{0.101} & \cellcolor{gray!6}{0.101} & \cellcolor{gray!6}{0.727}\\
model4 & lm & -9668.215 & -9648.399 & 0.004 & 0.004 & 0.100 & 0.100 & 0.667\\
\cellcolor{gray!6}{model1} & \cellcolor{gray!6}{lm} & \cellcolor{gray!6}{-2047.937} & \cellcolor{gray!6}{-2028.152} & \cellcolor{gray!6}{0.174} & \cellcolor{gray!6}{0.174} & \cellcolor{gray!6}{0.200} & \cellcolor{gray!6}{0.200} & \cellcolor{gray!6}{0.642}\\
model6 & lm & -5422.781 & -5400.909 & 0.073 & 0.073 & 0.188 & 0.188 & 0.560\\
\cellcolor{gray!6}{model2} & \cellcolor{gray!6}{lm} & \cellcolor{gray!6}{1499.602} & \cellcolor{gray!6}{1521.482} & \cellcolor{gray!6}{0.144} & \cellcolor{gray!6}{0.144} & \cellcolor{gray!6}{0.259} & \cellcolor{gray!6}{0.259} & \cellcolor{gray!6}{0.435}\\
\addlinespace
model3 & lm & 8385.804 & 8408.889 & 0.212 & 0.212 & 0.313 & 0.313 & 0.333\\
\bottomrule
\end{tabular}}
\end{table}

\newpage

\begin{table}[!h]

\caption{\label{tab:unnamed-chunk-22}Tests de l'analyse des résidus du modèle 1 :}
\centering
\begin{tabular}[t]{lrrr}
\toprule
  & Harrison\_McCabe & Breusch\_Pagan & Durbin\_Watson\\
\midrule
\cellcolor{gray!6}{p-value} & \cellcolor{gray!6}{0.017} & \cellcolor{gray!6}{0.000} & \cellcolor{gray!6}{0.263}\\
statistic & 0.479 & 587.741 & 1.983\\
\bottomrule
\end{tabular}
\end{table}

\begin{table}[!h]

\caption{\label{tab:unnamed-chunk-22}Tests de l'analyse des résidus du modèle 2 :}
\centering
\begin{tabular}[t]{lrrr}
\toprule
  & Harrison\_McCabe & Breusch\_Pagan & Durbin\_Watson\\
\midrule
\cellcolor{gray!6}{p-value} & \cellcolor{gray!6}{0.855} & \cellcolor{gray!6}{0.000} & \cellcolor{gray!6}{0.319}\\
statistic & 0.507 & 1136.047 & 1.991\\
\bottomrule
\end{tabular}
\end{table}

\begin{table}[!h]

\caption{\label{tab:unnamed-chunk-22}Tests de l'analyse des résidus du modèle 3 :}
\centering
\begin{tabular}[t]{lrrr}
\toprule
  & Harrison\_McCabe & Breusch\_Pagan & Durbin\_Watson\\
\midrule
\cellcolor{gray!6}{p-value} & \cellcolor{gray!6}{0.935} & \cellcolor{gray!6}{0.000} & \cellcolor{gray!6}{0.325}\\
statistic & 0.509 & 1010.537 & 1.993\\
\bottomrule
\end{tabular}
\end{table}

\begin{table}[!h]

\caption{\label{tab:unnamed-chunk-22}Tests de l'analyse des résidus du modèle 4 :}
\centering
\begin{tabular}[t]{lrrr}
\toprule
  & Harrison\_McCabe & Breusch\_Pagan & Durbin\_Watson\\
\midrule
\cellcolor{gray!6}{p-value} & \cellcolor{gray!6}{0.745} & \cellcolor{gray!6}{0.312} & \cellcolor{gray!6}{0.365}\\
statistic & 0.507 & 1.023 & 1.991\\
\bottomrule
\end{tabular}
\end{table}

\begin{table}[!h]

\caption{\label{tab:unnamed-chunk-22}Tests de l'analyse des résidus du modèle 5 :}
\centering
\begin{tabular}[t]{lrrr}
\toprule
  & Harrison\_McCabe & Breusch\_Pagan & Durbin\_Watson\\
\midrule
\cellcolor{gray!6}{p-value} & \cellcolor{gray!6}{0.536} & \cellcolor{gray!6}{0.001} & \cellcolor{gray!6}{0.895}\\
statistic & 0.501 & 11.256 & 2.034\\
\bottomrule
\end{tabular}
\end{table}

\begin{table}[!h]

\caption{\label{tab:unnamed-chunk-22}Tests de l'analyse des résidus du modèle 6 :}
\centering
\begin{tabular}[t]{lrrr}
\toprule
  & Harrison\_McCabe & Breusch\_Pagan & Durbin\_Watson\\
\midrule
\cellcolor{gray!6}{p-value} & \cellcolor{gray!6}{0.521} & \cellcolor{gray!6}{0.00} & \cellcolor{gray!6}{0.556}\\
statistic & 0.500 & 70.24 & 2.003\\
\bottomrule
\end{tabular}
\end{table}

En conclusion, nous conservons l'hypothèse d'homoscédasticité et nous en
déduisons que les résidus sont homoscédastiques au seuil de 0,1\%
uniquement dans le modèle 4 c'est à dire pour les logements coûtant
entre 321 950 dollars et 450 000 dollars. Est-ce que le prix d'un
logement varie grandement même lorsque ce dernier n'est pas très élevé ?
\textit{A priori}, nous supposons que oui concernant la ville de Seattle
dans l'État de Washington aux États-Unis.

\newpage

\hypertarget{mathcala_2-lanalyse-du-prix-dun-logement-en-fonction-de-son-nombre-de-chambres-de-son-nombre-de-salles-de-bain-et-de-son-nombre-duxe9tages}{%
\section{\texorpdfstring{\(\mathcal{A}_{2}\) : l'analyse du prix d'un
logement en fonction de son nombre de chambres, de son nombre de salles
de bain et de son nombre
d'étages}{\textbackslash mathcal\{A\}\_\{2\} : l'analyse du prix d'un logement en fonction de son nombre de chambres, de son nombre de salles de bain et de son nombre d'étages}}\label{mathcala_2-lanalyse-du-prix-dun-logement-en-fonction-de-son-nombre-de-chambres-de-son-nombre-de-salles-de-bain-et-de-son-nombre-duxe9tages}}

\[\mathcal{A}_{2}=\{\mathcal{M}_{1},\mathcal{M}_{2},\mathcal{M}_{3},\mathcal{M}_{4}\}\]

\begin{table}[!htbp] \centering 
  \caption{Régressions linéaires de l'analyse 2 :} 
  \label{} 
\small 
\begin{tabular}{@{\extracolsep{1pt}}lcc} 
\\[-1.8ex]\hline 
\hline \\[-1.8ex] 
 & \multicolumn{2}{c}{\textit{Dependent variable:}} \\ 
\cline{2-3} 
\\[-1.8ex] & log(price) & price \\ 
\\[-1.8ex] & (1) & (2)\\ 
\hline \\[-1.8ex] 
 bedrooms & 0.049$^{***}$ & 20,024.380$^{***}$ \\ 
  & (0.004) & (2,680.838) \\ 
  bathrooms & 0.327$^{***}$ & 238,461.400$^{***}$ \\ 
  & (0.005) & (3,681.888) \\ 
  floors & 0.055$^{***}$ & $-$1,737.472 \\ 
  & (0.006) & (4,569.452) \\ 
  Constant & 12.109$^{***}$ & $-$29,102.610$^{***}$ \\ 
  & (0.013) & (9,209.213) \\ 
 \hline \\[-1.8ex] 
Observations & 21,613 & 21,613 \\ 
R$^{2}$ & 0.311 & 0.278 \\ 
Adjusted R$^{2}$ & 0.310 & 0.278 \\ 
Residual Std. Error (df = 21609) & 0.437 & 312,040.000 \\ 
F Statistic (df = 3; 21609) & 3,244.010$^{***}$ & 2,769.094$^{***}$ \\ 
\hline 
\hline \\[-1.8ex] 
\textit{Note:}  & \multicolumn{2}{r}{$^{*}$p$<$0.1; $^{**}$p$<$0.05; $^{***}$p$<$0.01} \\ 
\end{tabular} 
\end{table}

\begin{table}[!htbp] \centering 
  \caption{Régressions linéaires de l'analyse 2 :} 
  \label{} 
\small 
\begin{tabular}{@{\extracolsep{1pt}}lcc} 
\\[-1.8ex]\hline 
\hline \\[-1.8ex] 
 & \multicolumn{2}{c}{\textit{Dependent variable:}} \\ 
\cline{2-3} 
\\[-1.8ex] & log(price) & price \\ 
\\[-1.8ex] & (1) & (2)\\ 
\hline \\[-1.8ex] 
 log(sqft\_living) & 0.839$^{***}$ &  \\ 
  & (0.010) &  \\ 
  sqft\_living &  & 309.393$^{***}$ \\ 
  &  & (3.087) \\ 
  bedrooms & $-$0.070$^{***}$ & $-$57,847.960$^{***}$ \\ 
  & (0.004) & (2,347.323) \\ 
  bathrooms & 0.053$^{***}$ & 7,853.522$^{**}$ \\ 
  & (0.006) & (3,814.223) \\ 
  floors & 0.044$^{***}$ & 200.497 \\ 
  & (0.006) & (3,775.505) \\ 
  Constant & 6.772$^{***}$ & 74,669.670$^{***}$ \\ 
  & (0.067) & (7,679.122) \\ 
 \hline \\[-1.8ex] 
Observations & 21,613 & 21,613 \\ 
R$^{2}$ & 0.471 & 0.507 \\ 
Adjusted R$^{2}$ & 0.470 & 0.507 \\ 
Residual Std. Error (df = 21608) & 0.383 & 257,819.300 \\ 
F Statistic (df = 4; 21608) & 4,801.148$^{***}$ & 5,553.623$^{***}$ \\ 
\hline 
\hline \\[-1.8ex] 
\textit{Note:}  & \multicolumn{2}{r}{$^{*}$p$<$0.1; $^{**}$p$<$0.05; $^{***}$p$<$0.01} \\ 
\end{tabular} 
\end{table}

\newpage

\hypertarget{analyse-des-ruxe9sultats-2}{%
\subsection{Analyse des résultats}\label{analyse-des-ruxe9sultats-2}}

\begin{table}[!h]

\caption{\label{tab:unnamed-chunk-24}Aperçu de la base de données de l'analyse 2 :}
\centering
\begin{tabular}[t]{lrrrrr}
\toprule
  & price & bedrooms & bathrooms & sqft\_living & floors\\
\midrule
\cellcolor{gray!6}{19054} & \cellcolor{gray!6}{879950} & \cellcolor{gray!6}{4} & \cellcolor{gray!6}{2.25} & \cellcolor{gray!6}{3500} & \cellcolor{gray!6}{1.0}\\
5372 & 350000 & 2 & 1.00 & 1620 & 1.0\\
\cellcolor{gray!6}{20600} & \cellcolor{gray!6}{376000} & \cellcolor{gray!6}{3} & \cellcolor{gray!6}{2.00} & \cellcolor{gray!6}{1340} & \cellcolor{gray!6}{3.0}\\
12819 & 1313000 & 6 & 3.00 & 2980 & 1.5\\
\cellcolor{gray!6}{14746} & \cellcolor{gray!6}{355000} & \cellcolor{gray!6}{4} & \cellcolor{gray!6}{2.25} & \cellcolor{gray!6}{2200} & \cellcolor{gray!6}{2.0}\\
\addlinespace
9490 & 238000 & 5 & 2.25 & 2240 & 2.0\\
\cellcolor{gray!6}{2560} & \cellcolor{gray!6}{890000} & \cellcolor{gray!6}{4} & \cellcolor{gray!6}{2.75} & \cellcolor{gray!6}{2610} & \cellcolor{gray!6}{1.0}\\
14990 & 840000 & 4 & 3.25 & 3160 & 2.0\\
\cellcolor{gray!6}{642} & \cellcolor{gray!6}{305000} & \cellcolor{gray!6}{4} & \cellcolor{gray!6}{2.50} & \cellcolor{gray!6}{2250} & \cellcolor{gray!6}{1.0}\\
8372 & 435000 & 3 & 2.25 & 1890 & 1.0\\
\bottomrule
\end{tabular}
\end{table}

\begin{table}[!h]

\caption{\label{tab:unnamed-chunk-25}Comparaison des performances des régressions linéaires de l'analyse 2}
\centering
\resizebox{\linewidth}{!}{
\begin{tabular}[t]{llrrrrrrr}
\toprule
Name & Model & AIC & BIC & R2 & R2\_adjusted & RMSE & Sigma & Performance\_Score\\
\midrule
\cellcolor{gray!6}{model3} & \cellcolor{gray!6}{lm} & \cellcolor{gray!6}{19887.24} & \cellcolor{gray!6}{19935.13} & \cellcolor{gray!6}{0.471} & \cellcolor{gray!6}{0.470} & \cellcolor{gray!6}{0.383} & \cellcolor{gray!6}{0.383} & \cellcolor{gray!6}{0.947}\\
model1 & lm & 25593.43 & 25633.33 & 0.311 & 0.310 & 0.437 & 0.437 & 0.711\\
\cellcolor{gray!6}{model4} & \cellcolor{gray!6}{lm} & \cellcolor{gray!6}{599938.62} & \cellcolor{gray!6}{599986.51} & \cellcolor{gray!6}{0.507} & \cellcolor{gray!6}{0.507} & \cellcolor{gray!6}{257789.524} & \cellcolor{gray!6}{257819.348} & \cellcolor{gray!6}{0.396}\\
model2 & lm & 608188.26 & 608228.17 & 0.278 & 0.278 & 312011.094 & 312039.970 & 0.000\\
\bottomrule
\end{tabular}}
\end{table}

Nous allons analyser le modèle 1.

\[\mathcal{M}_{1}:\log(y_{i})=\alpha+\beta_{4}x_{i,4}+\beta_{5}x_{i,5}+\beta_{8}x_{i,8}+z_{i} \iff \mathcal{M}_{1}:\log(y_{i})=\alpha+0,049x_{i,4}+0,327x_{i,5}+0,055x_{i,8}+z_{i}\]

Pour \(\alpha\), nous avons moins de 0,1\% de chance de se tromper en
rejetant l'hypothèse nulle. Par conséquent, nous rejetons l'hypothèse
nulle et nous en déduisons que la constante du modèle 1 est différente
de 0 au seuil de 0,1\%.

Pour \(\beta_{4}\), nous avons moins de 0,1\% de chance de se tromper en
rejetant l'hypothèse nulle. Par conséquent, nous rejetons l'hypothèse
nulle et nous en déduisons que le coefficient du modèle 1 de la variable
explicative \texttt{bedrooms} est différent de 0 au seuil de 0,1\%.

Pour \(\beta_{5}\), nous avons moins de 0,1\% de chance de se tromper en
rejetant l'hypothèse nulle. Par conséquent, nous rejetons l'hypothèse
nulle et nous en déduisons que le coefficient du modèle 1 de la variable
explicative \texttt{bathrooms} est différent de 0 au seuil de 0,1\%.

Pour \(\beta_{8}\), nous avons moins de 0,1\% de chance de se tromper en
rejetant l'hypothèse nulle. Par conséquent, nous rejetons l'hypothèse
nulle et nous en déduisons que le coefficient du modèle 1 de la variable
explicative \texttt{floors} est différent de 0 au seuil de 0,1\%.

Nous avons \(R^{2}=0,311\) dans le modèle 1 donc 31,1\% de la variation
du prix des logements est expliquée par le nombre de chambres, le nombre
de salles de bain et le nombre d'étages de ces derniers.

Pour \(F\), nous avons moins de 0,1\% de chance de se tromper en
rejetant l'hypothèse d'absence globale de significativité. Par
conséquent, nous rejetons l'hypothèse d'absence globale de
significativité et nous en déduisons que le modèle 1 est globalement
avéré au seuil de 0,1\%.

\newpage

\hypertarget{analyse-des-ruxe9sidus-2}{%
\subsection{Analyse des résidus}\label{analyse-des-ruxe9sidus-2}}

\begin{center}\includegraphics{E2-L3S2--1-_files/figure-latex/data.figure8-1} \end{center}

\begin{figure}
\centering
\includegraphics[width=1\textwidth,height=\textheight]{plot (1).png}
\caption{Graphiques de l'analyse des résidus}
\end{figure}

\newpage

\begin{table}[!h]

\caption{\label{tab:unnamed-chunk-26}Tests de l'analyse des résidus :}
\centering
\begin{tabular}[t]{lrrr}
\toprule
  & Harrison\_McCabe & Breusch\_Pagan & Durbin\_Watson\\
\midrule
\cellcolor{gray!6}{p-value} & \cellcolor{gray!6}{0.860} & \cellcolor{gray!6}{0.000} & \cellcolor{gray!6}{0.001}\\
statistic & 0.505 & 147.128 & 1.958\\
\bottomrule
\end{tabular}
\end{table}

Pour \(DW\), nous avons moins de 0,1\% de chance de se tomper en
rejetant l'hypothèse d'absence d'autocorrélation. Par conséquent, nous
rejetons l'hypothèse d'absence d'autocorrélation et nous en déduisons
que les résidus sont autocorrélés au seuil de 0,1\%.

Nous observons que les résidus sont globalement répartis le long de la
droite. Nous en déduisons que les résidus suivent une loi normale.

Pour \(BP\), nous avons moins de 0,1\% de chance de se tomper en
rejetant l'hypothèse d'homoscédasticité. Par conséquent, nous rejetons
l'hypothèse d'homoscédasticité et nous en déduisons que les résidus sont
hétéroscédastiques au seuil de 0,1\%.

Nous observons l'existence de résidus qui ne sont globalement pas
linéaires et qui tendent vers la distance de Cook. Nous en déduisons
que, dans une moindre mesure, les logements 15871, 8639, 8547, 2445 et
876 ont un résidu aberrant.

\begin{Shaded}
\begin{Highlighting}[]
\NormalTok{data.analyse2 <-}\StringTok{ }\NormalTok{data[ }\OperatorTok{-}\StringTok{ }\KeywordTok{c}\NormalTok{( }\DecValTok{15871}\NormalTok{,}
                            \DecValTok{8639}\NormalTok{,}
                            \DecValTok{8547}\NormalTok{,}
                            \DecValTok{2445}\NormalTok{,}
                            \DecValTok{876}\NormalTok{ ), ]}
\end{Highlighting}
\end{Shaded}

Les hypothèses nécessaires des résultats sont avérés au seuil de 0,1\%.
Néanmoins les hypothèses nécessaires des résidus ne sont pas avérés au
seuil de 0,1\%. L'hypothèse d'hétéroscédasticité ne pose pas de problème
car le prix des logements et la variance du prix des logements ne sont
pas homogènes mais l'hypothèse d'autocorrélation pose un problème car
les résidus ne sont pas linéaires et aléatoires.

\newpage

\hypertarget{mathcala_2h-lanalyse-de-lautocorruxe9lation}{%
\subsection{\texorpdfstring{\(\mathcal{A}_{2}^{H}\) : l'analyse de
l'autocorrélation}{\textbackslash mathcal\{A\}\_\{2\}\^{}\{H\} : l'analyse de l'autocorrélation}}\label{mathcala_2h-lanalyse-de-lautocorruxe9lation}}

Nous allons essayer de conserver l'hypothèse d'autocorrélation des
résidus en segmentant le prix des logements. Nous allons créer de
nouvelles bases de données en fonction du minimum, du maximum et de la
médiane :

\begin{Shaded}
\begin{Highlighting}[]
\NormalTok{data.analyse2.min_q1 <-}\StringTok{ }\NormalTok{data.analyse2[ data.analyse2}\OperatorTok{$}\NormalTok{price }\OperatorTok{>=}\StringTok{ }\DecValTok{75000} \OperatorTok{&}
\StringTok{                                         }\NormalTok{data.analyse2}\OperatorTok{$}\NormalTok{price }\OperatorTok{<=}\StringTok{ }\DecValTok{321950}\NormalTok{, ]}\CommentTok{# modèle 1,}

\NormalTok{data.analyse2.min_q2 <-}\StringTok{ }\NormalTok{data.analyse2[ data.analyse2}\OperatorTok{$}\NormalTok{price }\OperatorTok{>=}\StringTok{ }\DecValTok{75000} \OperatorTok{&}
\StringTok{                                         }\NormalTok{data.analyse2}\OperatorTok{$}\NormalTok{price }\OperatorTok{<=}\StringTok{ }\DecValTok{450000}\NormalTok{, ]}\CommentTok{# modèle 2,}

\NormalTok{data.analyse2.min_q3 <-}\StringTok{ }\NormalTok{data.analyse2[ data.analyse2}\OperatorTok{$}\NormalTok{price }\OperatorTok{>=}\StringTok{ }\DecValTok{75000} \OperatorTok{&}
\StringTok{                                         }\NormalTok{data.analyse2}\OperatorTok{$}\NormalTok{price }\OperatorTok{<=}\StringTok{ }\DecValTok{645000}\NormalTok{, ]}\CommentTok{# modèle 3,}

\NormalTok{data.analyse2.q1_q2 <-}\StringTok{ }\NormalTok{data.analyse2[ data.analyse2}\OperatorTok{$}\NormalTok{price }\OperatorTok{>=}\StringTok{ }\DecValTok{321950} \OperatorTok{&}
\StringTok{                                        }\NormalTok{data.analyse2}\OperatorTok{$}\NormalTok{price }\OperatorTok{<=}\StringTok{ }\DecValTok{450000}\NormalTok{, ]}\CommentTok{# modèle 4,}

\NormalTok{data.analyse2.q2_q3 <-}\StringTok{ }\NormalTok{data.analyse2[ data.analyse2}\OperatorTok{$}\NormalTok{price }\OperatorTok{>=}\StringTok{ }\DecValTok{450000} \OperatorTok{&}
\StringTok{                                        }\NormalTok{data.analyse2}\OperatorTok{$}\NormalTok{price }\OperatorTok{<=}\StringTok{ }\DecValTok{645000}\NormalTok{, ]}\CommentTok{# modèle 5,}

\NormalTok{data.analyse2.q1_q3 <-}\StringTok{ }\NormalTok{data.analyse2[ data.analyse2}\OperatorTok{$}\NormalTok{price }\OperatorTok{>=}\StringTok{ }\DecValTok{321950} \OperatorTok{&}
\StringTok{                                        }\NormalTok{data.analyse2}\OperatorTok{$}\NormalTok{price }\OperatorTok{<=}\StringTok{ }\DecValTok{645000}\NormalTok{, ]}\CommentTok{# modèle 6,}
\end{Highlighting}
\end{Shaded}

\newpage

\[\mathcal{A}_{1}^{A}=\{\mathcal{M}_{1}^{A},\mathcal{M}_{2}^{A},\mathcal{M}_{3}^{A},\mathcal{M}_{4}^{A},\mathcal{M}_{5}^{A},\mathcal{M}_{6}^{A}\}\]

\begin{table}[!htbp] \centering 
  \caption{Régressions linéaires supplémentaire de l'analyse 2 :} 
  \label{} 
\small 
\begin{tabular}{@{\extracolsep{1pt}}lccc} 
\\[-1.8ex]\hline 
\hline \\[-1.8ex] 
 & \multicolumn{3}{c}{\textit{Dependent variable:}} \\ 
\cline{2-4} 
\\[-1.8ex] & \multicolumn{3}{c}{log(price)} \\ 
\\[-1.8ex] & (1) & (2) & (3)\\ 
\hline \\[-1.8ex] 
 bedrooms & 0.023$^{***}$ & 0.015$^{***}$ & 0.025$^{***}$ \\ 
  & (0.004) & (0.003) & (0.003) \\ 
  bathrooms & 0.106$^{***}$ & 0.116$^{***}$ & 0.161$^{***}$ \\ 
  & (0.006) & (0.005) & (0.005) \\ 
  floors & 0.044$^{***}$ & 0.046$^{***}$ & 0.048$^{***}$ \\ 
  & (0.007) & (0.006) & (0.006) \\ 
  Constant & 12.110$^{***}$ & 12.318$^{***}$ & 12.362$^{***}$ \\ 
  & (0.012) & (0.011) & (0.011) \\ 
 \hline \\[-1.8ex] 
Observations & 5,404 & 10,863 & 16,238 \\ 
R$^{2}$ & 0.156 & 0.115 & 0.142 \\ 
Adjusted R$^{2}$ & 0.156 & 0.115 & 0.142 \\ 
Residual Std. Error & 0.202 (df = 5400) & 0.264 (df = 10859) & 0.327 (df = 16234) \\ 
F Statistic & 333.772$^{***}$ (df = 3; 5400) & 472.413$^{***}$ (df = 3; 10859) & 893.796$^{***}$ (df = 3; 16234) \\ 
\hline 
\hline \\[-1.8ex] 
\textit{Note:}  & \multicolumn{3}{r}{$^{*}$p$<$0.1; $^{**}$p$<$0.05; $^{***}$p$<$0.01} \\ 
\end{tabular} 
\end{table}

\begin{table}[!htbp] \centering 
  \caption{Régressions linéaires supplémentaires de l'analyse 2 :} 
  \label{} 
\small 
\begin{tabular}{@{\extracolsep{1pt}}lccc} 
\\[-1.8ex]\hline 
\hline \\[-1.8ex] 
 & \multicolumn{3}{c}{\textit{Dependent variable:}} \\ 
\cline{2-4} 
\\[-1.8ex] & \multicolumn{3}{c}{log(price)} \\ 
\\[-1.8ex] & (1) & (2) & (3)\\ 
\hline \\[-1.8ex] 
 bedrooms & $-$0.004$^{**}$ & 0.009$^{***}$ & 0.008$^{***}$ \\ 
  & (0.002) & (0.002) & (0.002) \\ 
  bathrooms & 0.011$^{***}$ & 0.011$^{***}$ & 0.049$^{***}$ \\ 
  & (0.003) & (0.003) & (0.004) \\ 
  floors & $-$0.006$^{**}$ & 0.008$^{***}$ & $-$0.001 \\ 
  & (0.003) & (0.003) & (0.004) \\ 
  Constant & 12.857$^{***}$ & 13.122$^{***}$ & 12.897$^{***}$ \\ 
  & (0.006) & (0.007) & (0.009) \\ 
 \hline \\[-1.8ex] 
Observations & 5,460 & 5,546 & 10,835 \\ 
R$^{2}$ & 0.003 & 0.019 & 0.033 \\ 
Adjusted R$^{2}$ & 0.002 & 0.019 & 0.033 \\ 
Residual Std. Error & 0.100 (df = 5456) & 0.103 (df = 5542) & 0.192 (df = 10831) \\ 
F Statistic & 5.536$^{***}$ (df = 3; 5456) & 36.286$^{***}$ (df = 3; 5542) & 122.607$^{***}$ (df = 3; 10831) \\ 
\hline 
\hline \\[-1.8ex] 
\textit{Note:}  & \multicolumn{3}{r}{$^{*}$p$<$0.1; $^{**}$p$<$0.05; $^{***}$p$<$0.01} \\ 
\end{tabular} 
\end{table}

\begin{table}[!h]

\caption{\label{tab:unnamed-chunk-30}Comparaison des performances des régressions linéaires supplémentaires de l'analyse 2}
\centering
\resizebox{\linewidth}{!}{
\begin{tabular}[t]{llrrrrrrr}
\toprule
Name & Model & AIC & BIC & R2 & R2\_adjusted & RMSE & Sigma & Performance\_Score\\
\midrule
\cellcolor{gray!6}{model1} & \cellcolor{gray!6}{lm} & \cellcolor{gray!6}{-1928.296} & \cellcolor{gray!6}{-1895.322} & \cellcolor{gray!6}{0.156} & \cellcolor{gray!6}{0.156} & \cellcolor{gray!6}{0.202} & \cellcolor{gray!6}{0.202} & \cellcolor{gray!6}{0.717}\\
model5 & lm & -9510.411 & -9477.307 & 0.019 & 0.019 & 0.103 & 0.103 & 0.695\\
\cellcolor{gray!6}{model4} & \cellcolor{gray!6}{lm} & \cellcolor{gray!6}{-9656.521} & \cellcolor{gray!6}{-9623.495} & \cellcolor{gray!6}{0.003} & \cellcolor{gray!6}{0.002} & \cellcolor{gray!6}{0.100} & \cellcolor{gray!6}{0.100} & \cellcolor{gray!6}{0.667}\\
model6 & lm & -4958.215 & -4921.762 & 0.033 & 0.033 & 0.192 & 0.192 & 0.515\\
\cellcolor{gray!6}{model2} & \cellcolor{gray!6}{lm} & \cellcolor{gray!6}{1859.896} & \cellcolor{gray!6}{1896.362} & \cellcolor{gray!6}{0.115} & \cellcolor{gray!6}{0.115} & \cellcolor{gray!6}{0.263} & \cellcolor{gray!6}{0.264} & \cellcolor{gray!6}{0.474}\\
\addlinespace
model3 & lm & 9783.224 & 9821.699 & 0.142 & 0.142 & 0.327 & 0.327 & 0.302\\
\bottomrule
\end{tabular}}
\end{table}

\newpage

\begin{table}[!h]

\caption{\label{tab:unnamed-chunk-31}Tests de l'analyse des résidus du modèle 1 :}
\centering
\begin{tabular}[t]{lrrr}
\toprule
  & Harrison\_McCabe & Breusch\_Pagan & Durbin\_Watson\\
\midrule
\cellcolor{gray!6}{p-value} & \cellcolor{gray!6}{0.096} & \cellcolor{gray!6}{0.000} & \cellcolor{gray!6}{0.410}\\
statistic & 0.487 & 407.933 & 1.994\\
\bottomrule
\end{tabular}
\end{table}

\begin{table}[!h]

\caption{\label{tab:unnamed-chunk-31}Tests de l'analyse des résidus du modèle 2 :}
\centering
\begin{tabular}[t]{lrrr}
\toprule
  & Harrison\_McCabe & Breusch\_Pagan & Durbin\_Watson\\
\midrule
\cellcolor{gray!6}{p-value} & \cellcolor{gray!6}{0.917} & \cellcolor{gray!6}{0.000} & \cellcolor{gray!6}{0.418}\\
statistic & 0.510 & 849.865 & 1.996\\
\bottomrule
\end{tabular}
\end{table}

\begin{table}[!h]

\caption{\label{tab:unnamed-chunk-31}Tests de l'analyse des résidus du modèle 3 :}
\centering
\begin{tabular}[t]{lrrr}
\toprule
  & Harrison\_McCabe & Breusch\_Pagan & Durbin\_Watson\\
\midrule
\cellcolor{gray!6}{p-value} & \cellcolor{gray!6}{0.980} & \cellcolor{gray!6}{0.000} & \cellcolor{gray!6}{0.183}\\
statistic & 0.512 & 802.932 & 1.986\\
\bottomrule
\end{tabular}
\end{table}

\begin{table}[!h]

\caption{\label{tab:unnamed-chunk-31}Tests de l'analyse des résidus du modèle 4 :}
\centering
\begin{tabular}[t]{lrrr}
\toprule
  & Harrison\_McCabe & Breusch\_Pagan & Durbin\_Watson\\
\midrule
\cellcolor{gray!6}{p-value} & \cellcolor{gray!6}{0.706} & \cellcolor{gray!6}{0.553} & \cellcolor{gray!6}{0.336}\\
statistic & 0.505 & 2.093 & 1.989\\
\bottomrule
\end{tabular}
\end{table}

\begin{table}[!h]

\caption{\label{tab:unnamed-chunk-31}Tests de l'analyse des résidus du modèle 5 :}
\centering
\begin{tabular}[t]{lrrr}
\toprule
  & Harrison\_McCabe & Breusch\_Pagan & Durbin\_Watson\\
\midrule
\cellcolor{gray!6}{p-value} & \cellcolor{gray!6}{0.474} & \cellcolor{gray!6}{0.062} & \cellcolor{gray!6}{0.872}\\
statistic & 0.499 & 7.343 & 2.031\\
\bottomrule
\end{tabular}
\end{table}

\begin{table}[!h]

\caption{\label{tab:unnamed-chunk-31}Tests de l'analyse des résidus du modèle 6 :}
\centering
\begin{tabular}[t]{lrrr}
\toprule
  & Harrison\_McCabe & Breusch\_Pagan & Durbin\_Watson\\
\midrule
\cellcolor{gray!6}{p-value} & \cellcolor{gray!6}{0.385} & \cellcolor{gray!6}{0.000} & \cellcolor{gray!6}{0.480}\\
statistic & 0.498 & 51.599 & 1.999\\
\bottomrule
\end{tabular}
\end{table}

\newpage

\begin{figure}
\centering
\includegraphics[width=1\textwidth,height=\textheight]{plot (11).png}
\caption{Graphiques de l'analyse des résidus}
\end{figure}

\[\mathcal{M}_{1}^{A}:\log(y_{i})=\alpha+\beta_{4}x_{i,4}+\beta_{5}x_{i,5}+\beta_{8}x_{i,8}+z_{i} \iff \mathcal{M}_{1}^{A}:\log(y_{i})=12,11+0,023x_{i,4}+0,106x_{i,5}+0,044x_{i,8}+z_{i}\]

En conclusion, nous conservons les hypothèses d'absence
d'autocorrélation et nous en déduisons que les résidus ne sont pas
autocorrélés au seuil de 0,1\% dans les modèles 1, 2, 3, 4, 5 et 6. De
manière biaisé par les graphiques de l'analyse des résidus qui ne sont
pas bons, nous avons 0,1\% de chance de se tromper en affirmant que plus
le nombre de chambres, le nombre de salles de bain et le nombre d'étages
sont élevés et plus le prix des logements est élevé selon la relation
\(\mathcal{M}_{1}^{A}\).

\newpage

\hypertarget{mathcala_3-lanalyse-du-prix-dun-logement-uxe0-partir-de-645-000-dollars-en-fonction-de-sa-vue-sur-la-mer-ou-non-de-son-point-de-vue-et-de-son}{%
\section{\texorpdfstring{\(\mathcal{A}_{3}\) : l'analyse du prix d'un
logement à partir de 645 000 dollars en fonction de sa vue sur la mer ou
non, de son point de vue et de son
\textit{design}}{\textbackslash mathcal\{A\}\_\{3\} : l'analyse du prix d'un logement à partir de 645 000 dollars en fonction de sa vue sur la mer ou non, de son point de vue et de son }}\label{mathcala_3-lanalyse-du-prix-dun-logement-uxe0-partir-de-645-000-dollars-en-fonction-de-sa-vue-sur-la-mer-ou-non-de-son-point-de-vue-et-de-son}}

Nous allons essayer d'analyser le \textit{standing} dont nous avons
parlé dans le \textbf{4.4}. Par conséquent, nous allons segmenter la
base de données \texttt{BDD\_data.csv} en ne prenant en compte que les
logements coûtant plus de 645 000 dollars.

\begin{Shaded}
\begin{Highlighting}[]
\NormalTok{data.analyse3 <-}\StringTok{ }\NormalTok{data[ data}\OperatorTok{$}\NormalTok{price }\OperatorTok{>=}\StringTok{ }\DecValTok{645000}\NormalTok{, ]}
\end{Highlighting}
\end{Shaded}

\[\mathcal{A}_{3}=\{\mathcal{M}_{1},\mathcal{M}_{2},\mathcal{M}_{3},\mathcal{M}_{4}\}\]

\begin{table}[!htbp] \centering 
  \caption{Régressions linéaires de l'analyse 3 :} 
  \label{} 
\small 
\begin{tabular}{@{\extracolsep{1pt}}lcc} 
\\[-1.8ex]\hline 
\hline \\[-1.8ex] 
 & \multicolumn{2}{c}{\textit{Dependent variable:}} \\ 
\cline{2-3} 
\\[-1.8ex] & log(price) & price \\ 
\\[-1.8ex] & (1) & (2)\\ 
\hline \\[-1.8ex] 
 waterfront2 & 0.332$^{***}$ & 600,302.300$^{***}$ \\ 
  & (0.027) & (37,981.760) \\ 
  view2 & 0.130$^{***}$ & 166,096.600$^{***}$ \\ 
  & (0.019) & (26,035.900) \\ 
  view3 & 0.096$^{***}$ & 104,150.500$^{***}$ \\ 
  & (0.012) & (16,487.350) \\ 
  view4 & 0.172$^{***}$ & 190,977.600$^{***}$ \\ 
  & (0.014) & (20,029.270) \\ 
  view5 & 0.238$^{***}$ & 288,291.500$^{***}$ \\ 
  & (0.019) & (27,054.850) \\ 
  grade5 & 0.182 & 230,233.800 \\ 
  & (0.134) & (186,647.300) \\ 
  grade6 & 0.220$^{*}$ & 306,892.500$^{*}$ \\ 
  & (0.127) & (176,938.000) \\ 
  grade7 & 0.278$^{**}$ & 356,616.400$^{**}$ \\ 
  & (0.127) & (176,638.200) \\ 
  grade8 & 0.382$^{***}$ & 464,067.200$^{***}$ \\ 
  & (0.127) & (176,627.800) \\ 
  grade9 & 0.540$^{***}$ & 651,744.500$^{***}$ \\ 
  & (0.127) & (176,730.900) \\ 
  grade10 & 0.789$^{***}$ & 996,335.500$^{***}$ \\ 
  & (0.127) & (177,236.000) \\ 
  grade11 & 1.094$^{***}$ & 1,591,935.000$^{***}$ \\ 
  & (0.129) & (180,065.100) \\ 
  grade12 & 1.644$^{***}$ & 3,174,180.000$^{***}$ \\ 
  & (0.145) & (201,539.200) \\ 
  Constant & 13.280$^{***}$ & 415,491.000$^{**}$ \\ 
  & (0.127) & (176,509.000) \\ 
 \hline \\[-1.8ex] 
Observations & 5,413 & 5,413 \\ 
R$^{2}$ & 0.444 & 0.454 \\ 
Adjusted R$^{2}$ & 0.443 & 0.452 \\ 
Residual Std. Error (df = 5399) & 0.253 & 352,169.000 \\ 
F Statistic (df = 13; 5399) & 331.901$^{***}$ & 344.646$^{***}$ \\ 
\hline 
\hline \\[-1.8ex] 
\textit{Note:}  & \multicolumn{2}{r}{$^{*}$p$<$0.1; $^{**}$p$<$0.05; $^{***}$p$<$0.01} \\ 
\end{tabular} 
\end{table}

\begin{table}[!htbp] \centering 
  \caption{Régressions linéaires de l'analyse 3 :} 
  \label{} 
\small 
\begin{tabular}{@{\extracolsep{1pt}}lcc} 
\\[-1.8ex]\hline 
\hline \\[-1.8ex] 
 & \multicolumn{2}{c}{\textit{Dependent variable:}} \\ 
\cline{2-3} 
\\[-1.8ex] & log(price) & price \\ 
\\[-1.8ex] & (1) & (2)\\ 
\hline \\[-1.8ex] 
 log(sqft\_living) & 0.310$^{***}$ &  \\ 
  & (0.014) &  \\ 
  sqft\_living &  & 177.778$^{***}$ \\ 
  &  & (6.344) \\ 
  waterfront2 & 0.329$^{***}$ & 589,476.800$^{***}$ \\ 
  & (0.026) & (35,493.090) \\ 
  view2 & 0.119$^{***}$ & 145,610.000$^{***}$ \\ 
  & (0.018) & (24,339.490) \\ 
  view3 & 0.084$^{***}$ & 82,137.050$^{***}$ \\ 
  & (0.011) & (15,426.160) \\ 
  view4 & 0.155$^{***}$ & 157,060.900$^{***}$ \\ 
  & (0.014) & (18,754.880) \\ 
  view5 & 0.217$^{***}$ & 244,800.600$^{***}$ \\ 
  & (0.019) & (25,328.230) \\ 
  grade5 & 0.106 & 158,734.500 \\ 
  & (0.129) & (174,426.000) \\ 
  grade6 & 0.032 & 114,381.100 \\ 
  & (0.122) & (165,477.400) \\ 
  grade7 & 0.030 & 92,888.610 \\ 
  & (0.122) & (165,322.700) \\ 
  grade8 & 0.073 & 106,255.900 \\ 
  & (0.122) & (165,538.000) \\ 
  grade9 & 0.180 & 199,861.400 \\ 
  & (0.123) & (165,926.500) \\ 
  grade10 & 0.365$^{***}$ & 398,163.300$^{**}$ \\ 
  & (0.123) & (166,982.900) \\ 
  grade11 & 0.609$^{***}$ & 810,832.700$^{***}$ \\ 
  & (0.126) & (170,549.600) \\ 
  grade12 & 1.064$^{***}$ & 2,036,456.000$^{***}$ \\ 
  & (0.141) & (192,648.700) \\ 
  Constant & 11.114$^{***}$ & 243,224.000 \\ 
  & (0.156) & (165,048.300) \\ 
 \hline \\[-1.8ex] 
Observations & 5,413 & 5,413 \\ 
R$^{2}$ & 0.490 & 0.523 \\ 
Adjusted R$^{2}$ & 0.489 & 0.522 \\ 
Residual Std. Error (df = 5398) & 0.242 & 329,074.400 \\ 
F Statistic (df = 14; 5398) & 370.816$^{***}$ & 422.624$^{***}$ \\ 
\hline 
\hline \\[-1.8ex] 
\textit{Note:}  & \multicolumn{2}{r}{$^{*}$p$<$0.1; $^{**}$p$<$0.05; $^{***}$p$<$0.01} \\ 
\end{tabular} 
\end{table}

\newpage

\hypertarget{analyse-des-ruxe9sultats-3}{%
\subsection{Analyse des résultats}\label{analyse-des-ruxe9sultats-3}}

\begin{table}[!h]

\caption{\label{tab:unnamed-chunk-34}Comparaison des performances des régressions linéaires de l'analyse 3}
\centering
\resizebox{\linewidth}{!}{
\begin{tabular}[t]{llrrrrrrr}
\toprule
Name & Model & AIC & BIC & R2 & R2\_adjusted & RMSE & Sigma & Performance\_Score\\
\midrule
\cellcolor{gray!6}{model3} & \cellcolor{gray!6}{lm} & \cellcolor{gray!6}{36.558} & \cellcolor{gray!6}{142.103} & \cellcolor{gray!6}{0.490} & \cellcolor{gray!6}{0.489} & \cellcolor{gray!6}{0.242} & \cellcolor{gray!6}{0.242} & \cellcolor{gray!6}{0.862}\\
model1 & lm & 502.793 & 601.741 & 0.444 & 0.443 & 0.253 & 0.253 & 0.666\\
\cellcolor{gray!6}{model4} & \cellcolor{gray!6}{lm} & \cellcolor{gray!6}{152912.336} & \cellcolor{gray!6}{153017.881} & \cellcolor{gray!6}{0.523} & \cellcolor{gray!6}{0.522} & \cellcolor{gray!6}{328618.144} & \cellcolor{gray!6}{329074.410} & \cellcolor{gray!6}{0.357}\\
model2 & lm & 153645.636 & 153744.584 & 0.454 & 0.452 & 351713.255 & 352168.968 & 0.039\\
\bottomrule
\end{tabular}}
\end{table}

\begin{quote}
\[\mathcal{M}_{1}:\log(y_{i})=\alpha+\beta_{9}x_{i,9}+\beta_{10}x_{i,10}+\beta_{12}x_{i,12}+z_{i}\]
\end{quote}

Nous allons analyser le modèle 1.

Pour \(\alpha\), nous avons moins de 0,1\% de chance de se tromper en
rejetant l'hypothèse nulle. Par conséquent, nous rejetons l'hypothèse
nulle et nous en déduisons que la constante du modèle 1 est différente
de 0 au seuil de 0,1\%.

Pour \(\beta_{9}^{1}\), nous avons moins de 0,1\% de chance de se
tromper en rejetant l'hypothèse nulle. Par conséquent, nous rejetons
l'hypothèse nulle et nous en déduisons que le coefficient du modèle 1 de
la variable explicative \texttt{waterfront1} est différent de 0 au seuil
de 0,1\%.

Pour \(\beta_{10}^{2,3,4,5}\), nous avons moins de 0,1\% de chance de se
tromper en rejetant l'hypothèse nulle. Par conséquent, nous rejetons
l'hypothèse nulle et nous en déduisons que les coefficients du modèle 1
des variables explicatives \texttt{view2}, \texttt{view3},
\texttt{view4} et \texttt{view5} sont différents de 0 au seuil de 0,1\%.

Pour \(\beta_{12}^{6,7,8,9}\), nous avons entre 0,3\% et 17,6\% de
chance de se tromper en rejetant l'hypothèse nulle. Par conséquent, nous
conservons l'hypothèse nulle et nous en déduisons que les coefficients
du modèle 1 des variables explicatives \texttt{grade6}, \texttt{grade7},
\texttt{grade8} et \texttt{grade9} sont égales à 0 au seuil de 0,1\%.

Pour \(\beta_{12}^{10,11,12,13}\), nous avons moins de 0,1\% de chance
de se tromper en rejetant l'hypothèse nulle. Par conséquent, nous
rejetons l'hypothèse nulle et nous en déduisons que les coefficients du
modèle 1 des variables explicatives \texttt{grade10}, \texttt{grade11},
\texttt{grade12} et \texttt{grade13} sont différents de 0 au seuil de
0,1\%.

Nous avons \(R^{2}=0.444\) dans le modèle 1 donc 44,4\% de la variation
du prix des logements est expliquée par le point de vue, la vue sur la
mer et le \textit{design} de ces derniers.

Pour \(F\), nous avons moins de 0,1\% de chance de se tromper en
rejetant l'hypothèse d'absence globale de significativité. Par
conséquent, nous rejetons l'hypothèse d'absence globale de
significativité et nous en déduisons que le modèle 1 est globalement
avéré au seuil de 0,1\%.

\newpage

\hypertarget{analyse-des-ruxe9sidus-3}{%
\subsection{Analyse des résidus}\label{analyse-des-ruxe9sidus-3}}

\begin{center}\includegraphics{E2-L3S2--1-_files/figure-latex/data.figure9-1} \end{center}

\begin{figure}
\centering
\includegraphics[width=1\textwidth,height=\textheight]{plot (5).png}
\caption{Graphiques de l'analyse des résidus}
\end{figure}

\newpage

\begin{table}[!h]

\caption{\label{tab:unnamed-chunk-35}Tests de l'analyse des résidus :}
\centering
\begin{tabular}[t]{lrrr}
\toprule
  & Harrison\_McCabe & Breusch\_Pagan & Durbin\_Watson\\
\midrule
\cellcolor{gray!6}{p-value} & \cellcolor{gray!6}{0.079} & \cellcolor{gray!6}{0.000} & \cellcolor{gray!6}{0.084}\\
statistic & 0.486 & 462.261 & 1.963\\
\bottomrule
\end{tabular}
\end{table}

Pour \(DW\), nous avons 8,37\% de chance de se tomper en rejetant
l'hypothèse d'absence d'autocorrélation. Par conséquent, nous conservons
l'hypothèse d'absence d'autocorrélation et nous en déduisons que les
résidus ne sont pas autocorrélés au seuil de 0,1\%.

Nous observons que les résidus ne sont globalement pas répartis le long
de la droite. Néanmoins, nous en déduisons que les résidus suivent une
loi normale.

Pour \(BP\), nous avons moins de 0,1\% de chance de se tomper en
rejetant l'hypothèse d'homoscédasticité. Par conséquent, nous rejetons
l'hypothèse d'homoscédasticité et nous en déduisons que les résidus sont
hétéroscédastiques au seuil de 0,1\%.

Nous observons l'existence de résidus qui ne sont globalement pas
linéaires et qui tendent vers la distance de Cook. Nous en déduisons
que, dans une moindre mesure, les logements 21051, 12778, 10447, 8639,
7253 et 3915 ont un résidu aberrant.

Les hypothèses nécessaires des résultats et des résidus sont avérés au
seuil de 0,1\%. Notre constante, notre coefficient, notre
significativité, notre linéarité des résidus, notre absence
d'autocorrélation des résidus et notre appartenance à une loi normale
des résidus sont avérés au seuil de 0,1\%. L'hypothèse
d'hétéroscédasticité ne pose pas de problème car le prix des logements
et la variance du prix des logements ne sont pas homogènes
\textbf{notamment} pour les logements coûtant plus de 645 000 dollars.

En conclusion, nous avons 0,1\% de chance de se tromper en affirmant que
plus le point de vue est beau et plus le \textit{design} est classe et
plus le prix des logements les logements coûtant plus de 645 000 dollars
est élevé selon la relation \(\mathcal{M}_{1}\).

\newpage

\hypertarget{mathcala_4-lanalyse-du-prix-dun-logement-en-fonction-de-son-uxe9tat}{%
\section{\texorpdfstring{\(\mathcal{A}_{4}\) : l'analyse du prix d'un
logement en fonction de son
état}{\textbackslash mathcal\{A\}\_\{4\} : l'analyse du prix d'un logement en fonction de son état}}\label{mathcala_4-lanalyse-du-prix-dun-logement-en-fonction-de-son-uxe9tat}}

\begin{table}[!htbp] \centering 
  \caption{Régressions linéaires de l'analyse 4 :} 
  \label{} 
\small 
\begin{tabular}{@{\extracolsep{1pt}}lcc} 
\\[-1.8ex]\hline 
\hline \\[-1.8ex] 
 & \multicolumn{2}{c}{\textit{Dependent variable:}} \\ 
\cline{2-3} 
\\[-1.8ex] & log(price) & price \\ 
\\[-1.8ex] & (1) & (2)\\ 
\hline \\[-1.8ex] 
 condition2 & 0.047 & $-$7,144.521 \\ 
  & (0.103) & (72,395.060) \\ 
  condition3 & 0.565$^{***}$ & 207,580.900$^{***}$ \\ 
  & (0.096) & (66,874.610) \\ 
  condition4 & 0.521$^{***}$ & 186,768.700$^{***}$ \\ 
  & (0.096) & (66,979.450) \\ 
  condition5 & 0.667$^{***}$ & 277,986.400$^{***}$ \\ 
  & (0.096) & (67,389.750) \\ 
  Constant & 12.491$^{***}$ & 334,431.700$^{***}$ \\ 
  & (0.095) & (66,803.230) \\ 
 \hline \\[-1.8ex] 
Observations & 21,613 & 21,613 \\ 
R$^{2}$ & 0.014 & 0.007 \\ 
Adjusted R$^{2}$ & 0.014 & 0.007 \\ 
Residual Std. Error (df = 21608) & 0.523 & 365,896.400 \\ 
F Statistic (df = 4; 21608) & 75.880$^{***}$ & 37.412$^{***}$ \\ 
\hline 
\hline \\[-1.8ex] 
\textit{Note:}  & \multicolumn{2}{r}{$^{*}$p$<$0.1; $^{**}$p$<$0.05; $^{***}$p$<$0.01} \\ 
\end{tabular} 
\end{table}

\begin{table}[!htbp] \centering 
  \caption{Régressions linéaires de l'analyse 4 :} 
  \label{} 
\small 
\begin{tabular}{@{\extracolsep{1pt}}lcc} 
\\[-1.8ex]\hline 
\hline \\[-1.8ex] 
 & \multicolumn{2}{c}{\textit{Dependent variable:}} \\ 
\cline{2-3} 
\\[-1.8ex] & log(price) & price \\ 
\\[-1.8ex] & (1) & (2)\\ 
\hline \\[-1.8ex] 
 log(sqft\_living) & 0.837$^{***}$ &  \\ 
  & (0.006) &  \\ 
  sqft\_living &  & 282.642$^{***}$ \\ 
  &  & (1.938) \\ 
  condition2 & $-$0.089 & $-$61,993.580 \\ 
  & (0.076) & (51,394.280) \\ 
  condition3 & 0.073 & $-$56,136.330 \\ 
  & (0.071) & (47,508.390) \\ 
  condition4 & 0.101 & $-$20,971.100 \\ 
  & (0.071) & (47,569.710) \\ 
  condition5 & 0.220$^{***}$ & 49,919.100 \\ 
  & (0.071) & (47,865.210) \\ 
  Constant & 6.635$^{***}$ & $-$9,261.488 \\ 
  & (0.083) & (47,481.810) \\ 
 \hline \\[-1.8ex] 
Observations & 21,613 & 21,613 \\ 
R$^{2}$ & 0.462 & 0.500 \\ 
Adjusted R$^{2}$ & 0.462 & 0.499 \\ 
Residual Std. Error (df = 21607) & 0.386 & 259,748.100 \\ 
F Statistic (df = 5; 21607) & 3,711.535$^{***}$ & 4,313.423$^{***}$ \\ 
\hline 
\hline \\[-1.8ex] 
\textit{Note:}  & \multicolumn{2}{r}{$^{*}$p$<$0.1; $^{**}$p$<$0.05; $^{***}$p$<$0.01} \\ 
\end{tabular} 
\end{table}

\newpage

\hypertarget{analyse-des-ruxe9sultats-4}{%
\subsection{Analyse des résultats}\label{analyse-des-ruxe9sultats-4}}

\begin{table}[!h]

\caption{\label{tab:unnamed-chunk-37}Comparaison des performances des régressions linéaires de l'analyse 4}
\centering
\resizebox{\linewidth}{!}{
\begin{tabular}[t]{llrrrrrrr}
\toprule
Name & Model & AIC & BIC & R2 & R2\_adjusted & RMSE & Sigma & Performance\_Score\\
\midrule
\cellcolor{gray!6}{model3} & \cellcolor{gray!6}{lm} & \cellcolor{gray!6}{20234.11} & \cellcolor{gray!6}{20289.98} & \cellcolor{gray!6}{0.462} & \cellcolor{gray!6}{0.462} & \cellcolor{gray!6}{0.386} & \cellcolor{gray!6}{0.386} & \cellcolor{gray!6}{0.975}\\
model1 & lm & 33330.06 & 33377.94 & 0.014 & 0.014 & 0.523 & 0.523 & 0.664\\
\cellcolor{gray!6}{model4} & \cellcolor{gray!6}{lm} & \cellcolor{gray!6}{600261.78} & \cellcolor{gray!6}{600317.65} & \cellcolor{gray!6}{0.500} & \cellcolor{gray!6}{0.499} & \cellcolor{gray!6}{259711.998} & \cellcolor{gray!6}{259748.055} & \cellcolor{gray!6}{0.438}\\
model2 & lm & 615071.65 & 615119.54 & 0.007 & 0.007 & 365854.028 & 365896.354 & 0.000\\
\bottomrule
\end{tabular}}
\end{table}

Nous allons analyser le modèle 1.

\begin{quote}
\[\mathcal{M}_{1}:\log(y_{i})=\alpha+\beta_{11}x_{i,11}+z_{i}\]
\end{quote}

Pour \(\alpha\), nous avons moins de 0,1\% de chance de se tromper en
rejetant l'hypothèse nulle. Par conséquent, nous rejetons l'hypothèse
nulle et nous en déduisons que la constante du modèle 1 est différente
de 0 au seuil de 0,1\%.

Pour \(\beta_{11}^{2}\), nous avons 65\% de chance de se tromper en
rejetant l'hypothèse nulle. Par conséquent, nous conservons l'hypothèse
nulle et nous en déduisons que le coefficient du modèle 1 de la variable
explicative \texttt{condition2} est égale à 0 au seuil de 0,1\%.

Pour \(\beta_{11}^{3,4,5}\), nous avons moins de 0,1\% de chance de se
tromper en rejetant l'hypothèse nulle. Par conséquent, nous rejetons
l'hypothèse nulle et nous en déduisons que les coefficients du modèle 1
des variables explicatives \texttt{condition3}, \texttt{condition4} et
\texttt{condition5} sont différents de 0 au seuil de 0,1\%.

Nous avons \(R^{2}=0.014\) dans le modèle 1 donc 1.4\% de la variation
du prix des logements est expliquée par l'état de ces derniers.

Pour \(F\), nous avons moins de 0,1\% de chance de se tromper en
rejetant l'hypothèse d'absence globale de significativité. Par
conséquent, nous rejetons l'hypothèse d'absence globale de
significativité et nous en déduisons que le modèle 1 est globalement
avéré au seuil de 0,1\%.

\newpage

\hypertarget{analyse-des-ruxe9sidus-4}{%
\subsection{Analyse des résidus}\label{analyse-des-ruxe9sidus-4}}

\begin{center}\includegraphics{E2-L3S2--1-_files/figure-latex/data.figure10-1} \end{center}

\begin{figure}
\centering
\includegraphics[width=1\textwidth,height=\textheight]{plot (8).png}
\caption{Graphiques de l'analyse des résidus}
\end{figure}

\newpage

\begin{table}[!h]

\caption{\label{tab:unnamed-chunk-38}Tests de l'analyse des résidus :}
\centering
\begin{tabular}[t]{lrrr}
\toprule
  & Harrison\_McCabe & Breusch\_Pagan & Durbin\_Watson\\
\midrule
\cellcolor{gray!6}{p-value} & \cellcolor{gray!6}{0.366} & \cellcolor{gray!6}{0.000} & \cellcolor{gray!6}{0.001}\\
statistic & 0.498 & 21.666 & 1.959\\
\bottomrule
\end{tabular}
\end{table}

Pour \(DW\), nous avons 0,1\% de chance de se tomper en rejetant
l'hypothèse d'absence d'autocorrélation. Par conséquent, nous rejetons
l'hypothèse d'absence d'autocorrélation et nous en déduisons que les
résidus sont autocorrélés au seuil de 0,1\%.

Nous observons que les résidus ne sont globalement pas répartis le long
de la droite. Néanmoins, nous en déduisons que les résidus suivent une
loi normale.

Pour \(BP\), nous avons moins de 0,1\% de chance de se tomper en
rejetant l'hypothèse d'homoscédasticité. Par conséquent, nous rejetons
l'hypothèse d'homoscédasticité et nous en déduisons que les résidus sont
hétéroscédastiques au seuil de 0,1\%.

Nous observons l'existence de résidus qui ne sont globalement pas
linéaires et qui tendent vers la distance de Cook. Nous en déduisons
que, dans une moindre mesure, les logements 19294, 18877, 16199, 9255,
7253 et 3915 ont un résidu aberrant.

\begin{Shaded}
\begin{Highlighting}[]
\NormalTok{data.analyse4 <-}\StringTok{ }\NormalTok{data[ }\OperatorTok{-}\StringTok{ }\KeywordTok{c}\NormalTok{( }\DecValTok{19294}\NormalTok{,}
                            \DecValTok{18877}\NormalTok{,}
                            \DecValTok{16199}\NormalTok{,}
                            \DecValTok{9255}\NormalTok{,}
                            \DecValTok{7253}\NormalTok{,}
                            \DecValTok{3915}\NormalTok{ ), ]}
\end{Highlighting}
\end{Shaded}

Les hypothèses nécessaires des résultats sont avérés au seuil de 0,1\%.
Néanmoins les hypothèses nécessaires des résidus ne sont pas avérés au
seuil de 0,1\%. L'hypothèse d'hétéroscédasticité ne pose pas de problème
car le prix des logements et la variance du prix des logements ne sont
pas homogènes mais l'hypothèse d'autocorrélation pose un problème car
les résidus ne sont pas linéaires et aléatoires.

\newpage

\hypertarget{mathcala_4h-lanalyse-de-lautocorruxe9lation}{%
\subsection{\texorpdfstring{\(\mathcal{A}_{4}^{H}\) : l'analyse de
l'autocorrélation}{\textbackslash mathcal\{A\}\_\{4\}\^{}\{H\} : l'analyse de l'autocorrélation}}\label{mathcala_4h-lanalyse-de-lautocorruxe9lation}}

Nous allons essayer de conserver l'hypothèse d'autocorrélation des
résidus en segmentant le prix des logements. Nous allons créer de
nouvelles bases de données en fonction du minimum, du maximum et de la
médiane :

\begin{Shaded}
\begin{Highlighting}[]
\NormalTok{data.analyse4.min_q1 <-}\StringTok{ }\NormalTok{data.analyse4[ data.analyse4}\OperatorTok{$}\NormalTok{price }\OperatorTok{>=}\StringTok{ }\DecValTok{75000} \OperatorTok{&}
\StringTok{                                         }\NormalTok{data.analyse4}\OperatorTok{$}\NormalTok{price }\OperatorTok{<=}\StringTok{ }\DecValTok{321950}\NormalTok{, ]}\CommentTok{# modèle 1,}

\NormalTok{data.analyse4.min_q2 <-}\StringTok{ }\NormalTok{data.analyse4[ data.analyse4}\OperatorTok{$}\NormalTok{price }\OperatorTok{>=}\StringTok{ }\DecValTok{75000} \OperatorTok{&}
\StringTok{                                         }\NormalTok{data.analyse4}\OperatorTok{$}\NormalTok{price }\OperatorTok{<=}\StringTok{ }\DecValTok{450000}\NormalTok{, ]}\CommentTok{# modèle 2,}

\NormalTok{data.analyse4.min_q3 <-}\StringTok{ }\NormalTok{data.analyse4[ data.analyse4}\OperatorTok{$}\NormalTok{price }\OperatorTok{>=}\StringTok{ }\DecValTok{75000} \OperatorTok{&}
\StringTok{                                         }\NormalTok{data.analyse4}\OperatorTok{$}\NormalTok{price }\OperatorTok{<=}\StringTok{ }\DecValTok{645000}\NormalTok{, ]}\CommentTok{# modèle 3,}

\NormalTok{data.analyse4.q1_q2 <-}\StringTok{ }\NormalTok{data.analyse4[ data.analyse4}\OperatorTok{$}\NormalTok{price }\OperatorTok{>=}\StringTok{ }\DecValTok{321950} \OperatorTok{&}
\StringTok{                                        }\NormalTok{data.analyse4}\OperatorTok{$}\NormalTok{price }\OperatorTok{<=}\StringTok{ }\DecValTok{450000}\NormalTok{, ]}\CommentTok{# modèle 4,}

\NormalTok{data.analyse4.q2_q3 <-}\StringTok{ }\NormalTok{data.analyse4[ data.analyse4}\OperatorTok{$}\NormalTok{price }\OperatorTok{>=}\StringTok{ }\DecValTok{450000} \OperatorTok{&}
\StringTok{                                        }\NormalTok{data.analyse4}\OperatorTok{$}\NormalTok{price }\OperatorTok{<=}\StringTok{ }\DecValTok{645000}\NormalTok{, ]}\CommentTok{# modèle 5,}

\NormalTok{data.analyse4.q1_q3 <-}\StringTok{ }\NormalTok{data.analyse4[ data.analyse4}\OperatorTok{$}\NormalTok{price }\OperatorTok{>=}\StringTok{ }\DecValTok{321950} \OperatorTok{&}
\StringTok{                                        }\NormalTok{data.analyse4}\OperatorTok{$}\NormalTok{price }\OperatorTok{<=}\StringTok{ }\DecValTok{645000}\NormalTok{, ]}\CommentTok{# modèle 6,}
\end{Highlighting}
\end{Shaded}

\newpage

\[\mathcal{A}_{1}^{A}=\{\mathcal{M}_{1}^{A},\mathcal{M}_{2}^{A},\mathcal{M}_{3}^{A},\mathcal{M}_{4}^{A},\mathcal{M}_{5}^{A},\mathcal{M}_{6}^{A}\}\]

\begin{table}[!htbp] \centering 
  \caption{Régressions linéaires supplémentaires de l'analyse 4 :} 
  \label{} 
\small 
\begin{tabular}{@{\extracolsep{1pt}}lccc} 
\\[-1.8ex]\hline 
\hline \\[-1.8ex] 
 & \multicolumn{3}{c}{\textit{Dependent variable:}} \\ 
\cline{2-4} 
\\[-1.8ex] & \multicolumn{3}{c}{log(price)} \\ 
\\[-1.8ex] & (1) & (2) & (3)\\ 
\hline \\[-1.8ex] 
 condition2 & 0.119$^{**}$ & 0.023 & 0.033 \\ 
  & (0.057) & (0.061) & (0.073) \\ 
  condition3 & 0.320$^{***}$ & 0.314$^{***}$ & 0.397$^{***}$ \\ 
  & (0.053) & (0.057) & (0.067) \\ 
  condition4 & 0.305$^{***}$ & 0.274$^{***}$ & 0.360$^{***}$ \\ 
  & (0.053) & (0.057) & (0.068) \\ 
  condition5 & 0.311$^{***}$ & 0.293$^{***}$ & 0.407$^{***}$ \\ 
  & (0.054) & (0.058) & (0.068) \\ 
  Constant & 12.104$^{***}$ & 12.339$^{***}$ & 12.438$^{***}$ \\ 
  & (0.053) & (0.057) & (0.067) \\ 
 \hline \\[-1.8ex] 
Observations & 5,402 & 10,862 & 16,238 \\ 
R$^{2}$ & 0.023 & 0.019 & 0.014 \\ 
Adjusted R$^{2}$ & 0.022 & 0.019 & 0.014 \\ 
Residual Std. Error & 0.217 (df = 5397) & 0.277 (df = 10857) & 0.350 (df = 16233) \\ 
F Statistic & 31.480$^{***}$ (df = 4; 5397) & 53.388$^{***}$ (df = 4; 10857) & 59.000$^{***}$ (df = 4; 16233) \\ 
\hline 
\hline \\[-1.8ex] 
\textit{Note:}  & \multicolumn{3}{r}{$^{*}$p$<$0.1; $^{**}$p$<$0.05; $^{***}$p$<$0.01} \\ 
\end{tabular} 
\end{table}

\begin{table}[!htbp] \centering 
  \caption{Régressions linéaires supplémentaires de l'analyse 4 :} 
  \label{} 
\small 
\begin{tabular}{@{\extracolsep{1pt}}lccc} 
\\[-1.8ex]\hline 
\hline \\[-1.8ex] 
 & \multicolumn{3}{c}{\textit{Dependent variable:}} \\ 
\cline{2-4} 
\\[-1.8ex] & \multicolumn{3}{c}{log(price)} \\ 
\\[-1.8ex] & (1) & (2) & (3)\\ 
\hline \\[-1.8ex] 
 condition2 & $-$0.075$^{*}$ & $-$0.067 & $-$0.036 \\ 
  & (0.042) & (0.064) & (0.067) \\ 
  condition3 & $-$0.055 & $-$0.037 & 0.015 \\ 
  & (0.038) & (0.060) & (0.062) \\ 
  condition4 & $-$0.051 & $-$0.036 & 0.025 \\ 
  & (0.038) & (0.060) & (0.062) \\ 
  condition5 & $-$0.040 & $-$0.015 & 0.053 \\ 
  & (0.038) & (0.060) & (0.062) \\ 
  Constant & 12.911$^{***}$ & 13.223$^{***}$ & 13.004$^{***}$ \\ 
  & (0.038) & (0.060) & (0.062) \\ 
 \hline \\[-1.8ex] 
Observations & 5,461 & 5,548 & 10,837 \\ 
R$^{2}$ & 0.002 & 0.003 & 0.003 \\ 
Adjusted R$^{2}$ & 0.001 & 0.003 & 0.002 \\ 
Residual Std. Error & 0.100 (df = 5456) & 0.103 (df = 5543) & 0.195 (df = 10832) \\ 
F Statistic & 2.590$^{**}$ (df = 4; 5456) & 4.740$^{***}$ (df = 4; 5543) & 7.714$^{***}$ (df = 4; 10832) \\ 
\hline 
\hline \\[-1.8ex] 
\textit{Note:}  & \multicolumn{3}{r}{$^{*}$p$<$0.1; $^{**}$p$<$0.05; $^{***}$p$<$0.01} \\ 
\end{tabular} 
\end{table}

\begin{table}[!h]

\caption{\label{tab:unnamed-chunk-42}Comparaison des performances des régressions linéaires supplémentaires de l'analyse 4}
\centering
\resizebox{\linewidth}{!}{
\begin{tabular}[t]{llrrrrrrr}
\toprule
Name & Model & AIC & BIC & R2 & R2\_adjusted & RMSE & Sigma & Performance\_Score\\
\midrule
\cellcolor{gray!6}{model1} & \cellcolor{gray!6}{lm} & \cellcolor{gray!6}{-1155.893} & \cellcolor{gray!6}{-1116.326} & \cellcolor{gray!6}{0.023} & \cellcolor{gray!6}{0.022} & \cellcolor{gray!6}{0.217} & \cellcolor{gray!6}{0.217} & \cellcolor{gray!6}{0.713}\\
model5 & lm & -9419.085 & -9379.357 & 0.003 & 0.003 & 0.103 & 0.103 & 0.683\\
\cellcolor{gray!6}{model4} & \cellcolor{gray!6}{lm} & \cellcolor{gray!6}{-9648.516} & \cellcolor{gray!6}{-9608.884} & \cellcolor{gray!6}{0.002} & \cellcolor{gray!6}{0.001} & \cellcolor{gray!6}{0.100} & \cellcolor{gray!6}{0.100} & \cellcolor{gray!6}{0.667}\\
model2 & lm & 2962.730 & 3006.488 & 0.019 & 0.019 & 0.277 & 0.277 & 0.517\\
\cellcolor{gray!6}{model6} & \cellcolor{gray!6}{lm} & \cellcolor{gray!6}{-4624.939} & \cellcolor{gray!6}{-4581.194} & \cellcolor{gray!6}{0.003} & \cellcolor{gray!6}{0.002} & \cellcolor{gray!6}{0.195} & \cellcolor{gray!6}{0.195} & \cellcolor{gray!6}{0.480}\\
\addlinespace
model3 & lm & 12017.053 & 12063.224 & 0.014 & 0.014 & 0.350 & 0.350 & 0.202\\
\bottomrule
\end{tabular}}
\end{table}

\newpage

\begin{table}[!h]

\caption{\label{tab:unnamed-chunk-43}Tests de l'analyse des résidus du modèle 1 :}
\centering
\begin{tabular}[t]{lrrr}
\toprule
  & Harrison\_McCabe & Breusch\_Pagan & Durbin\_Watson\\
\midrule
\cellcolor{gray!6}{p-value} & \cellcolor{gray!6}{0.048} & \cellcolor{gray!6}{0.000} & \cellcolor{gray!6}{0.006}\\
statistic & 0.484 & 64.779 & 1.932\\
\bottomrule
\end{tabular}
\end{table}

\begin{table}[!h]

\caption{\label{tab:unnamed-chunk-43}Tests de l'analyse des résidus du modèle 2 :}
\centering
\begin{tabular}[t]{lrrr}
\toprule
  & Harrison\_McCabe & Breusch\_Pagan & Durbin\_Watson\\
\midrule
\cellcolor{gray!6}{p-value} & \cellcolor{gray!6}{0.812} & \cellcolor{gray!6}{0.000} & \cellcolor{gray!6}{0.065}\\
statistic & 0.506 & 84.047 & 1.971\\
\bottomrule
\end{tabular}
\end{table}

\begin{table}[!h]

\caption{\label{tab:unnamed-chunk-43}Tests de l'analyse des résidus du modèle 3 :}
\centering
\begin{tabular}[t]{lrrr}
\toprule
  & Harrison\_McCabe & Breusch\_Pagan & Durbin\_Watson\\
\midrule
\cellcolor{gray!6}{p-value} & \cellcolor{gray!6}{0.893} & \cellcolor{gray!6}{0.000} & \cellcolor{gray!6}{0.027}\\
statistic & 0.507 & 81.606 & 1.970\\
\bottomrule
\end{tabular}
\end{table}

\begin{table}[!h]

\caption{\label{tab:unnamed-chunk-43}Tests de l'analyse des résidus du modèle 4 :}
\centering
\begin{tabular}[t]{lrrr}
\toprule
  & Harrison\_McCabe & Breusch\_Pagan & Durbin\_Watson\\
\midrule
\cellcolor{gray!6}{p-value} & \cellcolor{gray!6}{0.713} & \cellcolor{gray!6}{0.055} & \cellcolor{gray!6}{0.326}\\
statistic & 0.505 & 9.260 & 1.988\\
\bottomrule
\end{tabular}
\end{table}

\begin{table}[!h]

\caption{\label{tab:unnamed-chunk-43}Tests de l'analyse des résidus du modèle 5 :}
\centering
\begin{tabular}[t]{lrrr}
\toprule
  & Harrison\_McCabe & Breusch\_Pagan & Durbin\_Watson\\
\midrule
\cellcolor{gray!6}{p-value} & \cellcolor{gray!6}{0.441} & \cellcolor{gray!6}{0.048} & \cellcolor{gray!6}{0.900}\\
statistic & 0.499 & 9.609 & 2.034\\
\bottomrule
\end{tabular}
\end{table}

\begin{table}[!h]

\caption{\label{tab:unnamed-chunk-43}Tests de l'analyse des résidus du modèle 6 :}
\centering
\begin{tabular}[t]{lrrr}
\toprule
  & Harrison\_McCabe & Breusch\_Pagan & Durbin\_Watson\\
\midrule
\cellcolor{gray!6}{p-value} & \cellcolor{gray!6}{0.224} & \cellcolor{gray!6}{0.862} & \cellcolor{gray!6}{0.449}\\
statistic & 0.495 & 1.297 & 1.998\\
\bottomrule
\end{tabular}
\end{table}

\newpage

\begin{figure}
\centering
\includegraphics[width=1\textwidth,height=\textheight]{plot (12).png}
\caption{Graphiques de l'analyse des résidus}
\end{figure}

En conclusion, nous conservons les hypothèses d'absence
d'autocorrélation et nous en déduisons que les résidus ne sont pas
autocorrélés au seuil de 0,1\% dans les modèles 1, 2, 3, 4, 5 et 6. De
manière biaisé par les graphiques de l'analyse des résidus qui ne sont
pas bons, nous avons 1\% de chance de se tromper en affirmant que plus
les logements sont en bon état et plus le prix de ces derniers est élevé
selon la relation \(\mathcal{M}_{1}^{A}\).

\newpage

\hypertarget{mathcala_5-lanalyse-du-prix-dun-logement-en-fonction-de-son-uxe2ge-de-construction-et-de-son-uxe2ge-de-ruxe9novation}{%
\section{\texorpdfstring{\(\mathcal{A}_{5}\) : l'analyse du prix d'un
logement en fonction de son âge de construction et de son âge de
rénovation}{\textbackslash mathcal\{A\}\_\{5\} : l'analyse du prix d'un logement en fonction de son âge de construction et de son âge de rénovation}}\label{mathcala_5-lanalyse-du-prix-dun-logement-en-fonction-de-son-uxe2ge-de-construction-et-de-son-uxe2ge-de-ruxe9novation}}

\begin{table}[!htbp] \centering 
  \caption{Régressions linéaires de l'analyse 5 :} 
  \label{} 
\small 
\begin{tabular}{@{\extracolsep{1pt}}lcc} 
\\[-1.8ex]\hline 
\hline \\[-1.8ex] 
 & \multicolumn{2}{c}{\textit{Dependent variable:}} \\ 
\cline{2-3} 
\\[-1.8ex] & \multicolumn{2}{c}{price} \\ 
\\[-1.8ex] & (1) & (2)\\ 
\hline \\[-1.8ex] 
 life\_built & $-$674.369$^{***}$ & $-$429.877$^{***}$ \\ 
  & (84.893) & (131.956) \\ 
  life\_renovated & $-$65.659 & 510.952$^{*}$ \\ 
  & (160.771) & (287.429) \\ 
  life\_built:life\_renovated &  & $-$13.282$^{**}$ \\ 
  &  & (5.488) \\ 
  Constant & 570,514.500$^{***}$ & 559,881.500$^{***}$ \\ 
  & (5,352.348) & (6,924.413) \\ 
 \hline \\[-1.8ex] 
Observations & 21,613 & 21,613 \\ 
R$^{2}$ & 0.003 & 0.003 \\ 
Adjusted R$^{2}$ & 0.003 & 0.003 \\ 
Residual Std. Error & 366,608.100 (df = 21610) & 366,566.900 (df = 21609) \\ 
F Statistic & 31.625$^{***}$ (df = 2; 21610) & 23.040$^{***}$ (df = 3; 21609) \\ 
\hline 
\hline \\[-1.8ex] 
\textit{Note:}  & \multicolumn{2}{r}{$^{*}$p$<$0.1; $^{**}$p$<$0.05; $^{***}$p$<$0.01} \\ 
\end{tabular} 
\end{table}

\begin{table}[!htbp] \centering 
  \caption{Régressions linéaires de l'analyse 5 :} 
  \label{} 
\small 
\begin{tabular}{@{\extracolsep{1pt}}lcc} 
\\[-1.8ex]\hline 
\hline \\[-1.8ex] 
 & \multicolumn{2}{c}{\textit{Dependent variable:}} \\ 
\cline{2-3} 
\\[-1.8ex] & \multicolumn{2}{c}{log(price)} \\ 
\\[-1.8ex] & (1) & (2)\\ 
\hline \\[-1.8ex] 
 life\_built & $-$0.001$^{***}$ & $-$0.001$^{***}$ \\ 
  & (0.0001) & (0.0002) \\ 
  life\_renovated & $-$0.0002 & 0.0004 \\ 
  & (0.0002) & (0.0004) \\ 
  life\_built:life\_renovated &  & $-$0.00001 \\ 
  &  & (0.00001) \\ 
  Constant & 13.113$^{***}$ & 13.103$^{***}$ \\ 
  & (0.008) & (0.010) \\ 
 \hline \\[-1.8ex] 
Observations & 21,613 & 21,613 \\ 
R$^{2}$ & 0.007 & 0.007 \\ 
Adjusted R$^{2}$ & 0.006 & 0.006 \\ 
Residual Std. Error & 0.525 (df = 21610) & 0.525 (df = 21609) \\ 
F Statistic & 70.822$^{***}$ (df = 2; 21610) & 48.048$^{***}$ (df = 3; 21609) \\ 
\hline 
\hline \\[-1.8ex] 
\textit{Note:}  & \multicolumn{2}{r}{$^{*}$p$<$0.1; $^{**}$p$<$0.05; $^{***}$p$<$0.01} \\ 
\end{tabular} 
\end{table}

\newpage

\hypertarget{analyse-des-ruxe9sultats-5}{%
\subsection{Analyse des résultats}\label{analyse-des-ruxe9sultats-5}}

\begin{table}[!h]

\caption{\label{tab:unnamed-chunk-45}Aperçu de la base de données de l'analyse 5 :}
\centering
\begin{tabular}[t]{lrrr}
\toprule
  & price & life\_built & life\_renovated\\
\midrule
\cellcolor{gray!6}{18130} & \cellcolor{gray!6}{382880} & \cellcolor{gray!6}{47} & \cellcolor{gray!6}{13}\\
21026 & 727000 & 5 & 4\\
\cellcolor{gray!6}{7946} & \cellcolor{gray!6}{237000} & \cellcolor{gray!6}{56} & \cellcolor{gray!6}{1}\\
5269 & 495000 & 75 & 9\\
\cellcolor{gray!6}{10116} & \cellcolor{gray!6}{555000} & \cellcolor{gray!6}{10} & \cellcolor{gray!6}{24}\\
\addlinespace
5396 & 299800 & 38 & 11\\
\cellcolor{gray!6}{1569} & \cellcolor{gray!6}{789500} & \cellcolor{gray!6}{9} & \cellcolor{gray!6}{17}\\
1407 & 300000 & 15 & 18\\
\cellcolor{gray!6}{2236} & \cellcolor{gray!6}{220000} & \cellcolor{gray!6}{41} & \cellcolor{gray!6}{0}\\
1682 & 1230000 & 13 & 32\\
\bottomrule
\end{tabular}
\end{table}

\begin{table}[!h]

\caption{\label{tab:unnamed-chunk-46}Comparaison des performances des régressions linéaires de l'analyse 5}
\centering
\resizebox{\linewidth}{!}{
\begin{tabular}[t]{llrrrrrrr}
\toprule
Name & Model & AIC & BIC & R2 & R2\_adjusted & RMSE & Sigma & Performance\_Score\\
\midrule
\cellcolor{gray!6}{model4} & \cellcolor{gray!6}{lm} & \cellcolor{gray!6}{33485.84} & \cellcolor{gray!6}{33525.75} & \cellcolor{gray!6}{0.007} & \cellcolor{gray!6}{0.006} & \cellcolor{gray!6}{0.525} & \cellcolor{gray!6}{0.525} & \cellcolor{gray!6}{1.000}\\
model3 & lm & 33486.33 & 33518.26 & 0.007 & 0.006 & 0.525 & 0.525 & 0.992\\
\cellcolor{gray!6}{model2} & \cellcolor{gray!6}{lm} & \cellcolor{gray!6}{615149.79} & \cellcolor{gray!6}{615189.70} & \cellcolor{gray!6}{0.003} & \cellcolor{gray!6}{0.003} & \cellcolor{gray!6}{366532.952} & \cellcolor{gray!6}{366566.875} & \cellcolor{gray!6}{0.022}\\
model1 & lm & 615153.65 & 615185.57 & 0.003 & 0.003 & 366582.615 & 366608.060 & 0.000\\
\bottomrule
\end{tabular}}
\end{table}

Nous allons analyser le modèle 4.

\[\mathcal{M}_{4}:\log(y_{i})=\alpha+\beta_{22}x_{i,22}+\beta_{23}x_{i,23}+\beta_{22.23}x_{i,22}x_{i,23}+z_{i}\]

Pour \(\alpha\), nous avons moins de 0,1\% de chance de se tromper en
rejetant l'hypothèse nulle. Par conséquent, nous rejetons l'hypothèse
nulle et nous en déduisons que la constante du modèle 4 est différente
de 0 au seuil de 0,1\%.

Pour \(\beta_{22}\), nous avons moins de 0,1\% de chance de se tromper
en rejetant l'hypothèse nulle. Par conséquent, nous rejetons l'hypothèse
nulle et nous en déduisons que le coefficient du modèle 4 de la variable
explicative \texttt{life\_built} est différent de 0 au seuil de 0,1\%.

Pour \(\beta_{23}\), nous avons 34,6\% de chance de se tromper en
rejetant l'hypothèse nulle. Par conséquent, nous conservons l'hypothèse
nulle et nous en déduisons que le coefficient du modèle 4 de la variable
explicative \texttt{life\_renovated} est égale à 0 au seuil de 0,1\%.

Pour \(\beta_{22.23}\), nous avons 11,5\% de chance de se tromper en
rejetant l'hypothèse nulle. Par conséquent, nous coonservons l'hypothèse
nulle et nous en déduisons que le coefficient du modèle 4 de la variable
explicative \texttt{life\_built.life\_renovated} est égale à 0 au seuil
de 0,1\%.

Nous avons \(R^{2}=0,007\) dans le modèle 4 donc 0,7\% de la variation
du prix des logements est expliquée par la surface intérieure de ces
derniers.

Pour \(F\), nous avons moins de 0,1\% de chance de se tromper en
rejetant l'hypothèse d'absence globale de significativité. Par
conséquent, nous rejetons l'hypothèse d'absence globale de
significativité et nous en déduisons que le modèle 4 est globalement
avéré au seuil de 0,1\%.

\newpage

\hypertarget{analyse-des-ruxe9sidus-5}{%
\subsection{Analyse des résidus}\label{analyse-des-ruxe9sidus-5}}

\begin{center}\includegraphics{E2-L3S2--1-_files/figure-latex/data.figure11-1} \end{center}

\begin{figure}
\centering
\includegraphics[width=1\textwidth,height=\textheight]{plot (9).png}
\caption{Graphiques de l'analyse des résidus}
\end{figure}

\newpage

\begin{table}[!h]

\caption{\label{tab:unnamed-chunk-47}Tests de l'analyse des résidus :}
\centering
\begin{tabular}[t]{lrrr}
\toprule
  & Harrison\_McCabe & Breusch\_Pagan & Durbin\_Watson\\
\midrule
\cellcolor{gray!6}{p-value} & \cellcolor{gray!6}{0.458} & \cellcolor{gray!6}{0.000} & \cellcolor{gray!6}{0.005}\\
statistic & 0.500 & 81.297 & 1.965\\
\bottomrule
\end{tabular}
\end{table}

Pour \(DW\) nous avons 0,5\% de chance de se tromper en rejetant
l'hypothèse d'absence d'autocorrélation. Par conséquent, nous conservons
l'hypothèse d'absence d'autocorrélation et nous en déduisons que les
résidus ne sont pas autocorrélés au seuil de 0,1\%.

Nous observons que les résidus sont globalement répartis le long de la
droite. Néanmoins nous en déduisons que les résidus ne suivent pas une
loi normale d'après le graphique représentant la densité des résidus.

Pour \(BP\), nous avons moins de 0,1\% de chance de se tomper en
rejetant l'hypothèse d'homoscédasticité. Par conséquent, nous rejetons
l'hypothèse d'homoscédasticité et nous en déduisons que les résidus sont
hétéroscédastiques au seuil de 0,1\%.

Nous observons l'existence de résidus qui ne sont globalement pas
linéaires et qui tendent vers la distance de Cook. Nous en déduisons
que, dans une moindre mesure, les logement 18483, 9255, 7253, 3915 et
3736 ont un résidu aberrant.

Les hypothèses nécessaires des résultats et des résidus sont avérés au
seuil de 0,1\%. Notre constante, notre coefficient, notre
significativité, notre linéarité des résidus, notre absence
d'autocorrélation des résidus et notre appartenance à une loi normale
des résidus sont avérés au seuil de 0,1\%. L'hypothèse
d'hétéroscédasticité ne pose pas de problème car le prix des logements
et la variance du prix des logements ne sont pas homogènes.

En conclusion, nous avons 0,1\% de chance de se tromper en affirmant que
plus les logements sont récents et plus le prix de ces derniers est
élevé selon la relation \(\mathcal{M}_{4}\).

\begin{center}\includegraphics{E2-L3S2--1-_files/figure-latex/data.figure14-1} \end{center}

\newpage

\hypertarget{mathcala_6-le-prix-dun-logement-dans-la-ville-de-seattle-dans-luxe9tat-de-washington-aux-uxe9tats-unis}{%
\section{\texorpdfstring{\(\mathcal{A}_{6}\) : le prix d'un logement
dans la ville de Seattle dans l'État de Washington aux
États-Unis}{\textbackslash mathcal\{A\}\_\{6\} : le prix d'un logement dans la ville de Seattle dans l'État de Washington aux États-Unis}}\label{mathcala_6-le-prix-dun-logement-dans-la-ville-de-seattle-dans-luxe9tat-de-washington-aux-uxe9tats-unis}}

Nous allons segmenter la base de données \texttt{BDD\_data.csv} en ne
prenant en compte que les logements coûtant moins de 645 000 dollars.

\begin{Shaded}
\begin{Highlighting}[]
\NormalTok{data.analyse6 <-}\StringTok{ }\NormalTok{data[ data}\OperatorTok{$}\NormalTok{price }\OperatorTok{<=}\StringTok{ }\DecValTok{645000}\NormalTok{, ]}
\end{Highlighting}
\end{Shaded}

\[\mathcal{A}_{6}=\{\mathcal{M}_{1}\}\]

\begin{table}[!htbp] \centering 
  \caption{Régressions linéaires de l'analyse 6 :} 
  \label{} 
\small 
\begin{tabular}{@{\extracolsep{1pt}}lcc} 
\\[-1.8ex]\hline 
\hline \\[-1.8ex] 
 & \multicolumn{2}{c}{\textit{Dependent variable:}} \\ 
\cline{2-3} 
\\[-1.8ex] & \multicolumn{2}{c}{log(price)} \\ 
\\[-1.8ex] & (1) & (2)\\ 
\hline \\[-1.8ex] 
 log(sqft\_living) & 0.829$^{***}$ & 0.460$^{***}$ \\ 
  & (0.010) & (0.010) \\ 
  bedrooms & $-$0.089$^{***}$ & $-$0.055$^{***}$ \\ 
  & (0.003) & (0.003) \\ 
  bathrooms & 0.137$^{***}$ & 0.089$^{***}$ \\ 
  & (0.006) & (0.006) \\ 
  floors & 0.134$^{***}$ & 0.109$^{***}$ \\ 
  & (0.006) & (0.006) \\ 
  condition2 & $-$0.002 & 0.064 \\ 
  & (0.071) & (0.062) \\ 
  condition3 & 0.204$^{***}$ & 0.297$^{***}$ \\ 
  & (0.066) & (0.057) \\ 
  condition4 & 0.210$^{***}$ & 0.290$^{***}$ \\ 
  & (0.066) & (0.057) \\ 
  condition5 & 0.259$^{***}$ & 0.324$^{***}$ \\ 
  & (0.066) & (0.058) \\ 
  life\_built & 0.005$^{***}$ & 0.003$^{***}$ \\ 
  & (0.0001) & (0.0001) \\ 
  Constant & 6.160$^{***}$ & 8.823$^{***}$ \\ 
  & (0.091) & (0.087) \\ 
 \hline \\[-1.8ex] 
Observations & 21,613 & 16,240 \\ 
R$^{2}$ & 0.532 & 0.273 \\ 
Adjusted R$^{2}$ & 0.532 & 0.273 \\ 
Residual Std. Error & 0.360 (df = 21603) & 0.301 (df = 16230) \\ 
F Statistic & 2,730.721$^{***}$ (df = 9; 21603) & 678.560$^{***}$ (df = 9; 16230) \\ 
\hline 
\hline \\[-1.8ex] 
\textit{Note:}  & \multicolumn{2}{r}{$^{*}$p$<$0.1; $^{**}$p$<$0.05; $^{***}$p$<$0.01} \\ 
\end{tabular} 
\end{table}

Les régressions linéaires de l'analyse 6 ne sont pas plus intéressantes
que les régressions linéaires des analyses 1, 2, 3, 4 et 5.

\hypertarget{synthuxe8se}{%
\section{Synthèse}\label{synthuxe8se}}

Vous devez vous référer à l'annexe
\texttt{DevoirEconometrieLubinVialatteAnnexe\ (2).pdf}.

\hypertarget{tools}{%
\section{\texorpdfstring{\texttt{tools}}{tools}}\label{tools}}

Nous nous sommes servis de ces \texttt{tools} :

\begin{Shaded}
\begin{Highlighting}[]
\NormalTok{tool.table <-}\StringTok{ }\ControlFlowTok{function}\NormalTok{( b, c, d )\{}
\NormalTok{  a <-}\StringTok{ }\NormalTok{b }\OperatorTok
\StringTok{    }\KeywordTok{kable}\NormalTok{( }\DataTypeTok{caption =}\NormalTok{ c,}
           \DataTypeTok{digits =} \DecValTok{3}\NormalTok{,}
           \DataTypeTok{booktabs =}\NormalTok{ T ) }\OperatorTok
\StringTok{    }\KeywordTok{kable_styling}\NormalTok{( }\DataTypeTok{full_width =}\NormalTok{ F,}
                   \DataTypeTok{position =} \StringTok{"center"}\NormalTok{,}
                   \DataTypeTok{latex_options =} \KeywordTok{c}\NormalTok{( }\StringTok{"striped"}\NormalTok{,}
                                      \StringTok{"condensed"}\NormalTok{,}
                                      \StringTok{"hold_position"}\NormalTok{,}
\NormalTok{                                      d ) )}
  \KeywordTok{return}\NormalTok{( a )}
\NormalTok{\}}\CommentTok{# pour les tableaux,}
\CommentTok{# a = ( b = data, c = caption, d = "scale_down" ),}

\NormalTok{tool.summary <-}\StringTok{ }\ControlFlowTok{function}\NormalTok{( b )\{}
\NormalTok{  a <-}\StringTok{ }\KeywordTok{c}\NormalTok{( }\KeywordTok{mean}\NormalTok{( b,}
                \DataTypeTok{na.rm =}\NormalTok{ T ),}
          \KeywordTok{sd}\NormalTok{( b,}
              \DataTypeTok{na.rm =}\NormalTok{ T ),}
          \KeywordTok{quantile}\NormalTok{( b,}
                    \DataTypeTok{na.rm =}\NormalTok{ T ) )}
\NormalTok{  a <-}\StringTok{ }\KeywordTok{round}\NormalTok{( a,}
              \DecValTok{3}\NormalTok{ )}
  \KeywordTok{names}\NormalTok{( a ) <-}\StringTok{ }\KeywordTok{c}\NormalTok{( }\StringTok{"Moyenne"}\NormalTok{,}
                   \StringTok{"Ecart-type"}\NormalTok{,}
                   \StringTok{"Minimum"}\NormalTok{,}
                   \StringTok{"Q1"}\NormalTok{,}
                   \StringTok{"Q2"}\NormalTok{,}
                   \StringTok{"Q3"}\NormalTok{,}
                   \StringTok{"Maximum"}\NormalTok{ )}
  \KeywordTok{return}\NormalTok{( a )}
\NormalTok{\}}\CommentTok{# pour les résumés des variables,}
\CommentTok{# a = ( b = data ),}

\NormalTok{tool.durbin <-}\StringTok{ }\ControlFlowTok{function}\NormalTok{( b )\{}
\NormalTok{  test1 <-}\StringTok{ }\KeywordTok{dwtest}\NormalTok{( b )}
\NormalTok{  a <-}\StringTok{ }\KeywordTok{c}\NormalTok{( test1}\OperatorTok{$}\NormalTok{p.value,}
\NormalTok{          test1}\OperatorTok{$}\NormalTok{statistic )}
\NormalTok{  a <-}\StringTok{ }\KeywordTok{round}\NormalTok{( a,}
              \DecValTok{3}\NormalTok{ )}
  \KeywordTok{names}\NormalTok{( a ) <-}\StringTok{ }\KeywordTok{c}\NormalTok{( }\StringTok{"p.value"}\NormalTok{,}
                   \StringTok{"DW"}\NormalTok{ )}
  \KeywordTok{return}\NormalTok{ ( a )}
\NormalTok{\}}\CommentTok{# pour les tests de Durbin-Watson,}
\CommentTok{# a = ( b = data ),}

\NormalTok{tool.shapiro <-}\StringTok{ }\ControlFlowTok{function}\NormalTok{( b )\{}
\NormalTok{  test1 <-}\StringTok{ }\KeywordTok{shapiro.test}\NormalTok{( b )}
\NormalTok{  a <-}\StringTok{ }\KeywordTok{c}\NormalTok{( test1}\OperatorTok{$}\NormalTok{p.value,}
\NormalTok{          test1}\OperatorTok{$}\NormalTok{statistic )}
\NormalTok{  a <-}\StringTok{ }\KeywordTok{round}\NormalTok{( a,}
              \DecValTok{3}\NormalTok{ )}
  \KeywordTok{names}\NormalTok{( a ) <-}\StringTok{ }\KeywordTok{c}\NormalTok{( }\StringTok{"p.value"}\NormalTok{,}
                   \StringTok{"W"}\NormalTok{ )}
  \KeywordTok{return}\NormalTok{( a )}
\NormalTok{\}}\CommentTok{# pour les tests de Shapiro-Wilk,}
\CommentTok{# a = ( b = data ),}

\NormalTok{tool.breusch <-}\StringTok{ }\ControlFlowTok{function}\NormalTok{( b )\{}
\NormalTok{  test1 <-}\StringTok{ }\KeywordTok{bptest}\NormalTok{( b )}
\NormalTok{  a <-}\StringTok{ }\KeywordTok{c}\NormalTok{( test1}\OperatorTok{$}\NormalTok{p.value,}
\NormalTok{          test1}\OperatorTok{$}\NormalTok{statistic )}
\NormalTok{  a <-}\StringTok{ }\KeywordTok{round}\NormalTok{( a,}
              \DecValTok{3}\NormalTok{ )}
  \KeywordTok{names}\NormalTok{( a ) <-}\StringTok{ }\KeywordTok{c}\NormalTok{( }\StringTok{"p.value"}\NormalTok{,}
                   \StringTok{"BP"}\NormalTok{ )}
  \KeywordTok{return}\NormalTok{ ( a )}
\NormalTok{\}}\CommentTok{# pour les tests de Breusch-Pagan,}
\CommentTok{# a = ( b = data ),}

\NormalTok{tool.harrison <-}\StringTok{ }\ControlFlowTok{function}\NormalTok{( b )\{}
\NormalTok{  test1 <-}\StringTok{ }\KeywordTok{hmctest}\NormalTok{( b )}
\NormalTok{  a <-}\StringTok{ }\KeywordTok{c}\NormalTok{( test1}\OperatorTok{$}\NormalTok{p.value,}
\NormalTok{          test1}\OperatorTok{$}\NormalTok{statistic )}
\NormalTok{  a <-}\StringTok{ }\KeywordTok{round}\NormalTok{( a,}
              \DecValTok{3}\NormalTok{ )}
  \KeywordTok{names}\NormalTok{( a ) <-}\StringTok{ }\KeywordTok{c}\NormalTok{( }\StringTok{"p.value"}\NormalTok{,}
                   \StringTok{"HMC"}\NormalTok{ )}
  \KeywordTok{return}\NormalTok{ ( a )}
\NormalTok{\}}\CommentTok{# pour les tests de Harrison-McCabe,}
\CommentTok{# a = ( b = data ),}

\NormalTok{tool.student <-}\StringTok{ }\ControlFlowTok{function}\NormalTok{( b, c )\{}
\NormalTok{ test1 <-}\StringTok{ }\KeywordTok{var.test}\NormalTok{( b }\OperatorTok{~}\StringTok{ }\NormalTok{c )}
\NormalTok{ test2 <-}\StringTok{ }\KeywordTok{t.test}\NormalTok{( b }\OperatorTok{~}\StringTok{ }\NormalTok{c,}
                  \DataTypeTok{equal =}\NormalTok{ test1}\OperatorTok{$}\NormalTok{p.value }\OperatorTok{>}\StringTok{ }\FloatTok{0.05}\NormalTok{ )}
\NormalTok{ a <-}\StringTok{ }\KeywordTok{c}\NormalTok{( }\KeywordTok{table}\NormalTok{( c[ }\OperatorTok{!}\KeywordTok{is.na}\NormalTok{( b ) ] ),}
         \KeywordTok{ifelse}\NormalTok{( test2}\OperatorTok{$}\NormalTok{p.value }\OperatorTok{>}\StringTok{ }\FloatTok{0.05}\NormalTok{,}
                 \StringTok{"Oui"}\NormalTok{,}
                 \StringTok{"Non"}\NormalTok{ ),}
         \KeywordTok{round}\NormalTok{( }\KeywordTok{c}\NormalTok{( test2}\OperatorTok{$}\NormalTok{estimate,}
\NormalTok{                   test2}\OperatorTok{$}\NormalTok{p.value ),}
                \DecValTok{3}\NormalTok{ ) )}
 \KeywordTok{names}\NormalTok{( a ) <-}\StringTok{ }\KeywordTok{c}\NormalTok{( }\KeywordTok{names}\NormalTok{( }\KeywordTok{table}\NormalTok{( c[ }\OperatorTok{!}\KeywordTok{is.na}\NormalTok{(b) ] ) ),}
                  \StringTok{"var.equal"}\NormalTok{,}
                  \KeywordTok{names}\NormalTok{( test2}\OperatorTok{$}\NormalTok{estimate ),}
                  \StringTok{"p.value"}\NormalTok{ )}
 \KeywordTok{return}\NormalTok{( a )}
\NormalTok{\}}\CommentTok{# pour les tests de Student de comparaison des moyennes et des variances,}
\CommentTok{# a = ( b = data1, c = data2 ),}

\NormalTok{tool.chi2 <-}\StringTok{ }\ControlFlowTok{function}\NormalTok{( b, c )\{}
\NormalTok{  test1 <-}\StringTok{ }\KeywordTok{chisq.test}\NormalTok{( b,}
\NormalTok{                       c )}
\NormalTok{  a <-}\StringTok{ }\KeywordTok{c}\NormalTok{( }\KeywordTok{min}\NormalTok{( test1}\OperatorTok{$}\NormalTok{expected ),}
\NormalTok{          test1}\OperatorTok{$}\NormalTok{p.value )}
\NormalTok{  a <-}\StringTok{ }\KeywordTok{round}\NormalTok{( a,}
              \DecValTok{3}\NormalTok{ )}
  \KeywordTok{names}\NormalTok{( a ) <-}\StringTok{ }\KeywordTok{c}\NormalTok{( }\StringTok{"Eff_théorique_min"}\NormalTok{,}
                   \StringTok{"p-value"}\NormalTok{ )}
  \KeywordTok{return}\NormalTok{( a )}
\NormalTok{\}}\CommentTok{# pour les tests du Chi 2 d'indépendance des variables,}
\CommentTok{# a = ( b = data1, c = data2 ),}
\end{Highlighting}
\end{Shaded}

\hypertarget{packages}{%
\section{\texorpdfstring{\texttt{packages}}{packages}}\label{packages}}

Nous nous sommes servis de ces \texttt{packages} :

\begin{Shaded}
\begin{Highlighting}[]
\KeywordTok{library}\NormalTok{( readxl )}\CommentTok{# pour la base de données,}
\KeywordTok{library}\NormalTok{( dplyr )}\CommentTok{# pour la syntaxe,}
\KeywordTok{library}\NormalTok{( tidyverse )}
\KeywordTok{library}\NormalTok{( forcats )}\CommentTok{# pour les vecteurs,}
\KeywordTok{library}\NormalTok{( lmtest )}\CommentTok{# pour les tests,}
\KeywordTok{library}\NormalTok{( performance )}
\KeywordTok{library}\NormalTok{( see )}\CommentTok{# pour le package performance,}
\KeywordTok{library}\NormalTok{( qqplotr )}\CommentTok{# pour le package performance,}
\KeywordTok{library}\NormalTok{( maps )}\CommentTok{# pour les cartes,}
\KeywordTok{library}\NormalTok{( mapdata )}\CommentTok{# pour les cartes,}
\KeywordTok{library}\NormalTok{( stringr )}
\KeywordTok{library}\NormalTok{( viridis )}

\KeywordTok{library}\NormalTok{( plm )}\CommentTok{# pour les régressions,}
\KeywordTok{library}\NormalTok{( car )}\CommentTok{# Pour les régressions,}
\KeywordTok{library}\NormalTok{( carData )}\CommentTok{# Pour les régressions,}
\KeywordTok{library}\NormalTok{( stargazer )}\CommentTok{# pour les régressions,}
\KeywordTok{library}\NormalTok{( lmtest )}\CommentTok{# pour les régressions,}
\KeywordTok{library}\NormalTok{( statsr )}\CommentTok{# pour les régressions,}
\CommentTok{# library( summarytools )# pour les régressions,}

\KeywordTok{library}\NormalTok{( printr )}\CommentTok{# pour les tableaux,}
\KeywordTok{library}\NormalTok{( knitr )}\CommentTok{# pour les tableaux,}
\KeywordTok{library}\NormalTok{( kableExtra )}\CommentTok{# pour les tableaux,}
\KeywordTok{library}\NormalTok{( modelsummary )}\CommentTok{# pour les tableaux,}
\KeywordTok{library}\NormalTok{( gtsummary )}\CommentTok{# pour les tableaux,}

\CommentTok{# library( ggplot1 )# pour les graphiques,}
\KeywordTok{library}\NormalTok{( ggplot2 )}\CommentTok{# pour les graphiques,}
\KeywordTok{library}\NormalTok{( ggcorrplot )}\CommentTok{# pour les graphiques,}
\KeywordTok{library}\NormalTok{( ggfortify )}\CommentTok{# pour les graphiques,}
\KeywordTok{library}\NormalTok{( ggpubr )}\CommentTok{# pour les graphiques,}
\KeywordTok{library}\NormalTok{( ggrepel )}\CommentTok{# pour les graphiques,}
\KeywordTok{library}\NormalTok{( ggridges )}\CommentTok{# pour les graphiques,}
\KeywordTok{library}\NormalTok{( ggsci )}\CommentTok{# pour les graphiques,}
\KeywordTok{library}\NormalTok{( ggsignif )}\CommentTok{# pour les graphiques,}
\end{Highlighting}
\end{Shaded}

\end{document}
